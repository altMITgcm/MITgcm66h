%% Generated by Sphinx.
\def\sphinxdocclass{report}
\documentclass[letterpaper,10pt,english]{sphinxmanual}
\ifdefined\pdfpxdimen
   \let\sphinxpxdimen\pdfpxdimen\else\newdimen\sphinxpxdimen
\fi \sphinxpxdimen=.75bp\relax

\usepackage[utf8]{inputenc}
\ifdefined\DeclareUnicodeCharacter
 \ifdefined\DeclareUnicodeCharacterAsOptional\else
  \DeclareUnicodeCharacter{00A0}{\nobreakspace}
\fi\fi
\usepackage{cmap}
\usepackage[T1]{fontenc}
\usepackage{amsmath,amssymb,amstext}
\usepackage{babel}
\usepackage{times}
\usepackage[Bjarne]{fncychap}
\usepackage{longtable}
\usepackage{sphinx}

\usepackage{geometry}
\usepackage{multirow}
\usepackage{eqparbox}

% Include hyperref last.
\usepackage{hyperref}
% Fix anchor placement for figures with captions.
\usepackage{hypcap}% it must be loaded after hyperref.
% Set up styles of URL: it should be placed after hyperref.
\urlstyle{same}
\addto\captionsenglish{\renewcommand{\contentsname}{Contents:}}

\addto\captionsenglish{\renewcommand{\figurename}{Figure}}
\addto\captionsenglish{\renewcommand{\tablename}{Table}}
\addto\captionsenglish{\renewcommand{\literalblockname}{Code}}

\addto\extrasenglish{\def\pageautorefname{page}}

\setcounter{tocdepth}{3}
\setcounter{secnumdepth}{3}


\title{MITgcm Documentation}
\date{Jun 08, 2017}
\release{1.0}
\author{Alistair Adcroft, Jean-Michel Campin, Stephanie Dutkiewicz, \and Constantinos Evangelinos, David Ferreira, Mick Follows, \and Gael Forget, Baylor Fox-Kemper, Patrick Heimbach, Chris Hill, \and Ed Hill, Helen Hill, Oliver Jahn, Martin Losch, John Marshall, \and Guillaume Maze, Dimitris Menemenlis and Andrea Molod}
\newcommand{\sphinxlogo}{}
\renewcommand{\releasename}{Release}
\makeindex

\begin{document}

\maketitle
\sphinxtableofcontents
\phantomsection\label{\detokenize{index::doc}}



\chapter{Overview}
\label{\detokenize{overview/overview:overview}}\label{\detokenize{overview/overview::doc}}\label{\detokenize{overview/overview:welcome-to-mitgcm-s-user-manual}}
This document provides the reader with the information necessary to
carry out numerical experiments using MITgcm. It gives a comprehensive
description of the continuous equations on which the model is based, the
numerical algorithms the model employs and a description of the associated
program code. Along with the hydrodynamical kernel, physical and
biogeochemical parameterizations of key atmospheric and oceanic processes
are available. A number of examples illustrating the use of the model in
both process and general circulation studies of the atmosphere and ocean are
also presented.


\section{Introduction}
\label{\detokenize{overview/overview:introduction}}
MITgcm has a number of novel aspects:
\begin{quote}
\begin{itemize}
\item {} 
it can be used to study both atmospheric and oceanic phenomena; one hydrodynamical kernel is used to drive forward both atmospheric and oceanic models - see \hyperref[\detokenize{overview/overview:onemodel}]{Figure \ref{\detokenize{overview/overview:onemodel}}}

\end{itemize}
\begin{quote}
\begin{figure}[htbp]
\centering
\capstart

\noindent\sphinxincludegraphics[width=0.700\linewidth]{{onemodel}.pdf}
\caption{MITgcm has a single dynamical kernel that can drive forward either oceanic or atmospheric simulations.}\label{\detokenize{overview/overview:onemodel}}\end{figure}
\end{quote}
\begin{itemize}
\item {} 
it has a non-hydrostatic capability and so can be used to study both small-scale and large scale processes - see \hyperref[\detokenize{overview/overview:all-scales}]{Figure \ref{\detokenize{overview/overview:all-scales}}}

\end{itemize}
\begin{quote}
\begin{figure}[htbp]
\centering
\capstart

\noindent\sphinxincludegraphics[width=0.700\linewidth]{{scales}.pdf}
\caption{MITgcm has non-hydrostatic capabilities, allowing the model to address a wide range of phenomenon - from convection on the left, all the way through to global circulation patterns on the right.}\label{\detokenize{overview/overview:all-scales}}\end{figure}
\end{quote}
\begin{itemize}
\item {} 
finite volume techniques are employed yielding an intuitive discretization and support for the treatment of irregular geometries using orthogonal curvilinear grids and shaved cells - see \hyperref[\detokenize{overview/overview:fvol}]{Figure \ref{\detokenize{overview/overview:fvol}}}

\end{itemize}
\begin{quote}
\begin{figure}[htbp]
\centering
\capstart

\noindent\sphinxincludegraphics[width=0.700\linewidth]{{fvol}.pdf}
\caption{Finite volume techniques (bottom panel) are user, permitting a treatment of topography that rivals \(\sigma\) (terrain following) coordinates.}\label{\detokenize{overview/overview:fvol}}\end{figure}
\end{quote}
\begin{itemize}
\item {} 
tangent linear and adjoint counterparts are automatically maintained along with the forward model, permitting sensitivity and optimization studies.

\item {} 
the model is developed to perform efficiently on a wide variety of computational platforms.

\end{itemize}
\end{quote}

Key publications reporting on and charting the development of the model are \phantomsection\label{\detokenize{overview/overview:id1}}{\hyperref[\detokenize{zreferences:hill-95}]{\sphinxcrossref{{[}HM95{]}}}}\phantomsection\label{\detokenize{overview/overview:id2}}{\hyperref[\detokenize{zreferences:marshall-97a}]{\sphinxcrossref{{[}MHPA97{]}}}}\phantomsection\label{\detokenize{overview/overview:id3}}{\hyperref[\detokenize{zreferences:marshall-97b}]{\sphinxcrossref{{[}MAH+97{]}}}}\phantomsection\label{\detokenize{overview/overview:id4}}{\hyperref[\detokenize{zreferences:adcroft-97}]{\sphinxcrossref{{[}AHM97{]}}}}\phantomsection\label{\detokenize{overview/overview:id5}}{\hyperref[\detokenize{zreferences:mars-eta-98}]{\sphinxcrossref{{[}MJH98{]}}}}\phantomsection\label{\detokenize{overview/overview:id6}}{\hyperref[\detokenize{zreferences:adcroft-99}]{\sphinxcrossref{{[}AM99{]}}}}\phantomsection\label{\detokenize{overview/overview:id7}}{\hyperref[\detokenize{zreferences:hill-99}]{\sphinxcrossref{{[}CHM99{]}}}}\phantomsection\label{\detokenize{overview/overview:id8}}{\hyperref[\detokenize{zreferences:maro-eta-99}]{\sphinxcrossref{{[}MGZ+99{]}}}}\phantomsection\label{\detokenize{overview/overview:id9}}{\hyperref[\detokenize{zreferences:adcroft-04a}]{\sphinxcrossref{{[}AC04{]}}}}\phantomsection\label{\detokenize{overview/overview:id10}}{\hyperref[\detokenize{zreferences:adcroft-04b}]{\sphinxcrossref{{[}ACHM04{]}}}}\phantomsection\label{\detokenize{overview/overview:id11}}{\hyperref[\detokenize{zreferences:marshall-04}]{\sphinxcrossref{{[}MAC+04{]}}}} (an overview on the model formulation can also be found in \phantomsection\label{\detokenize{overview/overview:id12}}{\hyperref[\detokenize{zreferences:adcroft-04c}]{\sphinxcrossref{{[}AHC+04{]}}}}):

Hill, C. and J. Marshall, (1995)
Application of a Parallel Navier-Stokes Model to Ocean Circulation in
Parallel Computational Fluid Dynamics
In Proceedings of Parallel Computational Fluid Dynamics: Implementations
and Results Using Parallel Computers, 545-552.
Elsevier Science B.V.: New York

Marshall, J., C. Hill, L. Perelman, and A. Adcroft, (1997)
Hydrostatic, quasi-hydrostatic, and nonhydrostatic ocean modeling
J. Geophysical Res., 102(C3), 5733-5752.

Marshall, J., A. Adcroft, C. Hill, L. Perelman, and C. Heisey, (1997)
A finite-volume, incompressible Navier Stokes model for studies of the ocean
on parallel computers,
J. Geophysical Res., 102(C3), 5753-5766.

Adcroft, A.J., Hill, C.N. and J. Marshall, (1997)
Representation of topography by shaved cells in a height coordinate ocean
model
Mon Wea Rev, vol 125, 2293-2315

Marshall, J., Jones, H. and C. Hill, (1998)
Efficient ocean modeling using non-hydrostatic algorithms
Journal of Marine Systems, 18, 115-134

Adcroft, A., Hill C. and J. Marshall: (1999)
A new treatment of the Coriolis terms in C-grid models at both high and low
resolutions,
Mon. Wea. Rev. Vol 127, pages 1928-1936

Hill, C, Adcroft,A., Jamous,D., and J. Marshall, (1999)
A Strategy for Terascale Climate Modeling.
In Proceedings of the Eighth ECMWF Workshop on the Use of Parallel Processors
in Meteorology, pages 406-425
World Scientific Publishing Co: UK

Marotzke, J, Giering,R., Zhang, K.Q., Stammer,D., Hill,C., and T.Lee, (1999)
Construction of the adjoint MIT ocean general circulation model and
application to Atlantic heat transport variability
J. Geophysical Res., 104(C12), 29,529-29,547.

We begin by briefly showing some of the results of the model in action to
give a feel for the wide range of problems that can be addressed using it.


\section{Illustrations of the model in action}
\label{\detokenize{overview/overview:illustrations-of-the-model-in-action}}
MITgcm has been designed and used to model a wide range of phenomena,
from convection on the scale of meters in the ocean to the global pattern of
atmospheric winds - see figure ref\{fig:all-scales\}. To give a flavor of the
kinds of problems the model has been used to study, we briefly describe some
of them here. A more detailed description of the underlying formulation,
numerical algorithm and implementation that lie behind these calculations is
given later. Indeed many of the illustrative examples shown below can be
easily reproduced: simply download the model (the minimum you need is a PC
running Linux, together with a FORTRAN77 compiler) and follow the examples
described in detail in the documentation.

subsection\{Global atmosphere: {\color{red}\bfseries{}{}`}Held-Suarez' benchmark\}
begin\{rawhtml\}
\textless{}!-- CMIREDIR:atmospheric\_example: --\textgreater{}
end\{rawhtml\}

A novel feature of MITgcm is its ability to simulate, using one basic algorithm,
both atmospheric and oceanographic flows at both small and large scales.

Figure ref\{fig:eddy\_cs\} shows an instantaneous plot of the 500\$mb\$
temperature field obtained using the atmospheric isomorph of MITgcm run at
\$2.8\textasciicircum{}\{circ \}\$ resolution on the cubed sphere. We see cold air over the pole
(blue) and warm air along an equatorial band (red). Fully developed
baroclinic eddies spawned in the northern hemisphere storm track are
evident. There are no mountains or land-sea contrast in this calculation,
but you can easily put them in. The model is driven by relaxation to a
radiative-convective equilibrium profile, following the description set out
in Held and Suarez; 1994 designed to test atmospheric hydrodynamical cores -
there are no mountains or land-sea contrast.

\%\% CNHbegin
input\{s\_overview/text/cubic\_eddies\_figure\}
\%\% CNHend

As described in Adcroft (2001), a {\color{red}\bfseries{}{}`}cubed sphere' is used to discretize the
globe permitting a uniform griding and obviated the need to Fourier filter.
The {\color{red}\bfseries{}{}`}vector-invariant' form of MITgcm supports any orthogonal curvilinear
grid, of which the cubed sphere is just one of many choices.

Figure ref\{fig:hs\_zave\_u\} shows the 5-year mean, zonally averaged zonal
wind from a 20-level configuration of
the model. It compares favorable with more conventional spatial
discretization approaches. The two plots show the field calculated using the
cube-sphere grid and the flow calculated using a regular, spherical polar
latitude-longitude grid. Both grids are supported within the model.

\%\% CNHbegin
input\{s\_overview/text/hs\_zave\_u\_figure\}
\%\% CNHend

subsection\{Ocean gyres\}
begin\{rawhtml\}
\textless{}!-- CMIREDIR:oceanic\_example: --\textgreater{}
end\{rawhtml\}
begin\{rawhtml\}
\textless{}!-- CMIREDIR:ocean\_gyres: --\textgreater{}
end\{rawhtml\}

Baroclinic instability is a ubiquitous process in the ocean, as well as the
atmosphere. Ocean eddies play an important role in modifying the
hydrographic structure and current systems of the oceans. Coarse resolution
models of the oceans cannot resolve the eddy field and yield rather broad,
diffusive patterns of ocean currents. But if the resolution of our models is
increased until the baroclinic instability process is resolved, numerical
solutions of a different and much more realistic kind, can be obtained.

Figure ref\{fig:ocean-gyres\} shows the surface temperature and
velocity field obtained from MITgcm run at \$frac\{1\}\{6\}\textasciicircum{}\{circ \}\$
horizontal resolution on a textit\{lat-lon\} grid in which the pole has
been rotated by \$90\textasciicircum{}\{circ \}\$ on to the equator (to avoid the
converging of meridian in northern latitudes). 21 vertical levels are
used in the vertical with a {\color{red}\bfseries{}{}`}lopped cell' representation of
topography. The development and propagation of anomalously warm and
cold eddies can be clearly seen in the Gulf Stream region. The
transport of warm water northward by the mean flow of the Gulf Stream
is also clearly visible.

\%\% CNHbegin
input\{s\_overview/text/atl6\_figure\}
\%\% CNHend

subsection\{Global ocean circulation\}
begin\{rawhtml\}
\textless{}!-- CMIREDIR:global\_ocean\_circulation: --\textgreater{}
end\{rawhtml\}

Figure ref\{fig:large-scale-circ\} (top) shows the pattern of ocean
currents at the surface of a \$4\textasciicircum{}\{circ \}\$ global ocean model run with
15 vertical levels. Lopped cells are used to represent topography on a
regular textit\{lat-lon\} grid extending from \$70\textasciicircum{}\{circ \}N\$ to
\$70\textasciicircum{}\{circ \}S\$. The model is driven using monthly-mean winds with
mixed boundary conditions on temperature and salinity at the surface.
The transfer properties of ocean eddies, convection and mixing is
parameterized in this model.

Figure ref\{fig:large-scale-circ\} (bottom) shows the meridional overturning
circulation of the global ocean in Sverdrups.

\%\%CNHbegin
input\{s\_overview/text/global\_circ\_figure\}
\%\%CNHend


\subsection{Convection and mixing over topography}
\label{\detokenize{overview/overview:convection-and-mixing-over-topography}}
Dense plumes generated by localized cooling on the continental shelf of the
ocean may be influenced by rotation when the deformation radius is smaller
than the width of the cooling region. Rather than gravity plumes, the
mechanism for moving dense fluid down the shelf is then through geostrophic
eddies. The simulation shown in \hyperref[\detokenize{overview/overview:fig-convect-and-topo}]{Figure \ref{\detokenize{overview/overview:fig-convect-and-topo}}}
(blue is cold dense fluid, red is
warmer, lighter fluid) employs the non-hydrostatic capability of MITgcm to
trigger convection by surface cooling. The cold, dense water falls down the
slope but is deflected along the slope by rotation. It is found that
entrainment in the vertical plane is reduced when rotational control is
strong, and replaced by lateral entrainment due to the baroclinic
instability of the along-slope current.
\begin{quote}
\begin{figure}[htbp]
\centering
\capstart

\noindent\sphinxincludegraphics[width=0.700\linewidth]{{plume}.pdf}
\caption{MITgcm run in a non-hydrostatic configuration to study convection over a slope.}\label{\detokenize{overview/overview:fig-convect-and-topo}}\end{figure}
\end{quote}


\subsection{Boundary forced internal waves}
\label{\detokenize{overview/overview:boundary-forced-internal-waves}}
The unique ability of MITgcm to treat non-hydrostatic dynamics in the
presence of complex geometry makes it an ideal tool to study internal wave
dynamics and mixing in oceanic canyons and ridges driven by large amplitude
barotropic tidal currents imposed through open boundary conditions.

Fig. ref\{fig:boundary-forced-wave\} shows the influence of cross-slope
topographic variations on
internal wave breaking - the cross-slope velocity is in color, the density
contoured. The internal waves are excited by application of open boundary
conditions on the left. They propagate to the sloping boundary (represented
using MITgcm's finite volume spatial discretization) where they break under
nonhydrostatic dynamics.

\%\%CNHbegin
input\{s\_overview/text/boundary\_forced\_waves\}
\%\%CNHend

subsection\{Parameter sensitivity using the adjoint of MITgcm\}
begin\{rawhtml\}
\textless{}!-- CMIREDIR:parameter\_sensitivity: --\textgreater{}
end\{rawhtml\}

Forward and tangent linear counterparts of MITgcm are supported using an
{\color{red}\bfseries{}{}`}automatic adjoint compiler'. These can be used in parameter sensitivity and
data assimilation studies.

As one example of application of the MITgcm adjoint, Figure
ref\{fig:hf-sensitivity\} maps the gradient \$frac\{partial J\}\{partial
\begin{quote}

mathcal\{H\}\}\$where \$J\$ is the magnitude of the overturning
\end{quote}

stream-function shown in figure ref\{fig:large-scale-circ\} at
\$60\textasciicircum{}\{circ \}N\$ and \$ mathcal\{H\}(lambda,varphi)\$ is the mean, local
air-sea heat flux over a 100 year period. We see that \$J\$ is sensitive
to heat fluxes over the Labrador Sea, one of the important sources of
deep water for the thermohaline circulations. This calculation also
yields sensitivities to all other model parameters.

\%\%CNHbegin
input\{s\_overview/text/adj\_hf\_ocean\_figure\}
\%\%CNHend

subsection\{Global state estimation of the ocean\}
begin\{rawhtml\}
\textless{}!-- CMIREDIR:global\_state\_estimation: --\textgreater{}
end\{rawhtml\}

An important application of MITgcm is in state estimation of the global
ocean circulation. An appropriately defined {\color{red}\bfseries{}{}`}cost function', which measures
the departure of the model from observations (both remotely sensed and
in-situ) over an interval of time, is minimized by adjusting {\color{red}\bfseries{}{}`}control
parameters' such as air-sea fluxes, the wind field, the initial conditions
etc. Figure ref\{fig:assimilated-globes\} shows the large scale planetary
circulation and a Hopf-Muller plot of Equatorial sea-surface height.
Both are obtained from assimilation bringing the model in to
consistency with altimetric and in-situ observations over the period
1992-1997.

\%\% CNHbegin
input\{s\_overview/text/assim\_figure\}
\%\% CNHend

subsection\{Ocean biogeochemical cycles\}
begin\{rawhtml\}
\textless{}!-- CMIREDIR:ocean\_biogeo\_cycles: --\textgreater{}
end\{rawhtml\}

MITgcm is being used to study global biogeochemical cycles in the
ocean. For example one can study the effects of interannual changes in
meteorological forcing and upper ocean circulation on the fluxes of
carbon dioxide and oxygen between the ocean and atmosphere. Figure
ref\{fig:biogeo\} shows the annual air-sea flux of oxygen and its
relation to density outcrops in the southern oceans from a single year
of a global, interannually varying simulation. The simulation is run
at \$1\textasciicircum{}\{circ\}times1\textasciicircum{}\{circ\}\$ resolution telescoping to
\$frac\{1\}\{3\}\textasciicircum{}\{circ\}timesfrac\{1\}\{3\}\textasciicircum{}\{circ\}\$ in the tropics (not
shown).

\%\%CNHbegin
input\{s\_overview/text/biogeo\_figure\}
\%\%CNHend

subsection\{Simulations of laboratory experiments\}
begin\{rawhtml\}
\textless{}!-- CMIREDIR:classroom\_exp: --\textgreater{}
end\{rawhtml\}

Figure ref\{fig:lab-simulation\} shows MITgcm being used to simulate a
laboratory experiment inquiring into the dynamics of the Antarctic Circumpolar Current (ACC). An
initially homogeneous tank of water (\$1m\$ in diameter) is driven from its
free surface by a rotating heated disk. The combined action of mechanical
and thermal forcing creates a lens of fluid which becomes baroclinically
unstable. The stratification and depth of penetration of the lens is
arrested by its instability in a process analogous to that which sets the
stratification of the ACC.

\%\%CNHbegin
input\{s\_overview/text/lab\_figure\}
\%\%CNHend

section\{Continuous equations in {\color{red}\bfseries{}{}`}r' coordinates\}
begin\{rawhtml\}
\textless{}!-- CMIREDIR:z-p\_isomorphism: --\textgreater{}
end\{rawhtml\}

To render atmosphere and ocean models from one dynamical core we exploit
{\color{red}\bfseries{}{}`}isomorphisms' between equation sets that govern the evolution of the
respective fluids - see figure ref\{fig:isomorphic-equations\}.
One system of hydrodynamical equations is written down
and encoded. The model variables have different interpretations depending on
whether the atmosphere or ocean is being studied. Thus, for example, the
vertical coordinate {\color{red}\bfseries{}{}`}\$r\$' is interpreted as pressure, \$p\$, if we are
modeling the atmosphere (right hand side of figure ref\{fig:isomorphic-equations\})
and height, \$z\$, if we are modeling the ocean (left hand side of figure
ref\{fig:isomorphic-equations\}).

\%\%CNHbegin
input\{s\_overview/text/zandpcoord\_figure.tex\}
\%\%CNHend

The state of the fluid at any time is characterized by the distribution of
velocity \$vec\{mathbf\{v\}\}\$, active tracers \$theta \$ and \$S\$, a
{\color{red}\bfseries{}{}`}geopotential' \$phi \$ and density \$rho =rho (theta ,S,p)\$ which may
depend on \$theta \$, \$S\$, and \$p\$. The equations that govern the evolution
of these fields, obtained by applying the laws of classical mechanics and
thermodynamics to a Boussinesq, Navier-Stokes fluid are, written in terms of
a generic vertical coordinate, \$r\$, so that the appropriate
kinematic boundary conditions can be applied isomorphically
see figure ref\{fig:zandp-vert-coord\}.

\%\%CNHbegin
input\{s\_overview/text/vertcoord\_figure.tex\}
\%\%CNHend

begin\{equation\}
frac\{Dvec\{mathbf\{v\}\_\{h\}\}\}\{Dt\}+left( 2vec\{Omega\}times vec\{mathbf\{v\}\}
right) \_\{h\}+mathbf\{nabla \}\_\{h\}phi =mathcal\{F\}\_\{vec\{mathbf\{v\}\_\{h\}\}\}
text\{ horizontal mtm\} label\{eq:horizontal\_mtm\}
end\{equation\}

begin\{equation\}
frac\{Ddot\{r\}\}\{Dt\}+widehat\{k\}cdot left( 2vec\{Omega\}times vec\{mathbf\{
v\}\}right) +frac\{partial phi \}\{partial r\}+b=mathcal\{F\}\_\{dot\{r\}\}text\{
vertical mtm\} label\{eq:vertical\_mtm\}
end\{equation\}

begin\{equation\}
mathbf\{nabla \}\_\{h\}cdot vec\{mathbf\{v\}\}\_\{h\}+frac\{partial dot\{r\}\}\{
partial r\}=0text\{ continuity\}  label\{eq:continuity\}
end\{equation\}

begin\{equation\}
b=b(theta ,S,r)text\{ equation of state\} label\{eq:equation\_of\_state\}
end\{equation\}

begin\{equation\}
frac\{Dtheta \}\{Dt\}=mathcal\{Q\}\_\{theta \}text\{ potential temperature\}
label\{eq:potential\_temperature\}
end\{equation\}

begin\{equation\}
frac\{DS\}\{Dt\}=mathcal\{Q\}\_\{S\}text\{ humidity/salinity\}
label\{eq:humidity\_salt\}
end\{equation\}

Here:

begin\{equation*\}
rtext\{ is the vertical coordinate\}
end\{equation*\}

begin\{equation*\}
frac\{D\}\{Dt\}=frac\{partial \}\{partial t\}+vec\{mathbf\{v\}\}cdot nabla text\{
is the total derivative\}
end\{equation*\}

begin\{equation*\}
mathbf\{nabla \}=mathbf\{nabla \}\_\{h\}+widehat\{k\}frac\{partial \}\{partial r\}
text\{ is the {\color{red}\bfseries{}{}`}grad' operator\}
end\{equation*\}
with \$mathbf\{nabla \}\_\{h\}\$ operating in the horizontal and \$widehat\{k\}
frac\{partial \}\{partial r\}\$ operating in the vertical, where \$widehat\{k\}\$
is a unit vector in the vertical

begin\{equation*\}
ttext\{ is time\}
end\{equation*\}

begin\{equation*\}
vec\{mathbf\{v\}\}=(u,v,dot\{r\})=(vec\{mathbf\{v\}\}\_\{h\},dot\{r\})text\{ is the
velocity\}
end\{equation*\}

begin\{equation*\}
phi text\{ is the {\color{red}\bfseries{}{}`}pressure'/{\color{red}\bfseries{}{}`}geopotential'\}
end\{equation*\}

begin\{equation*\}
vec\{Omega\}text\{ is the Earth's rotation\}
end\{equation*\}

begin\{equation*\}
btext\{ is the {\color{red}\bfseries{}{}`}buoyancy'\}
end\{equation*\}

begin\{equation*\}
theta text\{ is potential temperature\}
end\{equation*\}

begin\{equation*\}
Stext\{ is specific humidity in the atmosphere; salinity in the ocean\}
end\{equation*\}

begin\{equation*\}
mathcal\{F\}\_\{vec\{mathbf\{v\}\}\}text\{ are forcing and dissipation of \}vec\{
mathbf\{v\}\}
end\{equation*\}

begin\{equation*\}
mathcal\{Q\}\_\{theta \}mathcal\{\}text\{are forcing and dissipation of \}theta
end\{equation*\}

begin\{equation*\}
mathcal\{Q\}\_\{S\}mathcal\{\}text\{are forcing and dissipation of \}S
end\{equation*\}

The \$mathcal\{F\}\textasciicircum{}\{prime \}s\$ and \$mathcal\{Q\}\textasciicircum{}\{prime \}s\$ are provided by
{\color{red}\bfseries{}{}`}physics' and forcing packages for atmosphere and ocean. These are described
in later chapters.

subsection\{Kinematic Boundary conditions\}

subsubsection\{vertical\}

at fixed and moving \$r\$ surfaces we set (see figure ref\{fig:zandp-vert-coord\}):

begin\{equation\}
dot\{r\}=0 text\{at\} r=R\_\{fixed\}(x,y)text\{ (ocean bottom, top of the atmosphere)\}
label\{eq:fixedbc\}
end\{equation\}

begin\{equation\}
dot\{r\}=frac\{Dr\}\{Dt\} text\{at\} r=R\_\{moving\}text\{ (ocean surface,bottom of the atmosphere)\}  label\{eq:movingbc\}
end\{equation\}

Here

begin\{equation*\}
R\_\{moving\}=R\_\{o\}+eta
end\{equation*\}
where \$R\_\{o\}(x,y)\$ is the {\color{red}\bfseries{}{}`}\$r-\$value' (height or pressure, depending on
whether we are in the atmosphere or ocean) of the {\color{red}\bfseries{}{}`}moving surface' in the
resting fluid and \$eta \$ is the departure from \$R\_\{o\}(x,y)\$ in the presence
of motion.

subsubsection\{horizontal\}

begin\{equation\}
vec\{mathbf\{v\}\}cdot vec\{mathbf\{n\}\}=0  label\{eq:noflow\}
end\{equation\}
where \$vec\{mathbf\{n\}\}\$ is the normal to a solid boundary.

subsection\{Atmosphere\}

In the atmosphere, (see figure ref\{fig:zandp-vert-coord\}), we interpret:

begin\{equation\}
r=ptext\{ is the pressure\}  label\{eq:atmos-r\}
end\{equation\}

begin\{equation\}
dot\{r\}=frac\{Dp\}\{Dt\}=omega text\{ is the vertical velocity in \}ptext\{
coordinates\}  label\{eq:atmos-omega\}
end\{equation\}

begin\{equation\}
phi =g,ztext\{ is the geopotential height\}  label\{eq:atmos-phi\}
end\{equation\}

begin\{equation\}
b=frac\{partial Pi \}\{partial p\}theta text\{ is the buoyancy\}
label\{eq:atmos-b\}
end\{equation\}

begin\{equation\}
theta =T(frac\{p\_\{c\}\}\{p\})\textasciicircum{}\{kappa \}text\{ is potential temperature\}
label\{eq:atmos-theta\}
end\{equation\}

begin\{equation\}
S=q,text\{ is the specific humidity\}  label\{eq:atmos-s\}
end\{equation\}
where

begin\{equation*\}
Ttext\{ is absolute temperature\}
end\{equation*\}
begin\{equation*\}
ptext\{ is the pressure\}
end\{equation*\}
begin\{eqnarray*\}
\&\&ztext\{ is the height of the pressure surface\} \textbackslash{}
\&\&gtext\{ is the acceleration due to gravity\}
end\{eqnarray*\}

In the above the ideal gas law, \$p=rho RT\$, has been expressed in terms of
the Exner function \$Pi (p)\$ given by (see Appendix Atmosphere)
begin\{equation\}
Pi (p)=c\_\{p\}(frac\{p\}\{p\_\{c\}\})\textasciicircum{}\{kappa \}  label\{eq:exner\}
end\{equation\}
where \$p\_\{c\}\$ is a reference pressure and \$kappa =R/c\_\{p\}\$ with \$R\$ the gas
constant and \$c\_\{p\}\$ the specific heat of air at constant pressure.

At the top of the atmosphere (which is {\color{red}\bfseries{}{}`}fixed' in our \$r\$ coordinate):

begin\{equation*\}
R\_\{fixed\}=p\_\{top\}=0
end\{equation*\}
In a resting atmosphere the elevation of the mountains at the bottom is
given by
begin\{equation*\}
R\_\{moving\}=R\_\{o\}(x,y)=p\_\{o\}(x,y)
end\{equation*\}
i.e. the (hydrostatic) pressure at the top of the mountains in a resting
atmosphere.

The boundary conditions at top and bottom are given by:

begin\{eqnarray\}
\&\&omega =0\textasciitilde{}text\{at \}r=R\_\{fixed\} text\{ (top of the atmosphere)\}
label\{eq:fixed-bc-atmos\} \textbackslash{}
omega \&=\&frac\{Dp\_\{s\}\}\{Dt\}text\{; at \}r=R\_\{moving\}text\{ (bottom of the
atmosphere)\}  label\{eq:moving-bc-atmos\}
end\{eqnarray\}

Then the (hydrostatic form of) equations
(ref\{eq:horizontal\_mtm\}-ref\{eq:humidity\_salt\}) yields a consistent
set of atmospheric equations which, for convenience, are written out
in \$p\$ coordinates in Appendix Atmosphere - see
eqs(ref\{eq:atmos-prime\}).

subsection\{Ocean\}

In the ocean we interpret:
begin\{eqnarray\}
r \&=\&ztext\{ is the height\}  label\{eq:ocean-z\} \textbackslash{}
dot\{r\} \&=\&frac\{Dz\}\{Dt\}=wtext\{ is the vertical velocity\}
label\{eq:ocean-w\} \textbackslash{}
phi \&=\&frac\{p\}\{rho \_\{c\}\}text\{ is the pressure\}  label\{eq:ocean-p\} \textbackslash{}
b(theta ,S,r) \&=\&frac\{g\}\{rho \_\{c\}\}left( rho (theta ,S,r)-rho
\_\{c\}right) text\{ is the buoyancy\}  label\{eq:ocean-b\}
end\{eqnarray\}
where \$rho \_\{c\}\$ is a fixed reference density of water and \$g\$ is the
acceleration due to gravity.noindent

In the above

At the bottom of the ocean: \$R\_\{fixed\}(x,y)=-H(x,y)\$.

The surface of the ocean is given by: \$R\_\{moving\}=eta \$

The position of the resting free surface of the ocean is given by \$
R\_\{o\}=Z\_\{o\}=0\$.

Boundary conditions are:

begin\{eqnarray\}
w \&=\&0\textasciitilde{}text\{at \}r=R\_\{fixed\}text\{ (ocean bottom)\}  label\{eq:fixed-bc-ocean\}
\textbackslash{}
w \&=\&frac\{Deta \}\{Dt\}text\{ at \}r=R\_\{moving\}=eta text\{ (ocean surface)
label\{eq:moving-bc-ocean\}\}
end\{eqnarray\}
where \$eta \$ is the elevation of the free surface.

Then equations (ref\{eq:horizontal\_mtm\}-ref\{eq:humidity\_salt\}) yield a consistent set
of oceanic equations
which, for convenience, are written out in \$z\$ coordinates in Appendix Ocean
- see eqs(ref\{eq:ocean-mom\}) to (ref\{eq:ocean-salt\}).

subsection\{Hydrostatic, Quasi-hydrostatic, Quasi-nonhydrostatic and
Non-hydrostatic forms\}
label\{sec:all\_hydrostatic\_forms\}
begin\{rawhtml\}
\textless{}!-- CMIREDIR:non\_hydrostatic: --\textgreater{}
end\{rawhtml\}

Let us separate \$phi \$ in to surface, hydrostatic and non-hydrostatic terms:

begin\{equation\}
phi (x,y,r)=phi \_\{s\}(x,y)+phi \_\{hyd\}(x,y,r)+phi \_\{nh\}(x,y,r)
label\{eq:phi-split\}
end\{equation\}
\%and write eq(ref\{eq:incompressible\}) in the form:
\%                  \textasciicircum{}- this eq is missing (jmc) ; replaced with:
and write eq( ref\{eq:horizontal\_mtm\}) in the form:

begin\{equation\}
frac\{partial vec\{mathbf\{v\}\_\{h\}\}\}\{partial t\}+mathbf\{nabla \}\_\{h\}phi
\_\{s\}+mathbf\{nabla \}\_\{h\}phi \_\{hyd\}+epsilon \_\{nh\}mathbf\{nabla \}\_\{h\}phi
\_\{nh\}=vec\{mathbf\{G\}\}\_\{vec\{v\}\_\{h\}\}  label\{eq:mom-h\}
end\{equation\}

begin\{equation\}
frac\{partial phi \_\{hyd\}\}\{partial r\}=-b  label\{eq:hydrostatic\}
end\{equation\}

begin\{equation\}
epsilon \_\{nh\}frac\{partial dot\{r\}\}\{partial t\}+frac\{partial phi \_\{nh\}\}\{
partial r\}=G\_\{dot\{r\}\}  label\{eq:mom-w\}
end\{equation\}
Here \$epsilon \_\{nh\}\$ is a non-hydrostatic parameter.

The \$left( vec\{mathbf\{G\}\}\_\{vec\{v\}\},G\_\{dot\{r\}\}right) \$ in eq(ref
\{eq:mom-h\}) and (ref\{eq:mom-w\}) represent advective, metric and Coriolis
terms in the momentum equations. In spherical coordinates they take the form
footnote\{
In the hydrostatic primitive equations (textbf\{HPE\}) all underlined terms
in (ref\{eq:gu-speherical\}), (ref\{eq:gv-spherical\}) and (ref
\{eq:gw-spherical\}) are omitted; the singly-underlined terms are included in
the quasi-hydrostatic model (textbf\{QH\}). The fully non-hydrostatic model (
textbf\{NH\}) includes all terms.\} - see Marshall et al 1997a for a full
discussion:

begin\{equation\}
left.
begin\{tabular\}\{l\}
\$G\_\{u\}=-vec\{mathbf\{v\}\}.nabla u\$ \textbackslash{}
\$-left\{ underline\{frac\{udot\{r\}\}\{\{r\}\}\}-frac\{uvtan varphi\}\{\{r\}\}right\} \$
\textbackslash{}
\$-left\{ -2Omega vsin varphi+underline\{2Omega dot\{r\}cos varphi\}right\} \$
\textbackslash{}
\$+mathcal\{F\}\_\{u\}\$
end\{tabular\}
right\} left\{
begin\{tabular\}\{l\}
textit\{advection\} \textbackslash{}
textit\{metric\} \textbackslash{}
textit\{Coriolis\} \textbackslash{}
textit\{Forcing/Dissipation\}
end\{tabular\}
right. qquad  label\{eq:gu-speherical\}
end\{equation\}

begin\{equation\}
left.
begin\{tabular\}\{l\}
\$G\_\{v\}=-vec\{mathbf\{v\}\}.nabla v\$ \textbackslash{}
\$-left\{ underline\{frac\{vdot\{r\}\}\{\{r\}\}\}-frac\{u\textasciicircum{}\{2\}tan varphi\}\{\{r\}\}right\}
\$ \textbackslash{}
\$-left\{ -2Omega usin varphi right\} \$ \textbackslash{}
\$+mathcal\{F\}\_\{v\}\$
end\{tabular\}
right\} left\{
begin\{tabular\}\{l\}
textit\{advection\} \textbackslash{}
textit\{metric\} \textbackslash{}
textit\{Coriolis\} \textbackslash{}
textit\{Forcing/Dissipation\}
end\{tabular\}
right. qquad  label\{eq:gv-spherical\}
end\{equation\}
qquad qquad qquad qquad qquad

begin\{equation\}
left.
begin\{tabular\}\{l\}
\$G\_\{dot\{r\}\}=-underline\{underline\{vec\{mathbf\{v\}\}.nabla dot\{r\}\}\}\$ \textbackslash{}
\$+left\{ underline\{frac\{u\textasciicircum{}\{\_\{\textasciicircum{}\{2\}\}\}+v\textasciicircum{}\{2\}\}\{\{r\}\}\}right\} \$ \textbackslash{}
\$\{+\}underline\{\{2Omega ucos varphi\}\}\$ \textbackslash{}
\$underline\{underline\{mathcal\{F\}\_\{dot\{r\}\}\}\}\$
end\{tabular\}
right\} left\{
begin\{tabular\}\{l\}
textit\{advection\} \textbackslash{}
textit\{metric\} \textbackslash{}
textit\{Coriolis\} \textbackslash{}
textit\{Forcing/Dissipation\}
end\{tabular\}
right.  label\{eq:gw-spherical\}
end\{equation\}
qquad qquad qquad qquad qquad

In the above {\color{red}\bfseries{}{}`}\$\{r\}\$' is the distance from the center of the earth and {\color{red}\bfseries{}{}`}\$varphi\$
` is latitude.

Grad and div operators in spherical coordinates are defined in appendix
OPERATORS.

\%\%CNHbegin
input\{s\_overview/text/sphere\_coord\_figure.tex\}
\%\%CNHend

subsubsection\{Shallow atmosphere approximation\}

Most models are based on the {\color{red}\bfseries{}{}`}hydrostatic primitive equations' (HPE's)
in which the vertical momentum equation is reduced to a statement of
hydrostatic balance and the {\color{red}\bfseries{}{}`}traditional approximation' is made in
which the Coriolis force is treated approximately and the shallow
atmosphere approximation is made.  MITgcm need not make the
{\color{red}\bfseries{}{}`}traditional approximation'. To be able to support consistent
non-hydrostatic forms the shallow atmosphere approximation can be
relaxed - when dividing through by \$ r \$ in, for example,
(ref\{eq:gu-speherical\}), we do not replace \$r\$ by \$a\$, the radius of
the earth.

subsubsection\{Hydrostatic and quasi-hydrostatic forms\}
label\{sec:hydrostatic\_and\_quasi-hydrostatic\_forms\}

These are discussed at length in Marshall et al (1997a).

In the {\color{red}\bfseries{}{}`}hydrostatic primitive equations' (textbf\{HPE)\} all the underlined
terms in Eqs. (ref\{eq:gu-speherical\} \$rightarrow \$ref\{eq:gw-spherical\})
are neglected and {\color{red}\bfseries{}{}`}\$\{r\}\$' is replaced by {\color{red}\bfseries{}{}`}\$a\$', the mean radius of the
earth. Once the pressure is found at one level - e.g. by inverting a 2-d
Elliptic equation for \$phi \_\{s\}\$ at \$r=R\_\{moving\}\$ - the pressure can be
computed at all other levels by integration of the hydrostatic relation, eq(
ref\{eq:hydrostatic\}).

In the {\color{red}\bfseries{}{}`}quasi-hydrostatic' equations (textbf\{QH)\} strict balance between
gravity and vertical pressure gradients is not imposed. The \$2Omega ucos
varphi \$ Coriolis term are not neglected and are balanced by a non-hydrostatic
contribution to the pressure field: only the terms underlined twice in Eqs. (
ref\{eq:gu-speherical\}\$rightarrow \$ref\{eq:gw-spherical\}) are set to zero
and, simultaneously, the shallow atmosphere approximation is relaxed. In
textbf\{QH\}textit\{all\} the metric terms are retained and the full
variation of the radial position of a particle monitored. The textbf\{QH\}vertical momentum equation (ref\{eq:mom-w\}) becomes:

begin\{equation*\}
frac\{partial phi \_\{nh\}\}\{partial r\}=2Omega ucos varphi
end\{equation*\}
making a small correction to the hydrostatic pressure.

textbf\{QH\} has good energetic credentials - they are the same as for
textbf\{HPE\}. Importantly, however, it has the same angular momentum
principle as the full non-hydrostatic model (textbf\{NH)\} - see Marshall
et.al., 1997a. As in textbf\{HPE \}only a 2-d elliptic problem need be solved.

subsubsection\{Non-hydrostatic and quasi-nonhydrostatic forms\}

MITgcm presently supports a full non-hydrostatic ocean isomorph, but
only a quasi-non-hydrostatic atmospheric isomorph.

paragraph\{Non-hydrostatic Ocean\}

In the non-hydrostatic ocean model all terms in equations Eqs.(ref
\{eq:gu-speherical\} \$rightarrow \$ref\{eq:gw-spherical\}) are retained. A
three dimensional elliptic equation must be solved subject to Neumann
boundary conditions (see below). It is important to note that use of the
full textbf\{NH\} does not admit any new {\color{red}\bfseries{}{}`}fast' waves in to the system - the
incompressible condition eq(ref\{eq:continuity\}) has already filtered out
acoustic modes. It does, however, ensure that the gravity waves are treated
accurately with an exact dispersion relation. The textbf\{NH\} set has a
complete angular momentum principle and consistent energetics - see White
and Bromley, 1995; Marshall et.al.1997a.

paragraph\{Quasi-nonhydrostatic Atmosphere\}

In the non-hydrostatic version of our atmospheric model we approximate \$dot\{
r\}\$ in the vertical momentum eqs(ref\{eq:mom-w\}) and (ref\{eq:gv-spherical\})
(but only here) by:

begin\{equation\}
dot\{r\}=frac\{Dp\}\{Dt\}=frac\{1\}\{g\}frac\{Dphi \}\{Dt\}  label\{eq:quasi-nh-w\}
end\{equation\}
where \$p\_\{hy\}\$ is the hydrostatic pressure.

subsubsection\{Summary of equation sets supported by model\}

paragraph\{Atmosphere\}

Hydrostatic, and quasi-hydrostatic and quasi non-hydrostatic forms of the
compressible non-Boussinesq equations in \$p-\$coordinates are supported.

subparagraph\{Hydrostatic and quasi-hydrostatic\}

The hydrostatic set is written out in \$p-\$coordinates in appendix Atmosphere
- see eq(ref\{eq:atmos-prime\}).

subparagraph\{Quasi-nonhydrostatic\}

A quasi-nonhydrostatic form is also supported.

paragraph\{Ocean\}

subparagraph\{Hydrostatic and quasi-hydrostatic\}

Hydrostatic, and quasi-hydrostatic forms of the incompressible Boussinesq
equations in \$z-\$coordinates are supported.

subparagraph\{Non-hydrostatic\}

Non-hydrostatic forms of the incompressible Boussinesq equations in \$z-\$
coordinates are supported - see eqs(ref\{eq:ocean-mom\}) to (ref
\{eq:ocean-salt\}).

subsection\{Solution strategy\}

The method of solution employed in the textbf\{HPE\}, textbf\{QH\} and textbf\{
NH\} models is summarized in Figure ref\{fig:solution-strategy\}.
Under all dynamics, a 2-d elliptic equation is
first solved to find the surface pressure and the hydrostatic pressure at
any level computed from the weight of fluid above. Under textbf\{HPE\} and
textbf\{QH\} dynamics, the horizontal momentum equations are then stepped
forward and \$dot\{r\}\$ found from continuity. Under textbf\{NH\} dynamics a
3-d elliptic equation must be solved for the non-hydrostatic pressure before
stepping forward the horizontal momentum equations; \$dot\{r\}\$ is found by
stepping forward the vertical momentum equation.

\%\%CNHbegin
input\{s\_overview/text/solution\_strategy\_figure.tex\}
\%\%CNHend

There is no penalty in implementing textbf\{QH\} over textbf\{HPE\} except, of
course, some complication that goes with the inclusion of \$cos varphi \$
Coriolis terms and the relaxation of the shallow atmosphere approximation.
But this leads to negligible increase in computation. In textbf\{NH\}, in
contrast, one additional elliptic equation - a three-dimensional one - must
be inverted for \$p\_\{nh\}\$. However the {\color{red}\bfseries{}{}`}overhead' of the textbf\{NH\} model is
essentially negligible in the hydrostatic limit (see detailed discussion in
Marshall et al, 1997) resulting in a non-hydrostatic algorithm that, in the
hydrostatic limit, is as computationally economic as the textbf\{HPEs\}.

subsection\{Finding the pressure field\}
label\{sec:finding\_the\_pressure\_field\}

Unlike the prognostic variables \$u\$, \$v\$, \$w\$, \$theta \$ and \$S\$, the
pressure field must be obtained diagnostically. We proceed, as before, by
dividing the total (pressure/geo) potential in to three parts, a surface
part, \$phi \_\{s\}(x,y)\$, a hydrostatic part \$phi \_\{hyd\}(x,y,r)\$ and a
non-hydrostatic part \$phi \_\{nh\}(x,y,r)\$, as in (ref\{eq:phi-split\}), and
writing the momentum equation as in (ref\{eq:mom-h\}).

subsubsection\{Hydrostatic pressure\}

Hydrostatic pressure is obtained by integrating (ref\{eq:hydrostatic\})
vertically from \$r=R\_\{o\}\$ where \$phi \_\{hyd\}(r=R\_\{o\})=0\$, to yield:

begin\{equation*\}
int\_\{r\}\textasciicircum{}\{R\_\{o\}\}frac\{partial phi \_\{hyd\}\}\{partial r\}dr=left{[} phi \_\{hyd\}
right{]} \_\{r\}\textasciicircum{}\{R\_\{o\}\}=int\_\{r\}\textasciicircum{}\{R\_\{o\}\}-bdr
end\{equation*\}
and so

begin\{equation\}
phi \_\{hyd\}(x,y,r)=int\_\{r\}\textasciicircum{}\{R\_\{o\}\}bdr  label\{eq:hydro-phi\}
end\{equation\}

The model can be easily modified to accommodate a loading term (e.g
atmospheric pressure pushing down on the ocean's surface) by setting:

begin\{equation\}
phi \_\{hyd\}(r=R\_\{o\})=loading  label\{eq:loading\}
end\{equation\}

subsubsection\{Surface pressure\}

The surface pressure equation can be obtained by integrating continuity,
(ref\{eq:continuity\}), vertically from \$r=R\_\{fixed\}\$ to \$r=R\_\{moving\}\$

begin\{equation*\}
int\_\{R\_\{fixed\}\}\textasciicircum{}\{R\_\{moving\}\}left( mathbf\{nabla \}\_\{h\}cdot vec\{mathbf\{v\}
\}\_\{h\}+partial \_\{r\}dot\{r\}right) dr=0
end\{equation*\}

Thus:

begin\{equation*\}
frac\{partial eta \}\{partial t\}+vec\{mathbf\{v\}\}.nabla eta
+int\_\{R\_\{fixed\}\}\textasciicircum{}\{R\_\{moving\}\}mathbf\{nabla \}\_\{h\}cdot vec\{mathbf\{v\}\}
\_\{h\}dr=0
end\{equation*\}
where \$eta =R\_\{moving\}-R\_\{o\}\$ is the free-surface \$r\$-anomaly in units of \$
r \$. The above can be rearranged to yield, using Leibnitz's theorem:

begin\{equation\}
frac\{partial eta \}\{partial t\}+mathbf\{nabla \}\_\{h\}cdot
int\_\{R\_\{fixed\}\}\textasciicircum{}\{R\_\{moving\}\}vec\{mathbf\{v\}\}\_\{h\}dr=text\{source\}
label\{eq:free-surface\}
end\{equation\}
where we have incorporated a source term.

Whether \$phi \$ is pressure (ocean model, \$p/rho \_\{c\}\$) or geopotential
(atmospheric model), in (ref\{eq:mom-h\}), the horizontal gradient term can
be written
begin\{equation\}
mathbf\{nabla \}\_\{h\}phi \_\{s\}=mathbf\{nabla \}\_\{h\}left( b\_\{s\}eta right)
label\{eq:phi-surf\}
end\{equation\}
where \$b\_\{s\}\$ is the buoyancy at the surface.

In the hydrostatic limit (\$epsilon \_\{nh\}=0\$), equations (ref\{eq:mom-h\}), (ref
\{eq:free-surface\}) and (ref\{eq:phi-surf\}) can be solved by inverting a 2-d
elliptic equation for \$phi \_\{s\}\$ as described in Chapter 2. Both {\color{red}\bfseries{}{}`}free
surface' and {\color{red}\bfseries{}{}`}rigid lid' approaches are available.

subsubsection\{Non-hydrostatic pressure\}

Taking the horizontal divergence of (ref\{eq:mom-h\}) and adding
\$frac\{partial \}\{partial r\}\$ of (ref\{eq:mom-w\}), invoking the continuity equation
(ref\{eq:continuity\}), we deduce that:

begin\{equation\}
nabla \_\{3\}\textasciicircum{}\{2\}phi \_\{nh\}=nabla .vec\{mathbf\{G\}\}\_\{vec\{v\}\}-left( mathbf\{
nabla \}\_\{h\}\textasciicircum{}\{2\}phi \_\{s\}+mathbf\{nabla \}\textasciicircum{}\{2\}phi \_\{hyd\}right) =nabla .
vec\{mathbf\{F\}\}  label\{eq:3d-invert\}
end\{equation\}

For a given rhs this 3-d elliptic equation must be inverted for \$phi \_\{nh\}\$
subject to appropriate choice of boundary conditions. This method is usually
called textit\{The Pressure Method\} {[}Harlow and Welch, 1965; Williams, 1969;
Potter, 1976{]}. In the hydrostatic primitive equations case (textbf\{HPE\}),
the 3-d problem does not need to be solved.

paragraph\{Boundary Conditions\}

We apply the condition of no normal flow through all solid boundaries - the
coasts (in the ocean) and the bottom:

begin\{equation\}
vec\{mathbf\{v\}\}.widehat\{n\}=0  label\{nonormalflow\}
end\{equation\}
where \$widehat\{n\}\$ is a vector of unit length normal to the boundary. The
kinematic condition (ref\{nonormalflow\}) is also applied to the vertical
velocity at \$r=R\_\{moving\}\$. No-slip \$left( v\_\{T\}=0right) \$or slip \$
left( partial v\_\{T\}/partial n=0right) \$conditions are employed on the
tangential component of velocity, \$v\_\{T\}\$, at all solid boundaries,
depending on the form chosen for the dissipative terms in the momentum
equations - see below.

Eq.(ref\{nonormalflow\}) implies, making use of (ref\{eq:mom-h\}), that:

begin\{equation\}
widehat\{n\}.nabla phi \_\{nh\}=widehat\{n\}.vec\{mathbf\{F\}\}
label\{eq:inhom-neumann-nh\}
end\{equation\}
where

begin\{equation*\}
vec\{mathbf\{F\}\}=vec\{mathbf\{G\}\}\_\{vec\{v\}\}-left( mathbf\{nabla \}\_\{h\}phi
\_\{s\}+mathbf\{nabla \}phi \_\{hyd\}right)
end\{equation*\}
presenting inhomogeneous Neumann boundary conditions to the Elliptic problem
(ref\{eq:3d-invert\}). As shown, for example, by Williams (1969), one can
exploit classical 3D potential theory and, by introducing an appropriately
chosen \$delta \$-function sheet of {\color{red}\bfseries{}{}`}source-charge', replace the
inhomogeneous boundary condition on pressure by a homogeneous one. The
source term \$rhs\$ in (ref\{eq:3d-invert\}) is the divergence of the vector \$
vec\{mathbf\{F\}\}.\$ By simultaneously setting \$
begin\{array\}\{l\}
widehat\{n\}.vec\{mathbf\{F\}\}
end\{array\}
=0\$and \$widehat\{n\}.nabla phi \_\{nh\}=0\$on the boundary the following
self-consistent but simpler homogenized Elliptic problem is obtained:

begin\{equation*\}
nabla \textasciicircum{}\{2\}phi \_\{nh\}=nabla .widetilde\{vec\{mathbf\{F\}\}\}qquad
end\{equation*\}
where \$widetilde\{vec\{mathbf\{F\}\}\}\$ is a modified \$vec\{mathbf\{F\}\}\$ such
that \$widetilde\{vec\{mathbf\{F\}\}\}.widehat\{n\}=0\$. As is implied by (ref
\{eq:inhom-neumann-nh\}) the modified boundary condition becomes:

begin\{equation\}
widehat\{n\}.nabla phi \_\{nh\}=0  label\{eq:hom-neumann-nh\}
end\{equation\}

If the flow is {\color{red}\bfseries{}{}`}close' to hydrostatic balance then the 3-d inversion
converges rapidly because \$phi \_\{nh\}\$is then only a small correction to
the hydrostatic pressure field (see the discussion in Marshall et al, a,b).

The solution \$phi \_\{nh\}\$to (ref\{eq:3d-invert\}) and (ref\{eq:inhom-neumann-nh\})
does not vanish at \$r=R\_\{moving\}\$, and so refines the pressure there.

subsection\{Forcing/dissipation\}

subsubsection\{Forcing\}

The forcing terms \$mathcal\{F\}\$ on the rhs of the equations are provided by
{\color{red}\bfseries{}{}`}physics packages' and forcing packages. These are described later on.

subsubsection\{Dissipation\}

paragraph\{Momentum\}

Many forms of momentum dissipation are available in the model. Laplacian and
biharmonic frictions are commonly used:

begin\{equation\}
D\_\{V\}=A\_\{h\}nabla \_\{h\}\textasciicircum{}\{2\}v+A\_\{v\}frac\{partial \textasciicircum{}\{2\}v\}\{partial z\textasciicircum{}\{2\}\}
+A\_\{4\}nabla \_\{h\}\textasciicircum{}\{4\}v  label\{eq:dissipation\}
end\{equation\}
where \$A\_\{h\}\$ and \$A\_\{v\}\$are (constant) horizontal and vertical viscosity
coefficients and \$A\_\{4\}\$is the horizontal coefficient for biharmonic
friction. These coefficients are the same for all velocity components.

paragraph\{Tracers\}

The mixing terms for the temperature and salinity equations have a similar
form to that of momentum except that the diffusion tensor can be
non-diagonal and have varying coefficients.
begin\{equation\}
D\_\{T,S\}=nabla .{[}underline\{underline\{K\}\}nabla (T,S){]}+K\_\{4\}nabla
\_\{h\}\textasciicircum{}\{4\}(T,S)  label\{eq:diffusion\}
end\{equation\}
where \$underline\{underline\{K\}\}\$is the diffusion tensor and the \$K\_\{4\}\$
horizontal coefficient for biharmonic diffusion. In the simplest case where
the subgrid-scale fluxes of heat and salt are parameterized with constant
horizontal and vertical diffusion coefficients, \$underline\{underline\{K\}\}\$,
reduces to a diagonal matrix with constant coefficients:

begin\{equation\}
qquad qquad qquad qquad K=left(
begin\{array\}\{ccc\}
K\_\{h\} \& 0 \& 0 \textbackslash{}
0 \& K\_\{h\} \& 0 \textbackslash{}
0 \& 0 \& K\_\{v\}
end\{array\}
right) qquad qquad qquad  label\{eq:diagonal-diffusion-tensor\}
end\{equation\}
where \$K\_\{h\}\$and \$K\_\{v\}\$are the horizontal and vertical diffusion
coefficients. These coefficients are the same for all tracers (temperature,
salinity ... ).

subsection\{Vector invariant form\}

For some purposes it is advantageous to write momentum advection in
eq(ref \{eq:horizontal\_mtm\}) and (ref\{eq:vertical\_mtm\}) in the
(so-called) {\color{red}\bfseries{}{}`}vector invariant' form:

begin\{equation\}
frac\{Dvec\{mathbf\{v\}\}\}\{Dt\}=frac\{partial vec\{mathbf\{v\}\}\}\{partial t\}
+left( nabla times vec\{mathbf\{v\}\}right) times vec\{mathbf\{v\}\}+nabla
left{[} frac\{1\}\{2\}(vec\{mathbf\{v\}\}cdot vec\{mathbf\{v\}\})right{]}
label\{eq:vi-identity\}
end\{equation\}
This permits alternative numerical treatments of the non-linear terms based
on their representation as a vorticity flux. Because gradients of coordinate
vectors no longer appear on the rhs of (ref\{eq:vi-identity\}), explicit
representation of the metric terms in (ref\{eq:gu-speherical\}), (ref
\{eq:gv-spherical\}) and (ref\{eq:gw-spherical\}), can be avoided: information
about the geometry is contained in the areas and lengths of the volumes used
to discretize the model.

subsection\{Adjoint\}

Tangent linear and adjoint counterparts of the forward model are described
in Chapter 5.

section\{Appendix ATMOSPHERE\}

subsection\{Hydrostatic Primitive Equations for the Atmosphere in pressure
coordinates\}

label\{sect-hpe-p\}

The hydrostatic primitive equations (HPEs) in p-coordinates are:
begin\{eqnarray\}
frac\{Dvec\{mathbf\{v\}\}\_\{h\}\}\{Dt\}+fhat\{mathbf\{k\}\}times vec\{mathbf\{v\}\}
\_\{h\}+mathbf\{nabla \}\_\{p\}phi \&=\&vec\{mathbf\{mathcal\{F\}\}\}
label\{eq:atmos-mom\} \textbackslash{}
frac\{partial phi \}\{partial p\}+alpha \&=\&0  label\{eq-p-hydro-start\} \textbackslash{}
mathbf\{nabla \}\_\{p\}cdot vec\{mathbf\{v\}\}\_\{h\}+frac\{partial omega \}\{
partial p\} \&=\&0  label\{eq:atmos-cont\} \textbackslash{}
palpha \&=\&RT  label\{eq:atmos-eos\} \textbackslash{}
c\_\{v\}frac\{DT\}\{Dt\}+pfrac\{Dalpha \}\{Dt\} \&=\&mathcal\{Q\}  label\{eq:atmos-heat\}
end\{eqnarray\}
where \$vec\{mathbf\{v\}\}\_\{h\}=(u,v,0)\$ is the {\color{red}\bfseries{}{}`}horizontal' (on pressure
surfaces) component of velocity, \$frac\{D\}\{Dt\}=frac\{partial\}\{partial t\}
+vec\{mathbf\{v\}\}\_\{h\}cdot mathbf\{nabla \}\_\{p\}+omega frac\{partial \}\{partial p\}\$
is the total derivative, \$f=2Omega sin varphi\$ is the Coriolis parameter,
\$phi =gz\$ is the geopotential, \$alpha =1/rho \$ is the specific volume,
\$omega =frac\{Dp \}\{Dt\}\$ is the vertical velocity in the \$p-\$coordinate.
Equation(ref \{eq:atmos-heat\}) is the first law of thermodynamics where internal
energy \$e=c\_\{v\}T\$, \$T\$ is temperature, \$Q\$ is the rate of heating per unit mass
and \$pfrac\{Dalpha \}\{Dt\}\$ is the work done by the fluid in compressing.

It is convenient to cast the heat equation in terms of potential temperature
\$theta \$ so that it looks more like a generic conservation law.
Differentiating (ref\{eq:atmos-eos\}) we get:
begin\{equation*\}
pfrac\{Dalpha \}\{Dt\}+alpha frac\{Dp\}\{Dt\}=Rfrac\{DT\}\{Dt\}
end\{equation*\}
which, when added to the heat equation (ref\{eq:atmos-heat\}) and using \$
c\_\{p\}=c\_\{v\}+R\$, gives:
begin\{equation\}
c\_\{p\}frac\{DT\}\{Dt\}-alpha frac\{Dp\}\{Dt\}=mathcal\{Q\}
label\{eq-p-heat-interim\}
end\{equation\}
Potential temperature is defined:
begin\{equation\}
theta =T(frac\{p\_\{c\}\}\{p\})\textasciicircum{}\{kappa \}  label\{eq:potential-temp\}
end\{equation\}
where \$p\_\{c\}\$ is a reference pressure and \$kappa =R/c\_\{p\}\$. For convenience
we will make use of the Exner function \$Pi (p)\$ which defined by:
begin\{equation\}
Pi (p)=c\_\{p\}(frac\{p\}\{p\_\{c\}\})\textasciicircum{}\{kappa \}  label\{Exner\}
end\{equation\}
The following relations will be useful and are easily expressed in terms of
the Exner function:
begin\{equation*\}
c\_\{p\}T=Pi theta ;;;;;frac\{partial Pi \}\{partial p\}=frac\{kappa Pi
\}\{p\};;;;;alpha =frac\{kappa Pi theta \}\{p\}=frac\{partial Pi \}\{
partial p\}theta ;;;;;frac\{DPi \}\{Dt\}=frac\{partial Pi \}\{partial p\}
frac\{Dp\}\{Dt\}
end\{equation*\}
where \$b=frac\{partial Pi \}\{partial p\}theta \$ is the buoyancy.

The heat equation is obtained by noting that
begin\{equation*\}
c\_\{p\}frac\{DT\}\{Dt\}=frac\{D(Pi theta )\}\{Dt\}=Pi frac\{Dtheta \}\{Dt\}+theta
frac\{DPi \}\{Dt\}=Pi frac\{Dtheta \}\{Dt\}+alpha frac\{Dp\}\{Dt\}
end\{equation*\}
and on substituting into (ref\{eq-p-heat-interim\}) gives:
begin\{equation\}
Pi frac\{Dtheta \}\{Dt\}=mathcal\{Q\}
label\{eq:potential-temperature-equation\}
end\{equation\}
which is in conservative form.

For convenience in the model we prefer to step forward (ref
\{eq:potential-temperature-equation\}) rather than (ref\{eq:atmos-heat\}).

subsubsection\{Boundary conditions\}

The upper and lower boundary conditions are :
begin\{eqnarray\}
mbox\{at the top:\};;p=0 \&\&text\{, \}omega =frac\{Dp\}\{Dt\}=0 \textbackslash{}
mbox\{at the surface:\};;p=p\_\{s\} \&\&text\{, \}phi =phi \_\{topo\}=g\textasciitilde{}Z\_\{topo\}
label\{eq:boundary-condition-atmosphere\}
end\{eqnarray\}
In \$p\$-coordinates, the upper boundary acts like a solid boundary (\$omega
=0 \$); in \$z\$-coordinates and the lower boundary is analogous to a free
surface (\$phi \$ is imposed and \$omega neq 0\$).

subsubsection\{Splitting the geo-potential\}
label\{sec:hpe-p-geo-potential-split\}

For the purposes of initialization and reducing round-off errors, the model
deals with perturbations from reference (or {\color{red}\bfseries{}{}`{}`}standard'`) profiles. For
example, the hydrostatic geopotential associated with the resting atmosphere
is not dynamically relevant and can therefore be subtracted from the
equations. The equations written in terms of perturbations are obtained by
substituting the following definitions into the previous model equations:
begin\{eqnarray\}
theta \&=\&theta \_\{o\}+theta \textasciicircum{}\{prime \}  label\{eq:atmos-ref-prof-theta\} \textbackslash{}
alpha \&=\&alpha \_\{o\}+alpha \textasciicircum{}\{prime \}  label\{eq:atmos-ref-prof-alpha\} \textbackslash{}
phi \&=\&phi \_\{o\}+phi \textasciicircum{}\{prime \}  label\{eq:atmos-ref-prof-phi\}
end\{eqnarray\}
The reference state (indicated by subscript {\color{red}\bfseries{}{}`{}`}0'`) corresponds to
horizontally homogeneous atmosphere at rest (\$theta \_\{o\},alpha \_\{o\},phi
\_\{o\}\$) with surface pressure \$p\_\{o\}(x,y)\$ that satisfies \$phi
\_\{o\}(p\_\{o\})=g\textasciitilde{}Z\_\{topo\}\$, defined:
begin\{eqnarray*\}
theta \_\{o\}(p) \&=\&f\textasciicircum{}\{n\}(p) \textbackslash{}
alpha \_\{o\}(p) \&=\&Pi \_\{p\}theta \_\{o\} \textbackslash{}
phi \_\{o\}(p) \&=\&phi \_\{topo\}-int\_\{p\_\{0\}\}\textasciicircum{}\{p\}alpha \_\{o\}dp
end\{eqnarray*\}
\%begin\{eqnarray*\}
\%phi'\_alpha \& = \& int\textasciicircum{}p\_\{p\_o\} (alpha\_o -alpha) dp \textbackslash{}
\%phi'\_s(x,y,t) \& = \& int\_\{p\_o\}\textasciicircum{}\{p\_s\} alpha dp
\%end\{eqnarray*\}

The final form of the HPE's in p coordinates is then:
begin\{eqnarray\}
frac\{Dvec\{mathbf\{v\}\}\_\{h\}\}\{Dt\}+fhat\{mathbf\{k\}\}times vec\{mathbf\{v\}\}
\_\{h\}+mathbf\{nabla \}\_\{p\}phi \textasciicircum{}\{prime \} \&=\&vec\{mathbf\{mathcal\{F\}\}\}
label\{eq:atmos-prime\} \textbackslash{}
frac\{partial phi \textasciicircum{}\{prime \}\}\{partial p\}+alpha \textasciicircum{}\{prime \} \&=\&0 \textbackslash{}
mathbf\{nabla \}\_\{p\}cdot vec\{mathbf\{v\}\}\_\{h\}+frac\{partial omega \}\{
partial p\} \&=\&0 \textbackslash{}
frac\{partial Pi \}\{partial p\}theta \textasciicircum{}\{prime \} \&=\&alpha \textasciicircum{}\{prime \} \textbackslash{}
frac\{Dtheta \}\{Dt\} \&=\&frac\{mathcal\{Q\}\}\{Pi \}
end\{eqnarray\}

section\{Appendix OCEAN\}

subsection\{Equations of motion for the ocean\}

We review here the method by which the standard (Boussinesq, incompressible)
HPE's for the ocean written in z-coordinates are obtained. The
non-Boussinesq equations for oceanic motion are:
begin\{eqnarray\}
frac\{Dvec\{mathbf\{v\}\}\_\{h\}\}\{Dt\}+fhat\{mathbf\{k\}\}times vec\{mathbf\{v\}\}
\_\{h\}+frac\{1\}\{rho \}mathbf\{nabla \}\_\{z\}p \&=\&vec\{mathbf\{mathcal\{F\}\}\} \textbackslash{}
epsilon \_\{nh\}frac\{Dw\}\{Dt\}+g+frac\{1\}\{rho \}frac\{partial p\}\{partial z\}
\&=\&epsilon \_\{nh\}mathcal\{F\}\_\{w\} \textbackslash{}
frac\{1\}\{rho \}frac\{Drho \}\{Dt\}+mathbf\{nabla \}\_\{z\}cdot vec\{mathbf\{v\}\}
\_\{h\}+frac\{partial w\}\{partial z\} \&=\&0 label\{eq-zns-cont\}\textbackslash{}
rho \&=\&rho (theta ,S,p) label\{eq-zns-eos\}\textbackslash{}
frac\{Dtheta \}\{Dt\} \&=\&mathcal\{Q\}\_\{theta \} label\{eq-zns-heat\}\textbackslash{}
frac\{DS\}\{Dt\} \&=\&mathcal\{Q\}\_\{s\}  label\{eq-zns-salt\}
label\{eq:non-boussinesq\}
end\{eqnarray\}
These equations permit acoustics modes, inertia-gravity waves,
non-hydrostatic motions, a geostrophic (Rossby) mode and a thermohaline
mode. As written, they cannot be integrated forward consistently - if we
step \$rho \$ forward in (ref\{eq-zns-cont\}), the answer will not be
consistent with that obtained by stepping (ref\{eq-zns-heat\}) and (ref
\{eq-zns-salt\}) and then using (ref\{eq-zns-eos\}) to yield \$rho \$. It is
therefore necessary to manipulate the system as follows. Differentiating the
EOS (equation of state) gives:

begin\{equation\}
frac\{Drho \}\{Dt\}=left. frac\{partial rho \}\{partial theta \}right\textbar{}
\_\{S,p\}frac\{Dtheta \}\{Dt\}+left. frac\{partial rho \}\{partial S\}right\textbar{}
\_\{theta ,p\}frac\{DS\}\{Dt\}+left. frac\{partial rho \}\{partial p\}right\textbar{}
\_\{theta ,S\}frac\{Dp\}\{Dt\}  label\{EOSexpansion\}
end\{equation\}

Note that \$frac\{partial rho \}\{partial p\}=frac\{1\}\{c\_\{s\}\textasciicircum{}\{2\}\}\$ is
the reciprocal of the sound speed (\$c\_\{s\}\$) squared. Substituting into
ref\{eq-zns-cont\} gives:
begin\{equation\}
frac\{1\}\{rho c\_\{s\}\textasciicircum{}\{2\}\}frac\{Dp\}\{Dt\}+mathbf\{nabla \}\_\{z\}cdot vec\{mathbf\{
v\}\}+partial \_\{z\}wapprox 0  label\{eq-zns-pressure\}
end\{equation\}
where we have used an approximation sign to indicate that we have assumed
adiabatic motion, dropping the \$frac\{Dtheta \}\{Dt\}\$ and \$frac\{DS\}\{Dt\}\$.
Replacing ref\{eq-zns-cont\} with ref\{eq-zns-pressure\} yields a system that
can be explicitly integrated forward:
begin\{eqnarray\}
frac\{Dvec\{mathbf\{v\}\}\_\{h\}\}\{Dt\}+fhat\{mathbf\{k\}\}times vec\{mathbf\{v\}\}
\_\{h\}+frac\{1\}\{rho \}mathbf\{nabla \}\_\{z\}p \&=\&vec\{mathbf\{mathcal\{F\}\}\}
label\{eq-cns-hmom\} \textbackslash{}
epsilon \_\{nh\}frac\{Dw\}\{Dt\}+g+frac\{1\}\{rho \}frac\{partial p\}\{partial z\}
\&=\&epsilon \_\{nh\}mathcal\{F\}\_\{w\}  label\{eq-cns-hydro\} \textbackslash{}
frac\{1\}\{rho c\_\{s\}\textasciicircum{}\{2\}\}frac\{Dp\}\{Dt\}+mathbf\{nabla \}\_\{z\}cdot vec\{mathbf\{
v\}\}\_\{h\}+frac\{partial w\}\{partial z\} \&=\&0  label\{eq-cns-cont\} \textbackslash{}
rho \&=\&rho (theta ,S,p)  label\{eq-cns-eos\} \textbackslash{}
frac\{Dtheta \}\{Dt\} \&=\&mathcal\{Q\}\_\{theta \}  label\{eq-cns-heat\} \textbackslash{}
frac\{DS\}\{Dt\} \&=\&mathcal\{Q\}\_\{s\}  label\{eq-cns-salt\}
end\{eqnarray\}

subsubsection\{Compressible z-coordinate equations\}

Here we linearize the acoustic modes by replacing \$rho \$ with \$rho \_\{o\}(z)\$
wherever it appears in a product (ie. non-linear term) - this is the
{\color{red}\bfseries{}{}`}Boussinesq assumption'. The only term that then retains the full variation
in \$rho \$ is the gravitational acceleration:
begin\{eqnarray\}
frac\{Dvec\{mathbf\{v\}\}\_\{h\}\}\{Dt\}+fhat\{mathbf\{k\}\}times vec\{mathbf\{v\}\}
\_\{h\}+frac\{1\}\{rho \_\{o\}\}mathbf\{nabla \}\_\{z\}p \&=\&vec\{mathbf\{mathcal\{F\}\}\}
label\{eq-zcb-hmom\} \textbackslash{}
epsilon \_\{nh\}frac\{Dw\}\{Dt\}+frac\{grho \}\{rho \_\{o\}\}+frac\{1\}\{rho \_\{o\}\}
frac\{partial p\}\{partial z\} \&=\&epsilon \_\{nh\}mathcal\{F\}\_\{w\}
label\{eq-zcb-hydro\} \textbackslash{}
frac\{1\}\{rho \_\{o\}c\_\{s\}\textasciicircum{}\{2\}\}frac\{Dp\}\{Dt\}+mathbf\{nabla \}\_\{z\}cdot vec\{
mathbf\{v\}\}\_\{h\}+frac\{partial w\}\{partial z\} \&=\&0  label\{eq-zcb-cont\} \textbackslash{}
rho \&=\&rho (theta ,S,p)  label\{eq-zcb-eos\} \textbackslash{}
frac\{Dtheta \}\{Dt\} \&=\&mathcal\{Q\}\_\{theta \}  label\{eq-zcb-heat\} \textbackslash{}
frac\{DS\}\{Dt\} \&=\&mathcal\{Q\}\_\{s\}  label\{eq-zcb-salt\}
end\{eqnarray\}
These equations still retain acoustic modes. But, because the
{\color{red}\bfseries{}{}`{}`}compressible'' terms are linearized, the pressure equation ref
\{eq-zcb-cont\} can be integrated implicitly with ease (the time-dependent
term appears as a Helmholtz term in the non-hydrostatic pressure equation).
These are the emph\{truly\} compressible Boussinesq equations. Note that the
EOS must have the same pressure dependency as the linearized pressure term,
ie. \$left. frac\{partial rho \}\{partial p\}right\textbar{} \_\{theta ,S\}=frac\{1\}\{
c\_\{s\}\textasciicircum{}\{2\}\}\$, for consistency.

subsubsection\{{\color{red}\bfseries{}{}`}Anelastic' z-coordinate equations\}

The anelastic approximation filters the acoustic mode by removing the
time-dependency in the continuity (now pressure-) equation (ref\{eq-zcb-cont\}
). This could be done simply by noting that \$frac\{Dp\}\{Dt\}approx -grho \_\{o\}
frac\{Dz\}\{Dt\}=-grho \_\{o\}w\$, but this leads to an inconsistency between
continuity and EOS. A better solution is to change the dependency on
pressure in the EOS by splitting the pressure into a reference function of
height and a perturbation:
begin\{equation*\}
rho =rho (theta ,S,p\_\{o\}(z)+epsilon \_\{s\}p\textasciicircum{}\{prime \})
end\{equation*\}
Remembering that the term \$frac\{Dp\}\{Dt\}\$ in continuity comes from
differentiating the EOS, the continuity equation then becomes:
begin\{equation*\}
frac\{1\}\{rho \_\{o\}c\_\{s\}\textasciicircum{}\{2\}\}left( frac\{Dp\_\{o\}\}\{Dt\}+epsilon \_\{s\}frac\{
Dp\textasciicircum{}\{prime \}\}\{Dt\}right) +mathbf\{nabla \}\_\{z\}cdot vec\{mathbf\{v\}\}\_\{h\}+
frac\{partial w\}\{partial z\}=0
end\{equation*\}
If the time- and space-scales of the motions of interest are longer than
those of acoustic modes, then \$frac\{Dp\textasciicircum{}\{prime \}\}\{Dt\}\textless{}\textless{}(frac\{Dp\_\{o\}\}\{Dt\},
mathbf\{nabla \}cdot vec\{mathbf\{v\}\}\_\{h\})\$ in the continuity equations and
\$left. frac\{partial rho \}\{partial p\}right\textbar{} \_\{theta ,S\}frac\{
Dp\textasciicircum{}\{prime \}\}\{Dt\}\textless{}\textless{}left. frac\{partial rho \}\{partial p\}right\textbar{} \_\{theta
,S\}frac\{Dp\_\{o\}\}\{Dt\}\$ in the EOS (ref\{EOSexpansion\}). Thus we set \$epsilon
\_\{s\}=0\$, removing the dependency on \$p\textasciicircum{}\{prime \}\$ in the continuity equation
and EOS. Expanding \$frac\{Dp\_\{o\}(z)\}\{Dt\}=-grho \_\{o\}w\$ then leads to the
anelastic continuity equation:
begin\{equation\}
mathbf\{nabla \}\_\{z\}cdot vec\{mathbf\{v\}\}\_\{h\}+frac\{partial w\}\{partial z\}-
frac\{g\}\{c\_\{s\}\textasciicircum{}\{2\}\}w=0  label\{eq-za-cont1\}
end\{equation\}
A slightly different route leads to the quasi-Boussinesq continuity equation
where we use the scaling \$frac\{partial rho \textasciicircum{}\{prime \}\}\{partial t\}+
mathbf\{nabla \}\_\{3\}cdot rho \textasciicircum{}\{prime \}vec\{mathbf\{v\}\}\textless{}\textless{}mathbf\{nabla \}
\_\{3\}cdot rho \_\{o\}vec\{mathbf\{v\}\}\$ yielding:
begin\{equation\}
mathbf\{nabla \}\_\{z\}cdot vec\{mathbf\{v\}\}\_\{h\}+frac\{1\}\{rho \_\{o\}\}frac\{
partial left( rho \_\{o\}wright) \}\{partial z\}=0  label\{eq-za-cont2\}
end\{equation\}
Equations ref\{eq-za-cont1\} and ref\{eq-za-cont2\} are in fact the same
equation if:
begin\{equation\}
frac\{1\}\{rho \_\{o\}\}frac\{partial rho \_\{o\}\}\{partial z\}=frac\{-g\}\{c\_\{s\}\textasciicircum{}\{2\}\}
end\{equation\}
Again, note that if \$rho \_\{o\}\$ is evaluated from prescribed \$theta \_\{o\}\$
and \$S\_\{o\}\$ profiles, then the EOS dependency on \$p\_\{o\}\$ and the term \$frac\{
g\}\{c\_\{s\}\textasciicircum{}\{2\}\}\$ in continuity should be referred to those same profiles. The
full set of {\color{red}\bfseries{}{}`}quasi-Boussinesq' or {\color{red}\bfseries{}{}`}anelastic' equations for the ocean are
then:
begin\{eqnarray\}
frac\{Dvec\{mathbf\{v\}\}\_\{h\}\}\{Dt\}+fhat\{mathbf\{k\}\}times vec\{mathbf\{v\}\}
\_\{h\}+frac\{1\}\{rho \_\{o\}\}mathbf\{nabla \}\_\{z\}p \&=\&vec\{mathbf\{mathcal\{F\}\}\}
label\{eq-zab-hmom\} \textbackslash{}
epsilon \_\{nh\}frac\{Dw\}\{Dt\}+frac\{grho \}\{rho \_\{o\}\}+frac\{1\}\{rho \_\{o\}\}
frac\{partial p\}\{partial z\} \&=\&epsilon \_\{nh\}mathcal\{F\}\_\{w\}
label\{eq-zab-hydro\} \textbackslash{}
mathbf\{nabla \}\_\{z\}cdot vec\{mathbf\{v\}\}\_\{h\}+frac\{1\}\{rho \_\{o\}\}frac\{
partial left( rho \_\{o\}wright) \}\{partial z\} \&=\&0  label\{eq-zab-cont\} \textbackslash{}
rho \&=\&rho (theta ,S,p\_\{o\}(z))  label\{eq-zab-eos\} \textbackslash{}
frac\{Dtheta \}\{Dt\} \&=\&mathcal\{Q\}\_\{theta \}  label\{eq-zab-heat\} \textbackslash{}
frac\{DS\}\{Dt\} \&=\&mathcal\{Q\}\_\{s\}  label\{eq-zab-salt\}
end\{eqnarray\}

subsubsection\{Incompressible z-coordinate equations\}

Here, the objective is to drop the depth dependence of \$rho \_\{o\}\$ and so,
technically, to also remove the dependence of \$rho \$ on \$p\_\{o\}\$. This would
yield the {\color{red}\bfseries{}{}`{}`}truly'' incompressible Boussinesq equations:
begin\{eqnarray\}
frac\{Dvec\{mathbf\{v\}\}\_\{h\}\}\{Dt\}+fhat\{mathbf\{k\}\}times vec\{mathbf\{v\}\}
\_\{h\}+frac\{1\}\{rho \_\{c\}\}mathbf\{nabla \}\_\{z\}p \&=\&vec\{mathbf\{mathcal\{F\}\}\}
label\{eq-ztb-hmom\} \textbackslash{}
epsilon \_\{nh\}frac\{Dw\}\{Dt\}+frac\{grho \}\{rho \_\{c\}\}+frac\{1\}\{rho \_\{c\}\}
frac\{partial p\}\{partial z\} \&=\&epsilon \_\{nh\}mathcal\{F\}\_\{w\}
label\{eq-ztb-hydro\} \textbackslash{}
mathbf\{nabla \}\_\{z\}cdot vec\{mathbf\{v\}\}\_\{h\}+frac\{partial w\}\{partial z\}
\&=\&0  label\{eq-ztb-cont\} \textbackslash{}
rho \&=\&rho (theta ,S)  label\{eq-ztb-eos\} \textbackslash{}
frac\{Dtheta \}\{Dt\} \&=\&mathcal\{Q\}\_\{theta \}  label\{eq-ztb-heat\} \textbackslash{}
frac\{DS\}\{Dt\} \&=\&mathcal\{Q\}\_\{s\}  label\{eq-ztb-salt\}
end\{eqnarray\}
where \$rho \_\{c\}\$ is a constant reference density of water.

subsubsection\{Compressible non-divergent equations\}

The above {\color{red}\bfseries{}{}`{}`}incompressible'' equations are incompressible in both the flow
and the density. In many oceanic applications, however, it is important to
retain compressibility effects in the density. To do this we must split the
density thus:
begin\{equation*\}
rho =rho \_\{o\}+rho \textasciicircum{}\{prime \}
end\{equation*\}
We then assert that variations with depth of \$rho \_\{o\}\$ are unimportant
while the compressible effects in \$rho \textasciicircum{}\{prime \}\$ are:
begin\{equation*\}
rho \_\{o\}=rho \_\{c\}
end\{equation*\}
begin\{equation*\}
rho \textasciicircum{}\{prime \}=rho (theta ,S,p\_\{o\}(z))-rho \_\{o\}
end\{equation*\}
This then yields what we can call the semi-compressible Boussinesq
equations:
begin\{eqnarray\}
frac\{Dvec\{mathbf\{v\}\}\_\{h\}\}\{Dt\}+fhat\{mathbf\{k\}\}times vec\{mathbf\{v\}\}
\_\{h\}+frac\{1\}\{rho \_\{c\}\}mathbf\{nabla \}\_\{z\}p\textasciicircum{}\{prime \} \&=\&vec\{mathbf\{
mathcal\{F\}\}\}  label\{eq:ocean-mom\} \textbackslash{}
epsilon \_\{nh\}frac\{Dw\}\{Dt\}+frac\{grho \textasciicircum{}\{prime \}\}\{rho \_\{c\}\}+frac\{1\}\{rho
\_\{c\}\}frac\{partial p\textasciicircum{}\{prime \}\}\{partial z\} \&=\&epsilon \_\{nh\}mathcal\{F\}\_\{w\}
label\{eq:ocean-wmom\} \textbackslash{}
mathbf\{nabla \}\_\{z\}cdot vec\{mathbf\{v\}\}\_\{h\}+frac\{partial w\}\{partial z\}
\&=\&0  label\{eq:ocean-cont\} \textbackslash{}
rho \textasciicircum{}\{prime \} \&=\&rho (theta ,S,p\_\{o\}(z))-rho \_\{c\}  label\{eq:ocean-eos\}
\textbackslash{}
frac\{Dtheta \}\{Dt\} \&=\&mathcal\{Q\}\_\{theta \}  label\{eq:ocean-theta\} \textbackslash{}
frac\{DS\}\{Dt\} \&=\&mathcal\{Q\}\_\{s\}  label\{eq:ocean-salt\}
end\{eqnarray\}
Note that the hydrostatic pressure of the resting fluid, including that
associated with \$rho \_\{c\}\$, is subtracted out since it has no effect on the
dynamics.

Though necessary, the assumptions that go into these equations are messy
since we essentially assume a different EOS for the reference density and
the perturbation density. Nevertheless, it is the hydrostatic (\$epsilon
\_\{nh\}=0\$ form of these equations that are used throughout the ocean modeling
community and referred to as the primitive equations (HPE).

section\{Appendix:OPERATORS\}

subsection\{Coordinate systems\}

subsubsection\{Spherical coordinates\}

In spherical coordinates, the velocity components in the zonal, meridional
and vertical direction respectively, are given by (see Fig.2) :

begin\{equation*\}
u=rcos varphi frac\{Dlambda \}\{Dt\}
end\{equation*\}

begin\{equation*\}
v=rfrac\{Dvarphi \}\{Dt\}
end\{equation*\}

begin\{equation*\}
dot\{r\}=frac\{Dr\}\{Dt\}
end\{equation*\}

Here \$varphi \$ is the latitude, \$lambda \$ the longitude, \$r\$ the radial
distance of the particle from the center of the earth, \$Omega \$ is the
angular speed of rotation of the Earth and \$D/Dt\$ is the total derivative.

The {\color{red}\bfseries{}{}`}grad' (\$nabla \$) and {\color{red}\bfseries{}{}`}div' (\$nablacdot\$) operators are defined by, in
spherical coordinates:

begin\{equation*\}
nabla equiv left( frac\{1\}\{rcos varphi \}frac\{partial \}\{partial lambda \}
,frac\{1\}\{r\}frac\{partial \}\{partial varphi \},frac\{partial \}\{partial r\}
right)
end\{equation*\}

begin\{equation*\}
nablacdot vequiv frac\{1\}\{rcos varphi \}left\{ frac\{partial u\}\{partial
lambda \}+frac\{partial \}\{partial varphi \}left( vcos varphi right) right\}
+frac\{1\}\{r\textasciicircum{}\{2\}\}frac\{partial left( r\textasciicircum{}\{2\}dot\{r\}right) \}\{partial r\}
end\{equation*\}


\chapter{Physical Parameterizations - Packages I}
\label{\detokenize{phys_pkgs/phys_pkgs:packagesi}}\label{\detokenize{phys_pkgs/phys_pkgs:physical-parameterizations-packages-i}}\label{\detokenize{phys_pkgs/phys_pkgs::doc}}
In this chapter and in the following chapter, the MITgcm `packages' are
described. While you can carry out many experiments with MITgcm by starting
from case studies in section ref\{sec:modelExamples\}, configuring
a brand new experiment or making major changes to an experimental configuration
requires some knowledge of the \sphinxstyleemphasis{packages}
that make up the full MITgcm code. Packages are used in MITgcm to
help organize and layer various code building blocks that are assembled
and selected to perform a specific experiment. Each of the specific experiments
described in section ref\{sec:modelExamples\} uses a particular combination
of packages.

\hyperref[\detokenize{phys_pkgs/phys_pkgs:package-organigramme}]{Figure \ref{\detokenize{phys_pkgs/phys_pkgs:package-organigramme}}} shows the full set of packages that
are available. As shown in the figure packages are classified into different
groupings that layer on top of each other. The top layer packages are
generally specialized to specific simulation types. In this layer there are
packages that deal with biogeochemical processes, ocean interior
and boundary layer processes, atmospheric processes, sea-ice, coupled
simulations and state estimation.
Below this layer are a set of general purpose
numerical and computational packages. The general purpose numerical packages
provide code for kernel numerical alogorithms
that apply to
many different simulation types. Similarly, the general purpose computational
packages implement non-numerical alogorithms that provide parallelism,
I/O and time-keeping functions that are used in many different scenarios.
\begin{quote}
\begin{figure}[htbp]
\centering
\capstart

\noindent\sphinxincludegraphics[width=0.700\linewidth]{{mitgcm_package_organisation}.pdf}
\caption{Hierarchy of code layers that are assembled to make up an MITgcm simulation. Conceptually (and in terms of code organization) MITgcm consists of several layers. At the base is a layer of core software that provides a basic numerical and computational foundation for MITgcm simulations. This layer is shown marked \sphinxstyleemphasis{Foundation Code} at the bottom of the figure and corresponds to code in the italicised subdirectories on the figure. This layer is not organized into packages. All code above the foundation layer is organized as packages.  Much of the code in MITgcm is contained in packages which serve as a useful way of organizing and layering the different levels of functionality that make up the full MITgcm software distribution. The figure shows the different packages in MITgcm as boxes containing bold face upper case names.  Directly above the foundation layer are two layers of general purpose infrastructure software that consist of computational and numerical packages.  These general purpose packages can be applied to both online and offline simulations and are used in many different physical simulation types.  Above these layers are more specialized packages.}\label{\detokenize{phys_pkgs/phys_pkgs:package-organigramme}}\end{figure}
\end{quote}

The following sections describe the packages shown in
figure ref\{fig:package\_organigramme\}. Section ref\{sec:pkg:using\}
describes the general procedure for using any package in MITgcm.
Following that sections ref\{sec:pkg:gad\}-ref\{sec:pkg:monitor\}
layout the algorithms implemented in specific packages
and describe how to use the individual packages. A brief synopsis of the
function of each package is given in table ref\{tab:package\_summary\_tab\}.
Organizationally package code is assigned a
separate subdirectory in the MITgcm code distribution
(within the source code directory texttt\{pkg\}).
The name of this subdirectory is used as the package name in
table ref\{tab:package\_summary\_tab\}.

\%\% In this chapter the schemes for parameterizing processes that are not
\%\% represented explicitly in MITgcm are described.  Some of these
\%\% processes are sub-grid scale (SGS) phenomena, other processes, such as
\%\% open-boundaries, are external to the simulation.

begin\{table\}
caption\{\textasciitilde{}\}
label\{tab:package\_summary\_tab\}.
end\{table\}

\% Overview
newpage
input\{s\_phys\_pkgs/text/packages.tex\}


\section{Packages Related to Hydrodynamical Kernel}
\label{\detokenize{phys_pkgs/phys_pkgs:packages-related-to-hydrodynamical-kernel}}
input\{s\_phys\_pkgs/text/generic\_advdiff.tex\}

newpage
input\{s\_phys\_pkgs/text/shap\_filt.tex\}

newpage
input\{s\_phys\_pkgs/text/zonal\_filt.tex\}

newpage
input\{s\_phys\_pkgs/text/exch2.tex\}

newpage
input\{s\_phys\_pkgs/text/gridalt.tex\}

\% Some Mention of Packages that are part of the main model document
newpage


\section{General purpose numerical infrastructure packages}
\label{\detokenize{phys_pkgs/phys_pkgs:general-purpose-numerical-infrastructure-packages}}

\subsection{OBCS: Open boundary conditions for regional modeling}
\label{\detokenize{phys_pkgs/phys_pkgs:pkg-obcs}}\label{\detokenize{phys_pkgs/phys_pkgs:obcs-open-boundary-conditions-for-regional-modeling}}
Authors:
Alistair Adcroft, Patrick Heimbach, Samar Katiwala, Martin Losch


\subsubsection{Introduction}
\label{\detokenize{phys_pkgs/phys_pkgs:introduction}}\label{\detokenize{phys_pkgs/phys_pkgs:pkg-obcs-intro}}
The OBCS-package is fundamental to regional ocean modelling with the
MITgcm, but there are so many details to be considered in
regional ocean modelling that this package cannot accomodate all
imaginable and possible options. Therefore, for a regional simulation
with very particular details, it is recommended to familiarize oneself
not only with the compile- and runtime-options of this package, but
also with the code itself. In many cases it will be necessary to adapt
the obcs-code (in particular code\{S/R OBCS\_CALC\}) to the application
in question; in these cases the obcs-package (together with the
rbcs-package, section ref\{sec:pkg:rbcs\}) is a very
useful infrastructure for implementing special regional models.


\subsubsection{OBCS configuration and compiling}
\label{\detokenize{phys_pkgs/phys_pkgs:pkg-obcs-config-compiling}}\label{\detokenize{phys_pkgs/phys_pkgs:obcs-configuration-and-compiling}}
As with all MITgcm packages, OBCS can be turned on or off
at compile time
\begin{itemize}
\item {} 
using the \sphinxcode{packages.conf} file by adding \sphinxcode{obcs} to it,

\item {} 
or using \sphinxcode{genmake2} adding \sphinxcode{-enable=obcs} or \sphinxcode{-disable=obcs} switches

\item {} 
\sphinxstyleemphasis{Required packages and CPP options:}
\begin{itemize}
\item {} 
Two alternatives are available for prescribing open boundary values, which differ in the way how OB's are treated in time:
\begin{itemize}
\item {} 
A simple time-management (e.g. constant in time, or cyclic with fixed fequency) is provided through S/R \sphinxcode{obcs\_external\_fields\_load}.

\item {} 
More sophisticated `real-time' (i.e. calendar time) management is available through \sphinxcode{obcs\_prescribe\_read}.

\end{itemize}

\item {} 
The latter case requires packages \sphinxcode{cal} and \sphinxcode{exf} to be enabled.

\end{itemize}

\end{itemize}

(see also Section ref\{sec:buildingCode\}).

Parts of the OBCS code can be enabled or disabled at compile time
via CPP preprocessor flags. These options are set in
\sphinxtitleref{OBCS\_OPTIONS.h}. \hyperref[\detokenize{phys_pkgs/phys_pkgs:pkg-obcs-cpp-opts}]{Table \ref{\detokenize{phys_pkgs/phys_pkgs:pkg-obcs-cpp-opts}}} summarizes these options.


\begin{threeparttable}
\capstart\caption{OBCS CPP options}\label{\detokenize{phys_pkgs/phys_pkgs:pkg-obcs-cpp-opts}}\label{\detokenize{phys_pkgs/phys_pkgs:id13}}
\noindent\begin{tabulary}{\linewidth}{|l|l|}
\hline

\sphinxstylestrong{CPP option}
&
\sphinxstylestrong{Description}
\\
\hline
\sphinxtitleref{ALLOW\_OBCS\_NORTH}
&
enable Northern OB
\\
\hline
\sphinxtitleref{ALLOW\_OBCS\_SOUTH}
&
enable Southern OB
\\
\hline
ALLOW\_OBCS\_EAST
&
enable Eastern OB
\\
\hline
ALLOW\_OBCS\_WEST
&
enable Western OB
\\
\hline&\\
\hline
ALLOW\_OBCS\_PRESCRIBE
&
enable code for prescribing OB's
\\
\hline
ALLOW\_OBCS\_SPONGE
&
enable sponge layer code
\\
\hline
ALLOW\_OBCS\_BALANCE
&
enable code for balancing transports through OB's
\\
\hline
ALLOW\_ORLANSKI
&
enable Orlanski radiation conditions at OB's
\\
\hline
ALLOW\_OBCS\_STEVENS
&
enable Stevens (1990) boundary conditions at OB's
\\
\hline&
(currently only implemented for eastern and
\\
\hline&
western boundaries and NOT for ptracers)
\\
\hline\end{tabulary}

\end{threeparttable}



\subsubsection{Run-time parameters}
\label{\detokenize{phys_pkgs/phys_pkgs:pkg-obcs-runtime}}\label{\detokenize{phys_pkgs/phys_pkgs:run-time-parameters}}
Run-time parameters are set in files
\sphinxcode{data.pkg}, \sphinxcode{data.obcs}, and \sphinxcode{data.exf
if 'real-time' prescription is requested
(i.e. package :code:{}`exf} enabled).
These parameter files are read in S/R
\sphinxcode{packages\_readparms.F}, \sphinxcode{obcs\_readparms.F}, and
\sphinxcode{exf\_readparms.F}, respectively.
Run-time parameters may be broken into 3 categories:
\begin{enumerate}
\item {} 
switching on/off the package at runtime,

\item {} 
OBCS package flags and parameters,

\item {} 
additional timing flags in \sphinxcode{data.exf}, if selected.

\end{enumerate}


\paragraph{Enabling the package}
\label{\detokenize{phys_pkgs/phys_pkgs:enabling-the-package}}
The OBCS package is switched on at runtime by setting
\sphinxcode{useOBCS = .TRUE.} in \sphinxcode{data.pkg}.


\paragraph{Package flags and parameters}
\label{\detokenize{phys_pkgs/phys_pkgs:package-flags-and-parameters}}
\hyperref[\detokenize{phys_pkgs/phys_pkgs:pkg-obcs-runtime-flags}]{Table \ref{\detokenize{phys_pkgs/phys_pkgs:pkg-obcs-runtime-flags}}} summarizes the
runtime flags that are set in \sphinxcode{data.obcs}, and
their default values.

\begin{longtable}{|l|c|l|}
\caption{OBCS CPP options}\label{\detokenize{phys_pkgs/phys_pkgs:pkg-obcs-runtime-flags}}\label{\detokenize{phys_pkgs/phys_pkgs:id14}}\\
\hline
\endfirsthead

\multicolumn{3}{c}%
{\makebox[0pt]{\tablecontinued{\tablename\ \thetable{} -- continued from previous page}}}\\
\hline
\endhead

\hline \multicolumn{3}{|r|}{\makebox[0pt][r]{\tablecontinued{Continued on next page}}}\\\hline
\endfoot

\endlastfoot


\sphinxstylestrong{Flag/parameter}
&
\sphinxstylestrong{default}
&
\sphinxstylestrong{Description}
\\
\hline
\sphinxstyleemphasis{basic flags \& parameters} (OBCS\_PARM01)
&&\\
\hline
OB\_Jnorth
&
0
&
Nx-vector of J-indices (w.r.t. Ny) of Northern OB at each I-position (w.r.t. Nx)
\\
\hline
OB\_Jsouth
&
0
&
Nx-vector of J-indices (w.r.t. Ny) of Southern OB at each I-position (w.r.t. Nx)
\\
\hline
OB\_Ieast
&
0
&
Ny-vector of I-indices (w.r.t. Nx) of Eastern OB at each J-position (w.r.t. Ny)
\\
\hline
OB\_Iwest
&
0
&
Ny-vector of I-indices (w.r.t. Nx) of Western OB at each J-position (w.r.t. Ny)
\\
\hline
useOBCSprescribe
&
\sphinxcode{.FALSE.}
&\\
\hline
useOBCSsponge
&
\sphinxcode{.FALSE.}
&\\
\hline
useOBCSbalance \& code\{.FALSE.\} \&
&&\\
\hline
\textasciitilde{} \textbackslash{}
&&\\
\hline
OBCS\_balanceFacN/S/E/W \& 1 \& factor(s) determining the details
&&\\
\hline
of the balaning code \textbackslash{}
&&\\
\hline
useOrlanskiNorth/South/EastWest \& code\{.FALSE.\} \&
&&\\
\hline
turn on Orlanski boundary conditions for individual boundary\textbackslash{}
&&\\
\hline
useStevensNorth/South/EastWest \& code\{.FALSE.\} \&
&&\\
\hline
turn on Stevens boundary conditions for individual boundary\textbackslash{}
&&\\
\hline
OBtextbf\{X\}textbf\{y\}File \& \textasciitilde{} \&
&&\\
\hline
file name of OB field \textbackslash{}
&&\\
\hline
\textasciitilde{} \& \textasciitilde{} \&
&&\\
\hline
textbf\{X\}: textbf\{N\}(orth)
&
textbf\{S\}(outh)
&\\
\hline
textbf\{E\}(ast)
&
textbf\{W\}(est) \textbackslash{}
&\\
\hline
\textasciitilde{} \& \textasciitilde{} \&
&&\\
\hline
textbf\{y\}: textbf\{t\}(emperature)
&
textbf\{s\}(salinity)
&\\
\hline
textbf\{u\}(-velocity)
&
textbf\{v\}(-velocity)
&
\textbackslash{}
\\
\hline
\textasciitilde{} \& \textasciitilde{} \&
&&\\
\hline
textbf\{w\}(-velocity)
&
textbf\{eta\}(sea surface height)\textbackslash{}
&\\
\hline
\textasciitilde{} \& \textasciitilde{} \&
&&\\
\hline
textbf\{a\}(sea ice area)
&
textbf\{h\}(sea ice thickness)
&\\
\hline
textbf\{sn\}(snow thickness)
&
textbf\{sl\}(sea ice salinity)\textbackslash{}
&\\
\hline
hline
&&\\
\hline
multicolumn\{3\}\{{\color{red}\bfseries{}\textbar{}c\textbar{}}\}\{textit\{Orlanski parameters\} (OBCS\_PARM02) \} \textbackslash{}
&&\\
\hline
hline
&&\\
\hline
cvelTimeScale \& 2000 sec \&
&&\\
\hline
averaging period for phase speed \textbackslash{}
&&\\
\hline
CMAX \& 0.45 m/s \&
&&\\
\hline
maximum allowable phase speed-CFL for AB-II \textbackslash{}
&&\\
\hline
CFIX \& 0.8 m/s \&
&&\\
\hline
fixed boundary phase speed \textbackslash{}
&&\\
\hline
useFixedCEast \& code\{.FALSE.\} \&
&&\\
\hline
\textasciitilde{} \textbackslash{}
&&\\
\hline
useFixedCWest \& code\{.FALSE.\} \&
&&\\
\hline
\textasciitilde{} \textbackslash{}
&&\\
\hline
hline
&&\\
\hline
multicolumn\{3\}\{{\color{red}\bfseries{}\textbar{}c\textbar{}}\}\{textit\{Sponge-layer parameters\} (OBCS\_PARM03)\} \textbackslash{}
&&\\
\hline
hline
&&\\
\hline
spongeThickness \& 0 \&
&&\\
\hline
sponge layer thickness (in \# grid points) \textbackslash{}
&&\\
\hline
Urelaxobcsinner \& 0 sec \&
&&\\
\hline
relaxation time scale at the
&&\\
\hline
innermost sponge layer point of a meridional OB \textbackslash{}
&&\\
\hline
Vrelaxobcsinner \& 0 sec \&
&&\\
\hline
relaxation time scale at the
&&\\
\hline
innermost sponge layer point of a zonal OB \textbackslash{}
&&\\
\hline
Urelaxobcsbound \& 0 sec \&
&&\\
\hline
relaxation time scale at the
&&\\
\hline
outermost sponge layer point of a meridional OB \textbackslash{}
&&\\
\hline
Vrelaxobcsbound \& 0 sec \&
&&\\
\hline
relaxation time scale at the
&&\\
\hline
outermost sponge layer point of a zonal OB \textbackslash{}
&&\\
\hline
hline
&&\\
\hline
multicolumn\{3\}\{{\color{red}\bfseries{}\textbar{}c\textbar{}}\}\{textit\{Stevens parameters\} (OBCS\_PARM04) \} \textbackslash{}
&&\\
\hline
hline
&&\\
\hline
T/SrelaxStevens \& 0\textasciitilde{}sec \& relaxation time scale for
&&\\
\hline
temperature/salinity \textbackslash{}
&&\\
\hline
useStevensPhaseVel \& code\{.TRUE.\} \& \textbackslash{}
&&\\
\hline
useStevensAdvection \& code\{.TRUE.\} \& \textbackslash{}
&&\\
\hline
hline
&&\\
\hline
hline
&&\\
\hline
end\{tabular\}
&&\\
\hline
\}
&&\\
\hline
caption\{pkg OBCS run-time parameters\}
&&\\
\hline
label\{tab:pkg:obcs:runtime\_flags\}
&&\\
\hline\end{longtable}


end\{table\}

\%----------------------------------------------------------------------

subsubsection\{Defining open boundary positions
label\{sec:pkg:obcs:defining\}\}

There are four open boundaries (OBs), a
Northern, Southern, Eastern, and Western.
All OB locations are specified by their absolute
meridional (Northern/Southern) or zonal (Eastern/Western) indices.
Thus, for each zonal position \$i=1,ldots,N\_x\$ a meridional index
\$j\$ specifies the Northern/Southern OB position,
and for each meridional position \$j=1,ldots,N\_y\$, a zonal index
\$i\$ specifies the Eastern/Western OB position.
For Northern/Southern OB this defines an \$N\_x\$-dimensional
{\color{red}\bfseries{}{}`{}`}row'' array \$tt OB\_Jnorth(Nx)\$ / \$tt OB\_Jsouth(Nx)\$,
and an \$N\_y\$-dimenisonal
{\color{red}\bfseries{}{}`{}`}column'' array \$tt OB\_Ieast(Ny)\$ / \$tt OB\_Iwest(Ny)\$.
Positions determined in this way allows Northern/Southern
OBs to be at variable \$j\$ (or \$y\$) positions, and Eastern/Western
OBs at variable \$i\$ (or \$x\$) positions.
Here, indices refer to tracer points on the C-grid.
A zero (0) element in \$tt OB\_Ildots\$, \$tt OB\_Jldots\$
means there is no corresponding OB in that column/row.
For a Northern/Southern OB, the OB V point is to the South/North.
For an Eastern/Western OB, the OB U point is to the West/East.
For example,
begin\{tabbing\}
\begin{quote}

code\{OB\_Jnorth(3)=34\} =  means that:  = \textbackslash{}
\textgreater{} code\{T(3,34)\} \textgreater{} is a an OB point  \textbackslash{}
\textgreater{} code\{U(3,34)\} \textgreater{} is a an OB point \textbackslash{}
\textgreater{} code\{V(3,34)\} \textgreater{} is a an OB point \textbackslash{}
code\{OB\_Jsouth(3)=1\} \textgreater{} means that: \textbackslash{}
\textgreater{} code\{T(3,1)\} \textgreater{} is a an OB point \textbackslash{}
\textgreater{} code\{U(3,1)\} \textgreater{} is a an OB point \textbackslash{}
\textgreater{} code\{V(3,2)\} \textgreater{} is a an OB point \textbackslash{}
code\{OB\_Ieast(10)=69\} \textgreater{}  means that:  \textgreater{}  \textbackslash{}
\textgreater{} code\{T(69,10)\} \textgreater{} is a an OB point \textbackslash{}
\textgreater{} code\{U(69,10)\} \textgreater{} is a an OB point \textbackslash{}
\textgreater{} code\{V(69,10)\} \textgreater{} is a an OB point \textbackslash{}
code\{OB\_Iwest(10)=1\} \textgreater{}  means that:  \textgreater{}  \textbackslash{}
\textgreater{} code\{T(1,10)\} \textgreater{} is a an OB point \textbackslash{}
\textgreater{} code\{U(2,10)\} \textgreater{} is a an OB point \textbackslash{}
\textgreater{} code\{V(1,10)\} \textgreater{} is a an OB point
\end{quote}

end\{tabbing\}
For convenience, negative values for code\{Jnorth\}/code\{Ieast\} refer to
points relative to the Northern/Eastern edges of the model
eg. \$tt OB\_Jnorth(3)=-1\$  means that the point \$tt (3,Ny)\$
is a northern OB.

noindenttextbf\{Simple examples:\} For a model grid with \$ N\_\{x\}times
N\_\{y\} = 120times144\$ horizontal grid points with four open boundaries
along the four egdes of the domain, the simplest way of specifying the
boundary points in code\{data.obcs\} is:
begin\{verbatim\}
\begin{quote}

OB\_Ieast = 144*-1,
\end{quote}
\begin{description}
\item[{\# or OB\_Ieast = 144*120,}] \leavevmode
OB\_Iwest = 144*1,
OB\_Jnorth = 120*-1,

\item[{\# or OB\_Jnorth = 120*144,}] \leavevmode
OB\_Jsouth = 120*1,

\end{description}

end\{verbatim\}
If only the first \$50\$ grid points of the southern boundary are
boundary points:
begin\{verbatim\}
\begin{quote}

OB\_Jsouth(1:50) = 50*1,
\end{quote}

end\{verbatim\}

noindent
textsf\{Add special comments for case \#define NONLIN\_FRSURF,
see obcs\_ini\_fixed.F\}

\%----------------------------------------------------------------------


\subsection{Equations and key routines}
\label{\detokenize{phys_pkgs/phys_pkgs:equations-and-key-routines}}
paragraph\{OBCS\_READPARMS:\} \textasciitilde{} \textbackslash{}
Set OB positions through arrays
\{tt OB\_Jnorth(Nx), OB\_Jsouth(Nx), OB\_Ieast(Ny), OB\_Iwest(Ny)\},
and runtime flags (see Table ref\{tab:pkg:obcs:runtime\_flags\}).

paragraph\{OBCS\_CALC:\} \textasciitilde{} \textbackslash{}
\%
Top-level routine for filling values to be applied at OB for
\$T,S,U,V,eta\$ into corresponding
{\color{red}\bfseries{}{}`{}`}slice'' arrays \$(x,z)\$, \$(y,z)\$ for each OB:
\$tt OB{[}N/S/E/W{]}{[}t/s/u/v{]}\$; e.g. for salinity array at
Southern OB, array name is \$tt OBSt\$.
Values filled are either
\%
begin\{itemize\}
\%
item
constant vertical \$T,S\$ profiles as specified in file
\{tt data\} (\{tt tRef(Nr), sRef(Nr)\}) with zero velocities \$U,V\$,
\%
item
\$T,S,U,V\$ values determined via Orlanski radiation conditions
(see below),
\%
item
prescribed time-constant or time-varying fields (see below).
\%
item
use prescribed boundary fields to compute Stevens boundary conditions.
end\{itemize\}

paragraph\{ORLANSKI:\} \textasciitilde{} \textbackslash{}
\%
Orlanski radiation conditions citep\{orl:76\}, examples can be found in
code\{verification/dome\} and
code\{verification/tutorial\_plume\_on\_slope\}
(ref\{sec:eg-gravityplume\}).

paragraph\{OBCS\_PRESCRIBE\_READ:\} \textasciitilde{} \textbackslash{}
\%
When code\{useOBCSprescribe = .TRUE.\} the model tries to read
temperature, salinity, u- and v-velocities from files specified in the
runtime parameters code\{OB{[}N/S/E/W{]}{[}t/s/u/v{]}File\}. These files are
the usual IEEE, big-endian files with dimensions of a section along an
open boundary:
begin\{itemize\}
item For North/South boundary files the dimensions are
\begin{quote}

\$(N\_xtimes N\_rtimesmbox\{time levels\})\$, for East/West boundary
files the dimensions are \$(N\_ytimes N\_rtimesmbox\{time levels\})\$.
\end{quote}
\begin{description}
\item[{item If a non-linear free surface is used}] \leavevmode
(ref\{sec:nonlinear-freesurface\}), additional files
code\{OB{[}N/S/E/W{]}etaFile\} for the sea surface height \$eta\$ with
dimension \$(N\_\{x/y\}timesmbox\{time levels\})\$ may be specified.

\item[{item If non-hydrostatic dynamics are used}] \leavevmode
(ref\{sec:non-hydrostatic\}), additional files
code\{OB{[}N/S/E/W{]}wFile\} for the vertical velocity \$w\$ with
dimensions \$(N\_\{x/y\}times N\_rtimesmbox\{time levels\})\$ can be
specified.

\item[{item If code\{useSEAICE=.TRUE.\} then additional files}] \leavevmode
code\{OB{[}N/S/E/W{]}{[}a,h,sl,sn,uice,vice{]}\} for sea ice area, thickness
(code\{HEFF\}), seaice salinity, snow and ice velocities
\$(N\_\{x/y\}timesmbox\{time levels\})\$ can be specified.

\end{description}

end\{itemize\}
As in code\{S/R external\_fields\_load\} or the code\{exf\}-package, the
code reads two time levels for each variable, e.g.code\{OBNu0\} and
code\{OBNu1\}, and interpolates linearly between these time levels to
obtain the value code\{OBNu\} at the current model time (step). When the
code\{exf\}-package is used, the time levels are controlled for each
boundary separately in the same way as the code\{exf\}-fields in
code\{data.exf\}, namelist code\{EXF\_NML\_OBCS\}. The runtime flags
follow the above naming conventions, e.g. for the western boundary the
corresponding flags are code\{OBCWstartdate1/2\} and
code\{OBCWperiod\}. Sea-ice boundary values are controlled separately
with code\{siobWstartdate1/2\} and code\{siobWperiod\}.  When the
code\{exf\}-package is not used, the time levels are controlled by the
runtime flags code\{externForcingPeriod\} and code\{externForcingCycle\}
in code\{data\}, see code\{verification/exp4\} for an example.

paragraph\{OBCS\_CALC\_STEVENS:\} \textasciitilde{} \textbackslash{}
(THE IMPLEMENTATION OF THESE BOUNDARY CONDITIONS IS NOT
COMPLETE. PASSIVE TRACERS, SEA ICE AND NON-LINEAR FREE SURFACE ARE NOT
SUPPORTED PROPERLY.) \textbackslash{}
The boundary conditions following citet\{stevens:90\} require the
vertically averaged normal velocity (originally specified as a stream
function along the open boundary) \$bar\{u\}\_\{ob\}\$ and the tracer fields
\$chi\_\{ob\}\$ (note: passive tracers are currently not implemented and
the code stops when package code\{ptracers\} is used together with this
option). Currently, the code vertically averages the normal velocity
as specified in code\{OB{[}E,W{]}u\} or code\{OB{[}N,S{]}v\}. From these
prescribed values the code computes the boundary values for the next
timestep \$n+1\$ as follows (as an example, we use the notation for an
eastern or western boundary):
begin\{itemize\}
item \$u\textasciicircum{}\{n+1\}(y,z) = bar\{u\}\_\{ob\}(y) + (u')\textasciicircum{}\{n\}(y,z)\$, where
\begin{quote}

\$(u')\textasciicircum{}\{n\}\$ is the deviation from the vertically averaged velocity at
timestep \$n\$ on the boundary. \$(u')\textasciicircum{}\{n\}\$ is computed in the previous
time step \$n\$ from the intermediate velocity \$u\textasciicircum{}*\$ prior to the
correction step (see section ref\{sec:time\_stepping\}, e.g.,
eq.,(ref\{eq:ustar-backward-free-surface\})).
\% and\textasciitilde{}(ref\{eq:vstar-backward-free-surface\})).
(This velocity is not
available at the beginning of the next time step \$n+1\$, when
S/R\textasciitilde{}OBCS\_CALC/OBCS\_CALC\_STEVENS are called, therefore it needs to
be saved in S/R\textasciitilde{}DYNAMICS by calling S/R\textasciitilde{}OBCS\_SAVE\_UV\_N and also
stored in a separate restart files
verb+pickup\_stevens{[}N/S/E/W{]}.\$\{iteration\}.data+)
\end{quote}

\%  Define CPP-flag OBCS\_STEVENS\_USE\_INTERIOR\_VELOCITY to use the
\%  velocity one grid point inward from the boundary.
item If \$u\textasciicircum{}\{n+1\}\$ is directed into the model domain, the boudary
\begin{quote}

value for tracer \$chi\$ is restored to the prescribed values:
{[}chi\textasciicircum{}\{n+1\} =   chi\textasciicircum{}\{n\} + frac\{Delta\{t\}\}\{tau\_chi\} (chi\_\{ob\} -
chi\textasciicircum{}\{n\}),{]} where \$tau\_chi\$ is the relaxation time
scale texttt\{T/SrelaxStevens\}. The new \$chi\textasciicircum{}\{n+1\}\$ is then subject
to the advection by \$u\textasciicircum{}\{n+1\}\$.
\end{quote}
\begin{description}
\item[{item If \$u\textasciicircum{}\{n+1\}\$ is directed out of the model domain, the tracer}] \leavevmode
\$chi\textasciicircum{}\{n+1\}\$ on the boundary at timestep \$n+1\$ is estimated from
advection out of the domain with \$u\textasciicircum{}\{n+1\}+c\$, where \$c\$ is
a phase velocity estimated as
\$frac\{1\}\{2\}frac\{partialchi\}\{partial\{t\}\}/frac\{partialchi\}\{partial\{x\}\}\$. The
numerical scheme is (as an example for an eastern boundary):
{[}chi\_\{i\_\{b\},j,k\}\textasciicircum{}\{n+1\} =   chi\_\{i\_\{b\},j,k\}\textasciicircum{}\{n\} + Delta\{t\}
(u\textasciicircum{}\{n+1\}+c)\_\{i\_\{b\},j,k\}frac\{chi\_\{i\_\{b\},j,k\}\textasciicircum{}\{n\}
\begin{itemize}
\item {} 
chi\_\{i\_\{b\}-1,j,k\}\textasciicircum{}\{n\}\}\{Delta\{x\}\_\{i\_\{b\},j\}\textasciicircum{}\{C\}\}mbox\{, if \}u\_\{i\_\{b\},j,k\}\textasciicircum{}\{n+1\}\textgreater{}0,

\end{itemize}

{]} where \$i\_\{b\}\$ is the boundary index.\textbackslash{}
For test purposes, the phase velocity contribution or the entire
advection can be turned off by setting the corresponding parameters
texttt\{useStevensPhaseVel\} and texttt\{useStevensAdvection\} to
texttt\{.FALSE.\}.

\end{description}

end\{itemize\}
See citet\{stevens:90\} for details. With this boundary condition
specifying the exact net transport across the open boundary is simple,
so that balancing the flow with (S/R\textasciitilde{}OBCS\_BALANCE\_FLOW, see next
paragraph) is usually not necessary.

paragraph\{OBCS\_BALANCE\_FLOW:\} \textasciitilde{} \textbackslash{}
\%
When turned on (code\{ALLOW\_OBCS\_BALANCE\}
defined in code\{OBCS\_OPTIONS.h\} and code\{useOBCSbalance=.true.\} in
code\{data.obcs/OBCS\_PARM01\}), this routine balances the net flow
across the open boundaries. By default the net flow across the
boundaries is computed and all normal velocities on boundaries are
adjusted to obtain zero net inflow.

This behavior can be controlled with the runtime flags
code\{OBCS\_balanceFacN/S/E/W\}. The values of these flags determine
how the net inflow is redistributed as small correction velocities
between the individual sections. A value {\color{red}\bfseries{}{}`{}`}code\{-1\}'' balances an
individual boundary, values \$\textgreater{}0\$ determine the relative size of the
correction. For example, the values
begin\{tabbing\}
\begin{quote}

code\{OBCS\_balanceFacE\}code\{ = 1.,\} \textbackslash{}
code\{OBCS\_balanceFacW\}code\{ = -1.,\} \textbackslash{}
code\{OBCS\_balanceFacN\}code\{ = 2.,\} \textbackslash{}
code\{OBCS\_balanceFacS\}code\{ = 0.,\}
\end{quote}

end\{tabbing\}
make the model
begin\{itemize\}
item correct Western code\{OBWu\} by substracting a uniform velocity to
ensure zero net transport through the Western open boundary;
item correct Eastern and Northern normal flow, with the Northern
\begin{quote}

velocity correction two times larger than the Eastern correction, but
emph\{not\} the Southern normal flow, to ensure that the total inflow through
East, Northern, and Southern open boundary is balanced.
\end{quote}

end\{itemize\}

The old method of balancing the net flow for all sections individually
can be recovered by setting all flags to -1. Then the normal
velocities across each of the four boundaries are modified separately,
so that the net volume transport across emph\{each\} boundary is
zero. For example, for the western boundary at \$i=i\_\{b\}\$, the modified
velocity is:
{[}
u(y,z) - int\_\{mbox\{western boundary\}\}u,dy,dz approx OBNu(j,k) - sum\_\{j,k\}
OBNu(j,k) h\_\{w\}(i\_\{b\},j,k)Delta\{y\_G(i\_\{b\},j)\}Delta\{z(k)\}.
{]}
This also ensures a net total inflow of zero through all boundaries,
but this combination of flags is emph\{not\} useful if you want to
simulate, say, a sector of the Southern Ocean with a strong ACC
entering through the western and leaving through the eastern boundary,
because the value of {\color{red}\bfseries{}{}`{}`}code\{-1\}'' for these flags will make sure that
the strong inflow is removed. Clearly, gobal balancing with
code\{OBCS\_balanceFacE/W/N/S\} \$ge0\$ is the preferred method.

paragraph\{OBCS\_APPLY\_*:\} \textasciitilde{} \textbackslash{}
\textasciitilde{}

paragraph\{OBCS\_SPONGE:\} \textasciitilde{} \textbackslash{}
\%
The sponge layer code (turned on with code\{ALLOW\_OBCS\_SPONGE\} and
code\{useOBCSsponge\}) adds a relaxation term to the right-hand-side of
the momentum and tracer equations. The variables are relaxed towards
the boundary values with a relaxation time scale that increases
linearly with distance from the boundary
{[}
G\_\{chi\}\textasciicircum{}\{mbox\{(sponge)\}\} =
- frac\{chi - {[}( L - delta\{L\} ) chi\_\{BC\} + delta\{L\}chi{]}/L\}
\{{[}(L-delta\{L\})tau\_\{b\}+delta\{L\}tau\_\{i\}{]}/L\}
= - frac\{chi - {[}( 1 - l ) chi\_\{BC\} + lchi{]}\}
\{{[}(1-l)tau\_\{b\}+ltau\_\{i\}{]}\}
{]}
where \$chi\$ is the model variable (U/V/T/S) in the interior,
\$chi\_\{BC\}\$ the boundary value, \$L\$ the thickness of the sponge layer
(runtime parameter code\{spongeThickness\} in number of grid points),
\$delta\{L\}in{[}0,L{]}\$ (\$frac\{delta\{L\}\}\{L\}=lin{[}0,1{]}\$) the distance from the boundary (also in grid points), and
\$tau\_\{b\}\$ (runtime parameters code\{Urelaxobcsbound\} and
code\{Vrelaxobcsbound\}) and \$tau\_\{i\}\$ (runtime parameters
code\{Urelaxobcsinner\} and code\{Vrelaxobcsinner\}) the relaxation time
scales on the boundary and at the interior termination of the sponge
layer. The parameters code\{Urelaxobcsbound/inner\} set the relaxation
time scales for the Eastern and Western boundaries,
code\{Vrelaxobcsbound/inner\} for the Northern and Southern boundaries.

paragraph\{OB's with nonlinear free surface\} \textasciitilde{} \textbackslash{}
\%
\textasciitilde{}

\%----------------------------------------------------------------------

subsubsection\{Flow chart
label\{sec:pkg:obcs:flowchart\}\}

\{footnotesize
begin\{verbatim\}

C     !CALLING SEQUENCE:
c ...

end\{verbatim\}
\}

\%----------------------------------------------------------------------

subsubsection\{OBCS diagnostics
label\{sec:pkg:obcs:diagnostics\}\}

Diagnostics output is available via the diagnostics package
(see Section ref\{sec:pkg:diagnostics\}).
Available output fields are summarized in
Table ref\{tab:pkg:obcs:diagnostics\}.

begin\{table\}{[}!ht{]}
centering
label\{tab:pkg:obcs:diagnostics\}
\{footnotesize
begin\{verbatim\}
------------------------------------------------------
\begin{quote}

\textless{}-Name-\textgreater{}\textbar{}Levs\textbar{}grid\textbar{}\textless{}--  Units   --\textgreater{}\textbar{}\textless{}- Tile (max=80c)
\end{quote}


\bigskip\hrule\bigskip


end\{verbatim\}
\}
caption\{\textasciitilde{}\}
end\{table\}

\%----------------------------------------------------------------------

subsubsection\{Reference experiments\}
In the directory code\{verifcation\}, the following experiments use
code\{obcs\}:
begin\{itemize\}
item code\{exp4\}: box with 4 open boundaries, simulating flow over a
\begin{quote}

Gaussian bump based on citet\{adcroft:97\}, also tests
Stevens-boundary conditions;
\end{quote}
\begin{description}
\item[{item code\{dome\}: based on the project {\color{red}\bfseries{}{}`{}`}Dynamics of Overflow Mixing}] \leavevmode
and Entrainment''
(url\{\sphinxurl{http://www.rsmas.miami.edu/personal/tamay/DOME/dome.html}\}), uses
Orlanski-BCs;

\end{description}

item code\{internal\_wave\}: uses a heavily modified code\{S/R\textasciitilde{}OBCS\_CALC\}
item code\{seaice\_obcs\}: simple example who to use the sea-ice
\begin{quote}

related code, based on code\{lab\_sea\};
\end{quote}
\begin{description}
\item[{item code\{tutorial\_plume\_on\_slope\}: uses Orlanski-BCs, see also}] \leavevmode
section\textasciitilde{}ref\{sec:eg-gravityplume\}.

\end{description}

end\{itemize\}

\%----------------------------------------------------------------------

subsubsection\{References\}

subsubsection\{Experiments and tutorials that use obcs\}
label\{sec:pkg:obcs:experiments\}

begin\{itemize\}
item code\{tutorial\_plume\_on\_slope\} (section\textasciitilde{}ref\{sec:eg-gravityplume\})
end\{itemize\}

\%\%\% Local Variables:
\%\%\% mode: latex
\%\%\% TeX-master: \sphinxquotedblright{}../../manual\sphinxquotedblright{}
\%\%\% End:

newpage
input\{s\_phys\_pkgs/text/rbcs.tex\}

newpage
input\{s\_phys\_pkgs/text/ptracers.tex\}

\% Ocean Packages
newpage


\section{Ocean Packages}
\label{\detokenize{phys_pkgs/phys_pkgs:ocean-packages}}
input\{s\_phys\_pkgs/text/gmredi.tex\}

newpage
input\{s\_phys\_pkgs/text/kpp.tex\}

newpage
input\{s\_phys\_pkgs/text/ggl90.tex\}

newpage
input\{s\_phys\_pkgs/text/opps.tex\}

newpage
input\{s\_phys\_pkgs/text/kl10.tex\}

newpage
input\{s\_phys\_pkgs/text/bulk\_force.tex\}

newpage
input\{s\_phys\_pkgs/text/exf.tex\}

newpage
input\{s\_phys\_pkgs/text/cal.tex\}

newpage
section\{Atmosphere Packages\}
input\{s\_phys\_pkgs/text/aim.tex\}

newpage
input\{s\_phys\_pkgs/text/land.tex\}

newpage
input\{s\_phys\_pkgs/text/fizhi.tex\}

newpage
section\{Sea Ice Packages\}
input\{s\_phys\_pkgs/text/thsice.tex\}

newpage
input\{s\_phys\_pkgs/text/seaice.tex\}

newpage
input\{s\_phys\_pkgs/text/shelfice.tex\}

newpage
input\{s\_phys\_pkgs/text/streamice.tex\}

newpage
section\{Packages Related to Coupled Model\}
input\{s\_phys\_pkgs/text/aim\_compon\_interf.tex\}

newpage
input\{s\_phys\_pkgs/text/atm\_ocn\_coupler.tex\}

newpage
input\{s\_phys\_pkgs/text/component\_communications.tex\}

newpage
section\{Biogeochemistry Packages\}
input\{s\_phys\_pkgs/text/gchem.tex\}

newpage
input\{s\_phys\_pkgs/text/dic.tex\}


\chapter{Getting Started with MITgcm}
\label{\detokenize{getting_started/getting_started:getting-started-with-mitgcm}}\label{\detokenize{getting_started/getting_started:chap-getting-started}}\label{\detokenize{getting_started/getting_started::doc}}
This chapter is divided into two main parts. The first part, which is
covered in sections ref\{sec:whereToFindInfo\} through
ref\{sec:testing\}, contains information about how to run experiments
using MITgcm. The second part, covered in sections
ref\{sec:eg-baro\} through ref\{sec:eg-offline\}, contains a set of
step-by-step tutorials for running specific pre-configured atmospheric
and oceanic experiments.

We believe the best way to familiarize yourself with the
model is to run the case study examples provided with the base
version. Information on how to obtain, compile, and run the code is
found here as well as a brief description of the model structure
directory and the case study examples. Information is also provided
here on how to customize the code when you are ready to try implementing
the configuration you have in mind.  The code and algorithm
are described more fully in chapters ref\{chap:discretization\} and
ref\{chap:sarch\}.


\section{Where to find information}
\label{\detokenize{getting_started/getting_started:sec-wheretofindinfo}}\label{\detokenize{getting_started/getting_started:where-to-find-information}}
There is a web-archived support mailing list for the model that
you can email at \sphinxhref{mailto:texttt\{MITgcm-support@mitgcm.org}{texttt\{MITgcm-support@mitgcm.org}\} or browse at:
begin\{rawhtml\} \textless{}A href=http://mitgcm.org/mailman/listinfo/mitgcm-support/ target=\sphinxquotedblright{}idontexist\sphinxquotedblright{}\textgreater{} end\{rawhtml\}
begin\{verbatim\}
\sphinxurl{http://mitgcm.org/mailman/listinfo/mitgcm-support/}
\sphinxurl{http://mitgcm.org/pipermail/mitgcm-support/}
end\{verbatim\}
begin\{rawhtml\} \textless{}/A\textgreater{} end\{rawhtml\}

section\{Obtaining the code\}
label\{sec:obtainingCode\}
begin\{rawhtml\}
\textless{}!-- CMIREDIR:obtainingCode: --\textgreater{}
end\{rawhtml\}

MITgcm can be downloaded from our system by following
the instructions below. As a courtesy we ask that you send e-mail to us at
begin\{rawhtml\} \textless{}A href=mailto:\sphinxhref{mailto:MITgcm-support@mitgcm.org}{MITgcm-support@mitgcm.org}\textgreater{} end\{rawhtml\}
\sphinxhref{mailto:MITgcm-support@mitgcm.org}{MITgcm-support@mitgcm.org}
begin\{rawhtml\} \textless{}/A\textgreater{} end\{rawhtml\}
to enable us to keep track of who's using the model and in what application.
You can download the model two ways:

begin\{enumerate\}
item Using CVS software. CVS is a freely available source code management
tool. To use CVS you need to have the software installed. Many systems
come with CVS pre-installed, otherwise good places to look for
the software for a particular platform are
begin\{rawhtml\} \textless{}A href=http://www.cvshome.org/ target=\sphinxquotedblright{}idontexist\sphinxquotedblright{}\textgreater{} end\{rawhtml\}
cvshome.org
begin\{rawhtml\} \textless{}/A\textgreater{} end\{rawhtml\}
and
begin\{rawhtml\} \textless{}A href=http://www.wincvs.org/ target=\sphinxquotedblright{}idontexist\sphinxquotedblright{}\textgreater{} end\{rawhtml\}
wincvs.org
begin\{rawhtml\} \textless{}/A\textgreater{} end\{rawhtml\}
.

item Using a tar file. This method is simple and does not
require any special software. However, this method does not
provide easy support for maintenance updates.

end\{enumerate\}

subsection\{Method 1 - Checkout from CVS\}
label\{sec:cvs\_checkout\}

If CVS is available on your system, we strongly encourage you to use it. CVS
provides an efficient and elegant way of organizing your code and keeping
track of your changes. If CVS is not available on your machine, you can also
download a tar file.

Before you can use CVS, the following environment variable(s) should
be set within your shell.  For a csh or tcsh shell, put the following
begin\{verbatim\}
\% setenv CVSROOT :pserver:cvsanon@mitgcm.org:/u/gcmpack
end\{verbatim\}
in your texttt\{.cshrc\} or texttt\{.tcshrc\} file.  For bash or sh
shells, put:
begin\{verbatim\}
\% export CVSROOT=':pserver:cvsanon@mitgcm.org:/u/gcmpack'
end\{verbatim\}
in your texttt\{.profile\} or texttt\{.bashrc\} file.

To get MITgcm through CVS, first register with the MITgcm CVS server
using command:
begin\{verbatim\}
\% cvs login ( CVS password: cvsanon )
end\{verbatim\}
You only need to do a {\color{red}\bfseries{}{}`{}`}cvs login'' once.

To obtain the latest sources type:
begin\{verbatim\}
\% cvs co -P MITgcm
end\{verbatim\}
or to get a specific release type:
begin\{verbatim\}
\% cvs co -P -r checkpoint52i\_post MITgcm
end\{verbatim\}
The CVS command {\color{red}\bfseries{}{}`{}`}texttt\{cvs co\}'' is the abreviation of the full-name
{\color{red}\bfseries{}{}`{}`}texttt\{cvs checkout\}'' command and using the option {\color{red}\bfseries{}{}`{}`}-P'' (texttt\{cvs co -P\})
will prevent to download unnecessary empty directories.

The MITgcm web site contains further directions concerning the source
code and CVS.  It also contains a web interface to our CVS archive so
that one may easily view the state of files, revisions, and other
development milestones:
begin\{rawhtml\} \textless{}A href=\sphinxquotedblright{}\sphinxurl{http://mitgcm.org/viewvc/MITgcm/MITgcm/}\sphinxquotedblright{} target=\sphinxquotedblright{}idontexist\sphinxquotedblright{}\textgreater{} end\{rawhtml\}
begin\{verbatim\}
\sphinxurl{http://mitgcm.org/viewvc/MITgcm/MITgcm/}
end\{verbatim\}
begin\{rawhtml\} \textless{}/A\textgreater{} end\{rawhtml\}

As a convenience, the MITgcm CVS server contains aliases which are
named subsets of the codebase.  These aliases can be especially
helpful when used over slow internet connections or on machines with
restricted storage space.  Table ref\{tab:cvsModules\} contains a list
of CVS aliases
begin\{table\}{[}htb{]}
\begin{quote}

centering
begin\{tabular\}{[}htb{]}\{{\color{red}\bfseries{}\textbar{}lp\{3.25in\}\textbar{}}\}hline
\begin{quote}

textbf\{Alias Name\}    \&  textbf\{Information (directories) Contained\}  \textbackslash{}hline
texttt\{MITgcm\_code\}  \&  Only the source code -- none of the verification examples.  \textbackslash{}
texttt\{MITgcm\_verif\_basic\}
\&  Source code plus a small set of the verification examples
(texttt\{global\_ocean.90x40x15\}, texttt\{aim.5l\_cs\}, texttt\{hs94.128x64x5\},
texttt\{front\_relax\}, and texttt\{plume\_on\_slope\}).  \textbackslash{}
texttt\{MITgcm\_verif\_atmos\}  \&  Source code plus all of the atmospheric examples.  \textbackslash{}
texttt\{MITgcm\_verif\_ocean\}  \&  Source code plus all of the oceanic examples.  \textbackslash{}
texttt\{MITgcm\_verif\_all\}    \&  Source code plus all of the
verification examples. \textbackslash{}hline
\end{quote}

end\{tabular\}
caption\{MITgcm CVS Modules\}
label\{tab:cvsModules\}
\end{quote}

end\{table\}

The checkout process creates a directory called texttt\{MITgcm\}. If
the directory texttt\{MITgcm\} exists this command updates your code
based on the repository. Each directory in the source tree contains a
directory texttt\{CVS\}. This information is required by CVS to keep
track of your file versions with respect to the repository. Don't edit
the files in texttt\{CVS\}!  You can also use CVS to download code
updates.  More extensive information on using CVS for maintaining
MITgcm code can be found
begin\{rawhtml\} \textless{}A href=\sphinxquotedblright{}\sphinxurl{http://mitgcm.org/public/using\_cvs.html}\sphinxquotedblright{} target=\sphinxquotedblright{}idontexist\sphinxquotedblright{}\textgreater{} end\{rawhtml\}
here
begin\{rawhtml\} \textless{}/A\textgreater{} end\{rawhtml\}.
It is important to note that the CVS aliases in Table
ref\{tab:cvsModules\} cannot be used in conjunction with the CVS
texttt\{-d DIRNAME\} option.  However, the texttt\{MITgcm\} directories
they create can be changed to a different name following the check-out:
begin\{verbatim\}
\begin{quote}

\%  cvs co -P MITgcm\_verif\_basic
\%  mv MITgcm MITgcm\_verif\_basic
\end{quote}

end\{verbatim\}

Note that it is possible to checkout code without {\color{red}\bfseries{}{}`{}`}cvs login'' and without
setting any shell environment variables by specifying the pserver name and
password in one line, for example:
begin\{verbatim\}
\begin{quote}

\%  cvs -d :pserver:cvsanon:cvsanon@mitgcm.org:/u/gcmpack co -P MITgcm
\end{quote}

end\{verbatim\}

subsubsection\{Upgrading from an earlier version\}

If you already have an earlier version of the code you can {\color{red}\bfseries{}{}`{}`}upgrade''
your copy instead of downloading the entire repository again. First,
{\color{red}\bfseries{}{}`{}`}cd'' (change directory) to the top of your working copy:
begin\{verbatim\}
\% cd MITgcm
end\{verbatim\}
and then issue the cvs update command such as:
begin\{verbatim\}
\% cvs -q update -d -P -r checkpoint52i\_post
end\{verbatim\}
This will update the {\color{red}\bfseries{}{}`{}`}tag'' to {\color{red}\bfseries{}{}`{}`}checkpoint52i\_post'`, add any new
directories (-d) and remove any empty directories (-P). The -q option
means be quiet which will reduce the number of messages you'll see in
the terminal. If you have modified the code prior to upgrading, CVS
will try to merge your changes with the upgrades. If there is a
conflict between your modifications and the upgrade, it will report
that file with a {\color{red}\bfseries{}{}`{}`}C'' in front, e.g.:
begin\{verbatim\}
C model/src/ini\_parms.F
end\{verbatim\}
If the list of conflicts scrolled off the screen, you can re-issue the
cvs update command and it will report the conflicts. Conflicts are
indicated in the code by the delimites {\color{red}\bfseries{}{}`{}`}\$\textless{}\textless{}\textless{}\textless{}\textless{}\textless{}\textless{}\$'`, {\color{red}\bfseries{}{}`{}`}======='' and
{\color{red}\bfseries{}{}`{}`}\$\textgreater{}\textgreater{}\textgreater{}\textgreater{}\textgreater{}\textgreater{}\textgreater{}\$'`. For example,
\{small
begin\{verbatim\}
\textless{}\textless{}\textless{}\textless{}\textless{}\textless{}\textless{} ini\_parms.F
\begin{quote}

\& bottomDragLinear,myOwnBottomDragCoefficient,
\end{quote}

end\{verbatim\}
\}
means that you added {\color{red}\bfseries{}{}`{}`}myOwnBottomDragCoefficient'' to a namelist at
the same time and place that we added {\color{red}\bfseries{}{}`{}`}bottomDragQuadratic'`. You
need to resolve this conflict and in this case the line should be
changed to:
\{small
begin\{verbatim\}
\begin{quote}

\& bottomDragLinear,bottomDragQuadratic,myOwnBottomDragCoefficient,
\end{quote}

end\{verbatim\}
\}
and the lines with the delimiters (\$\textless{}\textless{}\textless{}\textless{}\textless{}\textless{}\$,======,\$\textgreater{}\textgreater{}\textgreater{}\textgreater{}\textgreater{}\textgreater{}\$) be deleted.
Unless you are making modifications which exactly parallel
developments we make, these types of conflicts should be rare.

paragraph*\{Upgrading to the current pre-release version\}

We don't make a {\color{red}\bfseries{}{}`{}`}release'' for every little patch and bug fix in
order to keep the frequency of upgrades to a minimum. However, if you
have run into a problem for which {\color{red}\bfseries{}{}`{}`}we have already fixed in the
latest code'' and we haven't made a {\color{red}\bfseries{}{}`{}`}tag'' or {\color{red}\bfseries{}{}`{}`}release'' since that
patch then you'll need to get the latest code:
begin\{verbatim\}
\% cvs -q update -d -P -A
end\{verbatim\}
Unlike, the {\color{red}\bfseries{}{}`{}`}check-out'' and {\color{red}\bfseries{}{}`{}`}update'' procedures above, there is no
{\color{red}\bfseries{}{}`{}`}tag'' or release name. The -A tells CVS to upgrade to the
very latest version. As a rule, we don't recommend this since you
might upgrade while we are in the processes of checking in the code so
that you may only have part of a patch. Using this method of updating
also means we can't tell what version of the code you are working
with. So please be sure you understand what you're doing.

subsection\{Method 2 - Tar file download\}
label\{sec:conventionalDownload\}

If you do not have CVS on your system, you can download the model as a
tar file from the web site at:
begin\{rawhtml\} \textless{}A href=http://mitgcm.org/download/ target=\sphinxquotedblright{}idontexist\sphinxquotedblright{}\textgreater{} end\{rawhtml\}
begin\{verbatim\}
\sphinxurl{http://mitgcm.org/download/}
end\{verbatim\}
begin\{rawhtml\} \textless{}/A\textgreater{} end\{rawhtml\}
The tar file still contains CVS information which we urge you not to
delete; even if you do not use CVS yourself the information can help
us if you should need to send us your copy of the code.  If a recent
tar file does not exist, then please contact the developers through
the
begin\{rawhtml\} \textless{}A href=\sphinxquotedblright{}\sphinxurl{mailto:MITgcm-support@mitgcm.org}\sphinxquotedblleft{}\textgreater{} end\{rawhtml\}
\sphinxhref{mailto:MITgcm-support@mitgcm.org}{MITgcm-support@mitgcm.org}
begin\{rawhtml\} \textless{}/A\textgreater{} end\{rawhtml\}
mailing list.

section\{Model and directory structure\}
begin\{rawhtml\}
\textless{}!-- CMIREDIR:directory\_structure: --\textgreater{}
end\{rawhtml\}

The {\color{red}\bfseries{}{}`{}`}numerical'' model is contained within a execution environment
support wrapper. This wrapper is designed to provide a general
framework for grid-point models. MITgcmUV is a specific numerical
model that uses the framework. Under this structure the model is split
into execution environment support code and conventional numerical
model code. The execution environment support code is held under the
texttt\{eesupp\} directory. The grid point model code is held under the
texttt\{model\} directory. Code execution actually starts in the
texttt\{eesupp\} routines and not in the texttt\{model\} routines. For
this reason the top-level texttt\{MAIN.F\} is in the
texttt\{eesupp/src\} directory. In general, end-users should not need
to worry about this level. The top-level routine for the numerical
part of the code is in texttt\{model/src/THE\_MODEL\_MAIN.F\}. Here is
a brief description of the directory structure of the model under the
root tree (a detailed description is given in section 3: Code
structure).

begin\{itemize\}

item texttt\{doc\}: contains brief documentation notes.
\begin{description}
\item[{item texttt\{eesupp\}: contains the execution environment source code.}] \leavevmode
Also subdivided into two subdirectories texttt\{inc\} and
texttt\{src\}.

\item[{item texttt\{model\}: this directory contains the main source code.}] \leavevmode
Also subdivided into two subdirectories texttt\{inc\} and
texttt\{src\}.

\item[{item texttt\{pkg\}: contains the source code for the packages. Each}] \leavevmode
package corresponds to a subdirectory. For example, texttt\{gmredi\}
contains the code related to the Gent-McWilliams/Redi scheme,
texttt\{aim\} the code relative to the atmospheric intermediate
physics. The packages are described in detail in chapter ref\{chap:packagesI\}.

\item[{item texttt\{tools\}: this directory contains various useful tools.}] \leavevmode
For example, texttt\{genmake2\} is a script written in csh (C-shell)
that should be used to generate your makefile. The directory
texttt\{adjoint\} contains the makefile specific to the Tangent
linear and Adjoint Compiler (TAMC) that generates the adjoint code.
The latter is described in detail in part ref\{chap.ecco\}.
This directory also contains the subdirectory build\_options, which
contains the {\color{red}\bfseries{}{}`}optfiles' with the compiler options for the different
compilers and machines that can run MITgcm.

\item[{item texttt\{utils\}: this directory contains various utilities. The}] \leavevmode
subdirectory texttt\{knudsen2\} contains code and a makefile that
compute coefficients of the polynomial approximation to the knudsen
formula for an ocean nonlinear equation of state. The
texttt\{matlab\} subdirectory contains matlab scripts for reading
model output directly into matlab. texttt\{scripts\} contains C-shell
post-processing scripts for joining processor-based and tiled-based
model output. The subdirectory exch2 contains the code needed for
the exch2 package to work with different combinations of domain
decompositions.

\item[{item texttt\{verification\}: this directory contains the model}] \leavevmode
examples. See section ref\{sec:modelExamples\}.

\end{description}

item texttt\{jobs\}: contains sample job scripts for running MITgcm.

item texttt\{lsopt\}: Line search code used for optimization.

item texttt\{optim\}: Interface between MITgcm and line search code.

end\{itemize\}

section{[}Building MITgcm{]}\{Building the code\}
label\{sec:buildingCode\}
begin\{rawhtml\}
\textless{}!-- CMIREDIR:buildingCode: --\textgreater{}
end\{rawhtml\}

To compile the code, we use the texttt\{make\} program. This uses a
file (texttt\{Makefile\}) that allows us to pre-process source files,
specify compiler and optimization options and also figures out any
file dependencies. We supply a script (texttt\{genmake2\}), described
in section ref\{sec:genmake\}, that automatically creates the
texttt\{Makefile\} for you. You then need to build the dependencies and
compile the code.

As an example, assume that you want to build and run experiment
texttt\{verification/exp2\}. The are multiple ways and places to
actually do this but here let's build the code in
texttt\{verification/exp2/build\}:
begin\{verbatim\}
\% cd verification/exp2/build
end\{verbatim\}
First, build the texttt\{Makefile\}:
begin\{verbatim\}
\% ../../../tools/genmake2 -mods=../code
end\{verbatim\}
The command line option tells texttt\{genmake\} to override model source
code with any files in the directory texttt\{../code/\}.

On many systems, the texttt\{genmake2\} program will be able to
automatically recognize the hardware, find compilers and other tools
within the user's path ({\color{red}\bfseries{}{}`{}`}texttt\{echo \$PATH\}'`), and then choose an
appropriate set of options from the files ({\color{red}\bfseries{}{}`{}`}optfiles'`) contained in
the texttt\{tools/build\_options\} directory.  Under some
circumstances, a user may have to create a new {\color{red}\bfseries{}{}`{}`}optfile'' in order to
specify the exact combination of compiler, compiler flags, libraries,
and other options necessary to build a particular configuration of
MITgcm.  In such cases, it is generally helpful to read the existing
{\color{red}\bfseries{}{}`{}`}optfiles'' and mimic their syntax.

Through the MITgcm-support list, the MITgcm developers are willing to
provide help writing or modifing {\color{red}\bfseries{}{}`{}`}optfiles'`.  And we encourage users
to post new {\color{red}\bfseries{}{}`{}`}optfiles'' (particularly ones for new machines or
architectures) to the
begin\{rawhtml\} \textless{}A href=\sphinxquotedblright{}\sphinxurl{mailto:MITgcm-support@mitgcm.org}\sphinxquotedblleft{}\textgreater{} end\{rawhtml\}
\sphinxhref{mailto:MITgcm-support@mitgcm.org}{MITgcm-support@mitgcm.org}
begin\{rawhtml\} \textless{}/A\textgreater{} end\{rawhtml\}
list.

To specify an optfile to texttt\{genmake2\}, the syntax is:
begin\{verbatim\}
\% ../../../tools/genmake2 -mods=../code -of /path/to/optfile
end\{verbatim\}

Once a texttt\{Makefile\} has been generated, we create the
dependencies with the command:
begin\{verbatim\}
\% make depend
end\{verbatim\}
This modifies the texttt\{Makefile\} by attaching a (usually, long)
list of files upon which other files depend. The purpose of this is to
reduce re-compilation if and when you start to modify the code. The
\{tt make depend\} command also creates links from the model source to
this directory.  It is important to note that the \{tt make depend\}
stage will occasionally produce warnings or errors since the
dependency parsing tool is unable to find all of the necessary header
files (textit\{eg.\}  texttt\{netcdf.inc\}).  In these circumstances, it
is usually OK to ignore the warnings/errors and proceed to the next
step.

Next one can compile the code using:
begin\{verbatim\}
\% make
end\{verbatim\}
The \{tt make\} command creates an executable called texttt\{mitgcmuv\}.
Additional make {\color{red}\bfseries{}{}`{}`}targets'' are defined within the makefile to aid in
the production of adjoint and other versions of MITgcm.  On SMP
(shared multi-processor) systems, the build process can often be sped
up appreciably using the command:
begin\{verbatim\}
\% make -j 2
end\{verbatim\}
where the {\color{red}\bfseries{}{}`{}`}2'' can be replaced with a number that corresponds to the
number of CPUs available.

Now you are ready to run the model. General instructions for doing so are
given in section ref\{sec:runModel\}. Here, we can run the model by
first creating links to all the input files:
begin\{verbatim\}
ln -s ../input/* .
end\{verbatim\}
and then calling the executable with:
begin\{verbatim\}
./mitgcmuv \textgreater{} output.txt
end\{verbatim\}
where we are re-directing the stream of text output to the file
texttt\{output.txt\}.

subsection\{Building/compiling the code elsewhere\}

In the example above (section ref\{sec:buildingCode\}) we built the
executable in the \{em input\} directory of the experiment for
convenience. You can also configure and compile the code in other
locations, for example on a scratch disk with out having to copy the
entire source tree. The only requirement to do so is you have \{tt
\begin{quote}

genmake2\} in your path or you know the absolute path to \{tt
genmake2\}.
\end{quote}

The following sections outline some possible methods of organizing
your source and data.

subsubsection\{Building from the \{em ../code directory\}\}

This is just as simple as building in the \{em input/\} directory:
begin\{verbatim\}
\% cd verification/exp2/code
\% ../../../tools/genmake2
\% make depend
\% make
end\{verbatim\}
However, to run the model the executable (\{em mitgcmuv\}) and input
files must be in the same place. If you only have one calculation to make:
begin\{verbatim\}
\% cd ../input
\% cp ../code/mitgcmuv ./
\% ./mitgcmuv \textgreater{} output.txt
end\{verbatim\}
or if you will be making multiple runs with the same executable:
begin\{verbatim\}
\% cd ../
\% cp -r input run1
\% cp code/mitgcmuv run1
\% cd run1
\% ./mitgcmuv \textgreater{} output.txt
end\{verbatim\}

subsubsection\{Building from a new directory\}

Since the \{em input\} directory contains input files it is often more
useful to keep \{em input\} pristine and build in a new directory
within \{em verification/exp2/\}:
begin\{verbatim\}
\% cd verification/exp2
\% mkdir build
\% cd build
\% ../../../tools/genmake2 -mods=../code
\% make depend
\% make
end\{verbatim\}
This builds the code exactly as before but this time you need to copy
either the executable or the input files or both in order to run the
model. For example,
begin\{verbatim\}
\% cp ../input/* ./
\% ./mitgcmuv \textgreater{} output.txt
end\{verbatim\}
or if you tend to make multiple runs with the same executable then
running in a new directory each time might be more appropriate:
begin\{verbatim\}
\% cd ../
\% mkdir run1
\% cp build/mitgcmuv run1/
\% cp input/* run1/
\% cd run1
\% ./mitgcmuv \textgreater{} output.txt
end\{verbatim\}

subsubsection\{Building on a scratch disk\}

Model object files and output data can use up large amounts of disk
space so it is often the case that you will be operating on a large
scratch disk. Assuming the model source is in \{em \textasciitilde{}/MITgcm\} then the
following commands will build the model in \{em /scratch/exp2-run1\}:
begin\{verbatim\}
\% cd /scratch/exp2-run1
\% \textasciitilde{}/MITgcm/tools/genmake2 -rootdir=\textasciitilde{}/MITgcm 
\begin{quote}

-mods=\textasciitilde{}/MITgcm/verification/exp2/code
\end{quote}

\% make depend
\% make
end\{verbatim\}
To run the model here, you'll need the input files:
begin\{verbatim\}
\% cp \textasciitilde{}/MITgcm/verification/exp2/input/* ./
\% ./mitgcmuv \textgreater{} output.txt
end\{verbatim\}

As before, you could build in one directory and make multiple runs of
the one experiment:
begin\{verbatim\}
\% cd /scratch/exp2
\% mkdir build
\% cd build
\% \textasciitilde{}/MITgcm/tools/genmake2 -rootdir=\textasciitilde{}/MITgcm 
\begin{quote}

-mods=\textasciitilde{}/MITgcm/verification/exp2/code
\end{quote}

\% make depend
\% make
\% cd ../
\% cp -r \textasciitilde{}/MITgcm/verification/exp2/input run2
\% cd run2
\% ./mitgcmuv \textgreater{} output.txt
end\{verbatim\}

subsection\{Using texttt\{genmake2\}\}
label\{sec:genmake\}

To compile the code, first use the program texttt\{genmake2\} (located
in the texttt\{tools\} directory) to generate a Makefile.
texttt\{genmake2\} is a shell script written to work with all
{\color{red}\bfseries{}{}`{}`}sh'`--compatible shells including bash v1, bash v2, and Bourne.
\%Internally, texttt\{genmake2\} determines the locations of needed
\%files, the compiler, compiler options, libraries, and Unix tools.  It
\%relies upon a number of {\color{red}\bfseries{}{}`{}`}optfiles'' located in the
\%texttt\{tools/build\_options\} directory.
texttt\{genmake2\} parses information from the following sources:
begin\{description\}
item{[}-{]} a \{em gemake\_local\} file if one is found in the current
\begin{quote}

directory
\end{quote}

item{[}-{]} command-line options
item{[}-{]} an \sphinxquotedblleft{}options file\sphinxquotedblright{} as specified by the command-line option
\begin{quote}

texttt\{--optfile=/PATH/FILENAME\}
\end{quote}
\begin{description}
\item[{item{[}-{]} a \{em packages.conf\} file (if one is found) with the}] \leavevmode
specific list of packages to compile. The search path for
file \{em packages.conf\} is, first, the current directory and
then each of the \sphinxquotedblleft{}MODS\sphinxquotedblright{} directories in the given order (see below).

\end{description}

end\{description\}

subsubsection\{Optfiles in texttt\{tools/build\_options\} directory:\}

The purpose of the optfiles is to provide all the compilation options
for particular {\color{red}\bfseries{}{}`{}`}platforms'' (where {\color{red}\bfseries{}{}`{}`}platform'' roughly means the
combination of the hardware and the compiler) and code configurations.
Given the combinations of possible compilers and library dependencies
(\{it eg.\}  MPI and NetCDF) there may be numerous optfiles available
for a single machine.  The naming scheme for the majority of the
optfiles shipped with the code is
begin\{center\}
\begin{quote}

\{bf OS\_HARDWARE\_COMPILER \}
\end{quote}

end\{center\}
where
begin\{description\}
item{[}OS{]} is the name of the operating system (generally the
\begin{quote}

lower-case output of the \{tt `uname'\} command)
\end{quote}
\begin{description}
\item[{item{[}HARDWARE{]} is a string that describes the CPU type and}] \leavevmode
corresponds to output from the  \{tt `uname -m'\} command:
begin\{description\}
item{[}ia32{]} is for {\color{red}\bfseries{}{}`{}`}x86'' machines such as i386, i486, i586, i686,
\begin{quote}

and athlon
\end{quote}

item{[}ia64{]} is for Intel IA64 systems (eg. Itanium, Itanium2)
item{[}amd64{]} is AMD x86\_64 systems
item{[}ppc{]} is for Mac PowerPC systems
end\{description\}

\item[{item{[}COMPILER{]} is the compiler name (generally, the name of the}] \leavevmode
FORTRAN executable)

\end{description}

end\{description\}

In many cases, the default optfiles are sufficient and will result in
usable Makefiles.  However, for some machines or code configurations,
new {\color{red}\bfseries{}{}`{}`}optfiles'' must be written. To create a new optfile, it is
generally best to start with one of the defaults and modify it to suit
your needs.  Like texttt\{genmake2\}, the optfiles are all written
using a simple {\color{red}\bfseries{}{}`{}`}sh'`--compatible syntax.  While nearly all variables
used within texttt\{genmake2\} may be specified in the optfiles, the
critical ones that should be defined are:

begin\{description\}
item{[}FC{]} the FORTRAN compiler (executable) to use
item{[}DEFINES{]} the command-line DEFINE options passed to the compiler
item{[}CPP{]} the C pre-processor to use
item{[}NOOPTFLAGS{]} options flags for special files that should not be
\begin{quote}

optimized
\end{quote}

end\{description\}

For example, the optfile for a typical Red Hat Linux machine ({\color{red}\bfseries{}{}`{}`}ia32''
architecture) using the GCC (g77) compiler is
begin\{verbatim\}
FC=g77
DEFINES='-D\_BYTESWAPIO -DWORDLENGTH=4'
CPP='cpp  -traditional -P'
NOOPTFLAGS='-O0'
\#  For IEEE, use the \sphinxquotedblleft{}-ffloat-store\sphinxquotedblright{} option
if test \sphinxquotedblleft{}x\$IEEE\sphinxquotedblright{} = x ; then
\begin{quote}

FFLAGS='-Wimplicit -Wunused -Wuninitialized'
FOPTIM='-O3 -malign-double -funroll-loops'
\end{quote}
\begin{description}
\item[{else}] \leavevmode
FFLAGS='-Wimplicit -Wunused -ffloat-store'
FOPTIM='-O0 -malign-double'

\end{description}

fi
end\{verbatim\}

If you write an optfile for an unrepresented machine or compiler, you
are strongly encouraged to submit the optfile to the MITgcm project
for inclusion.  Please send the file to the
begin\{rawhtml\} \textless{}A href=\sphinxquotedblright{}mail-to:MITgcm-support@mitgcm.org\sphinxquotedblright{}\textgreater{} end\{rawhtml\}
begin\{center\}
\begin{quote}

\sphinxhref{mailto:MITgcm-support@mitgcm.org}{MITgcm-support@mitgcm.org}
\end{quote}

end\{center\}
begin\{rawhtml\} \textless{}/A\textgreater{} end\{rawhtml\}
mailing list.

subsubsection\{Command-line options:\}

In addition to the optfiles, texttt\{genmake2\} supports a number of
helpful command-line options.  A complete list of these options can be
obtained from:
begin\{verbatim\}
\% genmake2 -h
end\{verbatim\}

The most important command-line options are:
begin\{description\}
\begin{description}
\item[{item{[}texttt\{--optfile=/PATH/FILENAME\}{]} specifies the optfile that}] \leavevmode
should be used for a particular build.

If no \sphinxquotedblleft{}optfile\sphinxquotedblright{} is specified (either through the command line or the
MITGCM\_OPTFILE environment variable), genmake2 will try to make a
reasonable guess from the list provided in \{em
\begin{quote}

tools/build\_options\}.  The method used for making this guess is
\end{quote}

to first determine the combination of operating system and hardware
(eg. \sphinxquotedblleft{}linux\_ia32\sphinxquotedblright{}) and then find a working FORTRAN compiler within
the user's path.  When these three items have been identified,
genmake2 will try to find an optfile that has a matching name.

\item[{item{[}texttt\{--mods='DIR1 DIR2 DIR3 ...'\}{]} specifies a list of}] \leavevmode
directories containing {\color{red}\bfseries{}{}`{}`}modifications'`.  These directories contain
files with names that may (or may not) exist in the main MITgcm
source tree but will be overridden by any identically-named sources
within the {\color{red}\bfseries{}{}`{}`}MODS'' directories.

The order of precedence for this \sphinxquotedblleft{}name-hiding\sphinxquotedblright{} is as follows:
begin\{itemize\}
item {\color{red}\bfseries{}{}`{}`}MODS'' directories (in the order given)
item Packages either explicitly specified or provided by default
\begin{quote}

(in the order given)
\end{quote}
\begin{description}
\item[{item Packages included due to package dependencies (in the order}] \leavevmode
that that package dependencies are parsed)

\item[{item The \sphinxquotedblleft{}standard dirs\sphinxquotedblright{} (which may have been specified by the}] \leavevmode
{\color{red}\bfseries{}{}`{}`}-standarddirs'' option)

\end{description}

end\{itemize\}

\item[{item{[}texttt\{--pgroups=/PATH/FILENAME\}{]} specifies the file}] \leavevmode
where package groups are defined. If not set, the package-groups
definition will be read from \{em pkg/pkg\_groups\}.
It also contains the default list of packages (defined
as the group {\color{red}\bfseries{}{}`{}`}\{it default\_pkg\_list\}'' which is used
when no specific package list (\{em packages.conf\})
is found in current directory or in any \sphinxquotedblleft{}MODS\sphinxquotedblright{} directory.

\item[{item{[}texttt\{--pdepend=/PATH/FILENAME\}{]} specifies the dependency file}] \leavevmode
used for packages.

If not specified, the default dependency file \{em pkg/pkg\_depend\}
is used.  The syntax for this file is parsed on a line-by-line basis
where each line containes either a comment (\sphinxquotedblleft{}\#\sphinxquotedblright{}) or a simple
\sphinxquotedblleft{}PKGNAME1 (+\textbar{}-)PKGNAME2\sphinxquotedblright{} pairwise rule where the \sphinxquotedblleft{}+\sphinxquotedblright{} or \sphinxquotedblleft{}-\sphinxquotedblright{} symbol
specifies a \sphinxquotedblleft{}must be used with\sphinxquotedblright{} or a \sphinxquotedblleft{}must not be used with\sphinxquotedblright{}
relationship, respectively.  If no rule is specified, then it is
assumed that the two packages are compatible and will function
either with or without each other.

\item[{item{[}texttt\{--adof=/path/to/file\}{]} specifies the \sphinxquotedblleft{}adjoint\sphinxquotedblright{} or}] \leavevmode
automatic differentiation options file to be used.  The file is
analogous to the {\color{red}\bfseries{}{}`{}`}optfile'' defined above but it specifies
information for the AD build process.
\begin{description}
\item[{The default file is located in \{em}] \leavevmode
tools/adjoint\_options/adjoint\_default\} and it defines the \sphinxquotedblleft{}TAF\sphinxquotedblright{}

\end{description}

and \sphinxquotedblleft{}TAMC\sphinxquotedblright{} compilers.  An alternate version is also available at
\{em tools/adjoint\_options/adjoint\_staf\} that selects the newer
\sphinxquotedblleft{}STAF\sphinxquotedblright{} compiler.  As with any compilers, it is helpful to have their
directories listed in your \{tt \$PATH\} environment variable.

\item[{item{[}texttt\{--mpi\}{]} This option enables certain MPI features (using}] \leavevmode
CPP texttt\{\#define\}s) within the code and is necessary for MPI
builds (see Section ref\{sec:mpi-build\}).

\item[{item{[}texttt\{--make=/path/to/gmake\}{]} Due to the poor handling of}] \leavevmode
soft-links and other bugs common with the texttt\{make\} versions
provided by commercial Unix vendors, GNU texttt\{make\} (sometimes
called texttt\{gmake\}) should be preferred.  This option provides a
means for specifying the make executable to be used.

\item[{item{[}texttt\{--bash=/path/to/sh\}{]} On some (usually older UNIX)}] \leavevmode
machines, the {\color{red}\bfseries{}{}`{}`}bash'' shell is unavailable.  To run on these
systems, texttt\{genmake2\} can be invoked using an {\color{red}\bfseries{}{}`{}`}sh'' (that is,
a Bourne, POSIX, or compatible) shell.  The syntax in these
circumstances is:
begin\{center\}
\begin{quote}

texttt\{\%  /bin/sh genmake2 -bash=/bin/sh {[}...options...{]}\}
\end{quote}

end\{center\}
where texttt\{/bin/sh\} can be replaced with the full path and name
of the desired shell.

\end{description}

end\{description\}

subsection\{Building with MPI\}
label\{sec:mpi-build\}

Building MITgcm to use MPI libraries can be complicated due to the
variety of different MPI implementations available, their dependencies
or interactions with different compilers, and their often ad-hoc
locations within file systems.  For these reasons, its generally a
good idea to start by finding and reading the documentation for your
machine(s) and, if necessary, seeking help from your local systems
administrator.

The steps for building MITgcm with MPI support are:
begin\{enumerate\}
\begin{description}
\item[{item Determine the locations of your MPI-enabled compiler and/or MPI}] \leavevmode
libraries and put them into an options file as described in Section
ref\{sec:genmake\}.  One can start with one of the examples in:
begin\{rawhtml\} \textless{}A
\begin{quote}

href=\sphinxquotedblright{}\sphinxurl{http://mitgcm.org/viewvc/MITgcm/MITgcm/tools/build\_options/}\sphinxquotedblleft{}\textgreater{}
\end{quote}

end\{rawhtml\}
begin\{center\}
\begin{quote}

texttt\{MITgcm/tools/build\_options/\}
\end{quote}

end\{center\}
begin\{rawhtml\} \textless{}/A\textgreater{} end\{rawhtml\}
such as texttt\{linux\_ia32\_g77+mpi\_cg01\} or
texttt\{linux\_ia64\_efc+mpi\} and then edit it to suit the machine at
hand.  You may need help from your user guide or local systems
administrator to determine the exact location of the MPI libraries.
If libraries are not installed, MPI implementations and related
tools are available including:
begin\{itemize\}
item begin\{rawhtml\} \textless{}A
\begin{quote}
\begin{quote}

href=\sphinxquotedblright{}\sphinxurl{http://www-unix.mcs.anl.gov/mpi/mpich/}\sphinxquotedblleft{}\textgreater{}
\end{quote}

end\{rawhtml\}
MPICH
begin\{rawhtml\} \textless{}/A\textgreater{} end\{rawhtml\}
\end{quote}
\begin{description}
\item[{item begin\{rawhtml\} \textless{}A}] \leavevmode\begin{quote}

href=\sphinxquotedblright{}\sphinxurl{http://www.lam-mpi.org/}\sphinxquotedblleft{}\textgreater{}
\end{quote}

end\{rawhtml\}
LAM/MPI
begin\{rawhtml\} \textless{}/A\textgreater{} end\{rawhtml\}

\item[{item begin\{rawhtml\} \textless{}A}] \leavevmode\begin{quote}

href=\sphinxquotedblright{}\sphinxurl{http://www.osc.edu/~pw/mpiexec/}\sphinxquotedblleft{}\textgreater{}
\end{quote}

end\{rawhtml\}
MPIexec
begin\{rawhtml\} \textless{}/A\textgreater{} end\{rawhtml\}

\end{description}

end\{itemize\}

\item[{item Build the code with the texttt\{genmake2\} texttt\{-mpi\} option}] \leavevmode
(see Section ref\{sec:genmake\}) using commands such as:

\item[{\{footnotesize begin\{verbatim\}}] \leavevmode
\%  ../../../tools/genmake2 -mods=../code -mpi -of=YOUR\_OPTFILE
\%  make depend
\%  make

\end{description}

end\{verbatim\} \}
\begin{description}
\item[{item Run the code with the appropriate MPI {\color{red}\bfseries{}{}`{}`}run'' or {\color{red}\bfseries{}{}`{}`}exec'`}] \leavevmode
program provided with your particular implementation of MPI.
Typical MPI packages such as MPICH will use something like:

\item[{begin\{verbatim\}}] \leavevmode
\%  mpirun -np 4 -machinefile mf ./mitgcmuv

\item[{end\{verbatim\}}] \leavevmode
Sightly more complicated scripts may be needed for many machines
since execution of the code may be controlled by both the MPI
library and a job scheduling and queueing system such as PBS,
LoadLeveller, Condor, or any of a number of similar tools.  A few
example scripts (those used for our begin\{rawhtml\} \textless{}A
\begin{quote}

href=\sphinxquotedblright{}\sphinxurl{http://mitgcm.org/public/testing.html}\sphinxquotedblleft{}\textgreater{} end\{rawhtml\}regular
\end{quote}

verification runsbegin\{rawhtml\} \textless{}/A\textgreater{} end\{rawhtml\}) are available
at:
begin\{rawhtml\} \textless{}A
\begin{quote}

href=\sphinxquotedblright{}\sphinxurl{http://mitgcm.org/viewvc/MITgcm/MITgcm/tools/example\_scripts/}\sphinxquotedblleft{}\textgreater{}
\end{quote}

end\{rawhtml\}
\{footnotesize tt
\begin{quote}

\sphinxurl{http://mitgcm.org/viewvc/MITgcm/MITgcm/tools/example\_scripts/} \}
\end{quote}

begin\{rawhtml\} \textless{}/A\textgreater{} end\{rawhtml\}
or at:
begin\{rawhtml\} \textless{}A
\begin{quote}

href=\sphinxquotedblright{}\sphinxurl{http://mitgcm.org/viewvc/MITgcm/MITgcm\_contrib/test\_scripts/}\sphinxquotedblleft{}\textgreater{}
\end{quote}

end\{rawhtml\}
\{footnotesize tt
\begin{quote}

\sphinxurl{http://mitgcm.org/viewvc/MITgcm/MITgcm\_contrib/test\_scripts/} \}
\end{quote}

begin\{rawhtml\} \textless{}/A\textgreater{} end\{rawhtml\}

\end{description}

end\{enumerate\}

An example of the above process on the MITgcm cluster ({\color{red}\bfseries{}{}`{}`}cg01'`) using
the GNU g77 compiler and the mpich MPI library is:
\begin{description}
\item[{\{footnotesize begin\{verbatim\}}] \leavevmode
\%  cd MITgcm/verification/exp5
\%  mkdir build
\%  cd build
\%  ../../../tools/genmake2 -mpi -mods=../code 
\begin{quote}

-of=../../../tools/build\_options/linux\_ia32\_g77+mpi\_cg01
\end{quote}

\%  make depend
\%  make
\%  cd ../input
\%  /usr/local/pkg/mpi/mpi-1.2.4..8a-gm-1.5/g77/bin/mpirun.ch\_gm 
\begin{quote}

-machinefile mf --gm-kill 5 -v -np 2  ../build/mitgcmuv
\end{quote}

\end{description}

end\{verbatim\} \}

section{[}Running MITgcm{]}\{Running the model in prognostic mode\}
label\{sec:runModel\}
begin\{rawhtml\}
\textless{}!-- CMIREDIR:runModel: --\textgreater{}
end\{rawhtml\}

If compilation finished succesfully (section ref\{sec:buildingCode\})
then an executable called texttt\{mitgcmuv\} will now exist in the
local directory.

To run the model as a single process (textit\{ie.\} not in parallel)
simply type:
begin\{verbatim\}
\% ./mitgcmuv
end\{verbatim\}
The {\color{red}\bfseries{}{}`{}`}./'' is a safe-guard to make sure you use the local executable
in case you have others that exist in your path (surely odd if you
do!). The above command will spew out many lines of text output to
your screen.  This output contains details such as parameter values as
well as diagnostics such as mean Kinetic energy, largest CFL number,
etc. It is worth keeping this text output with the binary output so we
normally re-direct the texttt\{stdout\} stream as follows:
begin\{verbatim\}
\% ./mitgcmuv \textgreater{} output.txt
end\{verbatim\}
In the event that the model encounters an error and stops, it is very
helpful to include the last few line of this texttt\{output.txt\} file
along with the (texttt\{stderr\}) error message within any bug reports.

For the example experiments in texttt\{verification\}, an example of the
output is kept in texttt\{results/output.txt\} for comparison. You can
compare your texttt\{output.txt\} with the corresponding one for that
experiment to check that the set-up works.

subsection\{Output files\}

The model produces various output files and, when using texttt\{mnc\},
sometimes even directories.  Depending upon the I/O package(s)
selected at compile time (either texttt\{mdsio\} or texttt\{mnc\} or
both as determined by texttt\{code/packages.conf\}) and the run-time
flags set (in texttt\{input/data.pkg\}), the following output may
appear.

subsubsection\{MDSIO output files\}

The {\color{red}\bfseries{}{}`{}`}traditional'' output files are generated by the texttt\{mdsio\}
package.  At a minimum, the instantaneous {\color{red}\bfseries{}{}`{}`}state'' of the model is
written out, which is made of the following files:

begin\{itemize\}
item texttt\{U.00000nIter\} - zonal component of velocity field (m/s
\begin{quote}

and positive eastward).
\end{quote}
\begin{description}
\item[{item texttt\{V.00000nIter\} - meridional component of velocity field}] \leavevmode
(m/s and positive northward).

\item[{item texttt\{W.00000nIter\} - vertical component of velocity field}] \leavevmode
(ocean: m/s and positive upward, atmosphere: Pa/s and positive
towards increasing pressure i.e. downward).

\item[{item texttt\{T.00000nIter\} - potential temperature (ocean:}] \leavevmode
\$\textasciicircum{}\{circ\}mathrm\{C\}\$, atmosphere: \$\textasciicircum{}\{circ\}mathrm\{K\}\$).

\item[{item texttt\{S.00000nIter\} - ocean: salinity (psu), atmosphere: water}] \leavevmode
vapor (g/kg).

\item[{item texttt\{Eta.00000nIter\} - ocean: surface elevation (m),}] \leavevmode
atmosphere: surface pressure anomaly (Pa).

\end{description}

end\{itemize\}

The chain texttt\{00000nIter\} consists of ten figures that specify the
iteration number at which the output is written out. For example,
texttt\{U.0000000300\} is the zonal velocity at iteration 300.

In addition, a {\color{red}\bfseries{}{}`{}`}pickup'' or {\color{red}\bfseries{}{}`{}`}checkpoint'' file called:

begin\{itemize\}
item texttt\{pickup.00000nIter\}
end\{itemize\}

is written out. This file represents the state of the model in a condensed
form and is used for restarting the integration. If the C-D scheme is used,
there is an additional {\color{red}\bfseries{}{}`{}`}pickup'' file:

begin\{itemize\}
item texttt\{pickup\_cd.00000nIter\}
end\{itemize\}

containing the D-grid velocity data and that has to be written out as well
in order to restart the integration. Rolling checkpoint files are the same
as the pickup files but are named differently. Their name contain the chain
texttt\{ckptA\} or texttt\{ckptB\} instead of texttt\{00000nIter\}. They can be
used to restart the model but are overwritten every other time they are
output to save disk space during long integrations.

subsubsection\{MNC output files\}

Unlike the texttt\{mdsio\} output, the texttt\{mnc\}--generated output
is usually (though not necessarily) placed within a subdirectory with
a name such as texttt\{mnc\_test\_\$\{DATE\}\_\$\{SEQ\}\}.

subsection\{Looking at the output\}

The {\color{red}\bfseries{}{}`{}`}traditional'' or mdsio model data are written according to a
{\color{red}\bfseries{}{}`{}`}meta/data'' file format.  Each variable is associated with two files
with suffix names texttt\{.data\} and texttt\{.meta\}. The
texttt\{.data\} file contains the data written in binary form
(big\_endian by default). The texttt\{.meta\} file is a {\color{red}\bfseries{}{}`{}`}header'' file
that contains information about the size and the structure of the
texttt\{.data\} file. This way of organizing the output is particularly
useful when running multi-processors calculations. The base version of
the model includes a few matlab utilities to read output files written
in this format. The matlab scripts are located in the directory
texttt\{utils/matlab\} under the root tree. The script texttt\{rdmds.m\}
reads the data. Look at the comments inside the script to see how to
use it.

Some examples of reading and visualizing some output in \{em Matlab\}:
begin\{verbatim\}
\% matlab
\textgreater{}\textgreater{} H=rdmds(`Depth');
\textgreater{}\textgreater{} contourf(H');colorbar;
\textgreater{}\textgreater{} title(`Depth of fluid as used by model');

\textgreater{}\textgreater{} eta=rdmds(`Eta',10);
\textgreater{}\textgreater{} imagesc(eta');axis ij;colorbar;
\textgreater{}\textgreater{} title(`Surface height at iter=10');

\textgreater{}\textgreater{} eta=rdmds(`Eta',{[}0:10:100{]});
\textgreater{}\textgreater{} for n=1:11; imagesc(eta(:,:,n)');axis ij;colorbar;pause(.5);end
end\{verbatim\}

Similar scripts for netCDF output (texttt\{rdmnc.m\}) are available and
they are described in Section ref\{sec:pkg:mnc\}.

The MNC output files are all in the {\color{red}\bfseries{}{}`{}`}self-describing'' netCDF
format and can thus be browsed and/or plotted using tools such as:
begin\{itemize\}
item texttt\{ncdump\} is a utility which is typically included
\begin{quote}

with every netCDF install:
begin\{rawhtml\} \textless{}A href=\sphinxquotedblright{}\sphinxurl{http://www.unidata.ucar.edu/packages/netcdf/}\sphinxquotedblleft{}\textgreater{} end\{rawhtml\}
\end{quote}

begin\{verbatim\}
\sphinxurl{http://www.unidata.ucar.edu/packages/netcdf/}
end\{verbatim\}
\begin{quote}

begin\{rawhtml\} \textless{}/A\textgreater{} end\{rawhtml\} and it converts the netCDF
binaries into formatted ASCII text files.
\end{quote}
\begin{description}
\item[{item texttt\{ncview\} utility is a very convenient and quick way}] \leavevmode
to plot netCDF data and it runs on most OSes:
begin\{rawhtml\} \textless{}A href=\sphinxquotedblright{}\sphinxurl{http://meteora.ucsd.edu/~pierce/ncview\_home\_page.html}\sphinxquotedblleft{}\textgreater{} end\{rawhtml\}

\end{description}

begin\{verbatim\}
\sphinxurl{http://meteora.ucsd.edu/~pierce/ncview\_home\_page.html}
end\{verbatim\}
\begin{quote}

begin\{rawhtml\} \textless{}/A\textgreater{} end\{rawhtml\}
\end{quote}
\begin{description}
\item[{item MatLAB(c) and other common post-processing environments provide}] \leavevmode
various netCDF interfaces including:
begin\{rawhtml\} \textless{}A href=\sphinxquotedblright{}\sphinxurl{http://mexcdf.sourceforge.net/}\sphinxquotedblleft{}\textgreater{} end\{rawhtml\}

\end{description}

begin\{verbatim\}
\sphinxurl{http://mexcdf.sourceforge.net/}
end\{verbatim\}
\begin{quote}

begin\{rawhtml\} \textless{}/A\textgreater{} end\{rawhtml\}
begin\{rawhtml\} \textless{}A href=\sphinxquotedblright{}\sphinxurl{http://woodshole.er.usgs.gov/staffpages/cdenham/public\_html/MexCDF/nc4ml5.html}\sphinxquotedblleft{}\textgreater{} end\{rawhtml\}
\end{quote}

begin\{verbatim\}
\sphinxurl{http://woodshole.er.usgs.gov/staffpages/cdenham/public\_html/MexCDF/nc4ml5.html}
end\{verbatim\}
\begin{quote}

begin\{rawhtml\} \textless{}/A\textgreater{} end\{rawhtml\}
\end{quote}

end\{itemize\}


\chapter{MITgcm Example Experiments}
\label{\detokenize{examples/examples:mitgcm-example-experiments}}\label{\detokenize{examples/examples::doc}}\label{\detokenize{examples/examples:chap-modelexamples}}
The full MITgcm distribution comes with a set of pre-configured
numerical experiments.  Some of these example experiments are tests of
individual parts of the model code, but many are fully fledged
numerical simulations. Full tutorials exist for a few of the examples,
and are documented in sections ref\{sec:eg-baro\} -
ref\{sec:eg-tank\}. The other examples follow the same general
structure as the tutorial examples. However, they only include brief
instructions in a text file called \{it README\}.  The examples are
located in subdirectories under the directory texttt\{verification\}.
Each example is briefly described below.

subsection\{Full list of model examples\}

begin\{enumerate\}
\begin{description}
\item[{item texttt\{tutorial\_advection\_in\_gyre\} - Test of various}] \leavevmode
advection schemes in a single-layer double-gyre experiment.
This experiment is described in detail in section
ref\{sec:eg-adv-gyre\}.

\item[{item texttt\{tutorial\_baroclinic\_gyre\} - Four layer, ocean double}] \leavevmode
gyre. This experiment is described in detail in section
ref\{sec:eg-fourlayer\}.

\item[{item texttt\{tutorial\_barotropic\_gyre\} - Single layer, ocean double}] \leavevmode
gyre (barotropic with free-surface).
This experiment is described in detail in section ref\{sec:eg-baro\}.

\item[{item texttt\{tutorial\_cfc\_offline\} - Offline form of the MITgcm to}] \leavevmode
study advection of a passive tracer and CFCs.
This experiment is described in detail in section ref\{sec:eg-offline-cfc\}.

\item[{item texttt\{tutorial\_deep\_convection\} - Non-uniformly forced}] \leavevmode
ocean convection in a doubly periodic box. This experiment is
described in detail in section ref\{sec:eg-bconv\}.

\item[{item texttt\{tutorial\_dic\_adoffline\} - Offline form of MITgcm}] \leavevmode
dynamics coupled to the dissolved inorganic carbon biogeochemistry model;
adjoint set-up.

\item[{item texttt\{tutorial\_global\_oce\_biogeo\} - Ocean model coupled to}] \leavevmode
the dissolved inorganic carbon biogeochemistry model. This
experiment is described in detail in section
ref\{sec:eg-biogeochem\_tutorial\}.

\item[{item texttt\{tutorial\_global\_oce\_in\_p\} - Global ocean simulation in}] \leavevmode
pressure coordinate (non-Boussinesq ocean model). Described in
detail in section ref\{sec:eg-globalpressure\}.

\item[{item texttt\{tutorial\_global\_oce\_latlon\} - 4x4 degree global ocean}] \leavevmode
simulation with steady climatological forcing. This experiment is
described in detail in section ref\{sec:eg-global\}.

\item[{item texttt\{tutorial\_global\_oce\_optim\} - Global ocean state}] \leavevmode
estimation at \$4\textasciicircum{}circ\$ resolution.  This experiment is described in
detail in section ref\{sec:eg-global\_state\_estimate\}.

\item[{item texttt\{tutorial\_held\_suarez\_cs\} - 3D atmosphere dynamics}] \leavevmode
using Held and Suarez (1994) forcing on cubed sphere grid.  This
experiment is described in detail in section ref\{sec:eg-hs\}.

\item[{item texttt\{tutorial\_offline\} - Offline form of the MITgcm to study}] \leavevmode
advection of a passive tracer.  This experiment is described in
detail in section ref\{sec:eg-offline\}.

\item[{item texttt\{tutorial\_plume\_on\_slope\} - Gravity Plume on a}] \leavevmode
continental slope.  This experiment is described in detail in
section ref\{sec:eg-gravityplume\}.

\item[{item texttt\{tutorial\_tracer\_adjsens\} - Simple passive tracer}] \leavevmode
experiment. Includes derivative calculation. This experiment is
described in detail in section ref\{sec:eg-simple-tracer-adjoint\}.\textbackslash{}
Also contains an additional set-up using Secon Order Moment (SOM) advection
scheme (\{it input\_ad.som81/\}).

\end{description}

item texttt\{1D\_ocean\_ice\_column\} - Oceanic column with seaice on top.
\begin{description}
\item[{item texttt\{adjustment.128x64x1\} - Barotropic adjustment problem on}] \leavevmode
latitude longitude grid with 128x64 grid points (\$2.8\textasciicircum{}circ\$ resolution).

\item[{item texttt\{adjustment.cs-32x32x1\} - Barotropic adjustment problem on}] \leavevmode
cube sphere grid with 32x32 points per face (roughly \$2.8\textasciicircum{}circ\$
resolution).\textbackslash{}
Also contains a non-linear free-surface adjustment version (\{it input.nlfs/\}).

\item[{item texttt\{advect\_cs\} - Two-dimensional passive advection test on}] \leavevmode
cube sphere grid (32x32 grid points per face, roughly \$2.8\textasciicircum{}circ\$ resolution)

\item[{item texttt\{advect\_xy\} - Two-dimensional (horizontal plane) passive}] \leavevmode
advection test on Cartesian grid.\textbackslash{}
Also contains an additional set-up using Adams-Bashforth 3 (\{it input.ab3\_c4/\}).

\item[{item texttt\{advect\_xz\} - Two-dimensional (vertical plane) passive}] \leavevmode
advection test on Cartesian grid.\textbackslash{}
Also contains an additional set-up using non-linear free-surface
\begin{quote}

with divergent barotropic flow and implicit vertical advection (\{it input.nlfs/\}).
\end{quote}

\item[{item texttt\{aim.5l\_Equatorial\_Channel\} -}] \leavevmode
5-levels Intermediate Atmospheric physics,
3D Equatorial Channel configuration.

\item[{item texttt\{aim.5l\_LatLon\} - 5-levels Intermediate Atmospheric physics,}] \leavevmode
Global configuration, on latitude longitude grid with 128x64x5 grid
points (\$2.8\textasciicircum{}circ\$ resolution).

\item[{item texttt\{aim.5l\_cs\} - 5-levels Intermediate Atmospheric physics,}] \leavevmode
Global configuration on cube sphere grid
(32x32 grid points per face, roughly \$2.8\textasciicircum{}circ\$).\textbackslash{}
Also contains an additional set-up with a slab-ocean and thermodynamic
sea-ice (\{it input.thSI/\}).

\item[{item texttt\{bottom\_ctrl\_5x5\} - Adjoint test using the bottom}] \leavevmode
topography as the control parameter.

\item[{item texttt\{cfc\_example\} - Global ocean with online computation and}] \leavevmode
advection of CFC11 and CFC12.

\item[{item texttt\{cheapAML\_box\} - Example using cheap atmospheric mixed layer}] \leavevmode
(cheapAML) package.

\item[{item texttt\{cpl\_aim+ocn\} - Coupled Ocean - Atmosphere realistic}] \leavevmode
configuration on cubed-sphere cs32 horizontal grid,
using Intermediate Atmospheric physics (\{it pkg/aim\_v23\})
thermodynamic seaice (\{it pkg/thsice\}) and land packages.
on cubed-sphere cs32 in a realistic configuration.

\item[{item texttt\{cpl\_atm2d+ocn\} - Coupled Ocean - Atmosphere realistic}] \leavevmode
configuration using 2-D Atmospheric Model (\{it pkg/atm2d\}).

\item[{item texttt\{deep\_anelastic\} - Convection simulation on a giant planet:}] \leavevmode
relax both the Boussinesq approximation (anelastic) and the thin atmosphere
approximation (deep atmosphere).

\end{description}

item texttt\{dome\} - Idealized 3D test of a density-driven bottom current.
\begin{description}
\item[{item texttt\{exp2\} - Old version of the global ocean experiment (no GM,}] \leavevmode\begin{quote}

no partial-cells).\textbackslash{}
\end{quote}

Also contains an additional set-up with rigid-lid (\{it input.rigidLid/\}).

\item[{item texttt\{exp4\} - Flow over a Gaussian bump in open-water or}] \leavevmode
channel with open boundaries.\textbackslash{}
Also contains an additional set-up using non-linear free-surface (\{it input.nlfs/\}).

\item[{item texttt\{fizhi-cs-32x32x40\} - Global atmospheric simulation with}] \leavevmode
realistic topography, 40 vertical levels, a cubed sphere grid and
the full atmospheric physics package.

\item[{item texttt\{fizhi-cs-aqualev20\} - Global atmospheric simulation on an}] \leavevmode
aqua planet with full atmospheric physics. Run is perpetual march
with an analytical SST distribution.  This is the configuration for
the APE (Aqua Planet Experiment) participation experiment.

\item[{item texttt\{fizhi-gridalt-hs\} - Global atmospheric simulation}] \leavevmode
Held-Suarez (1994) forcing, with the physical forcing and the
dynamical forcing running on different vertical grids.

\end{description}

item texttt\{flt\_example\} - Example of using float package.
\begin{description}
\item[{item texttt\{front\_relax\} - Relaxation of an ocean thermal front}] \leavevmode
(test for Gent/McWilliams scheme). 2D (y-z).\textbackslash{}
Also contains additional set-ups:
begin\{enumerate\}
\begin{quote}
\begin{description}
\item[{item using the Boundary-Value Problem method}] \leavevmode
(Ferrari et al., 2010) (\{it input.bvp/\}).

\item[{item with Mixed-Layer Eddy parameterization}] \leavevmode
(Ferrari \& McWilliams, 2007) (\{it input.mxl/\}).

\end{description}
\end{quote}

end\{enumerate\}

\item[{item texttt\{global\_ocean.90x40x15\} - Global ocean simulation at 4x4}] \leavevmode
degree resolution. Similar to tutorial\_global\_oce\_latlon, but using
\$z\textasciicircum{}*\$ coordinates with quasi-non-hydrostatic and non-hydrostatic metric terms.
This experiment also illustrate the use of SBO package.
Also contains additional set-ups:
begin\{enumerate\}
\begin{quote}

item using down-slope package (\{it pkg/down\_slope\}) (\{it input.dwnslp/\})
item an Open-AD adjoint set-up (\{it code\_oad/, input\_oad/\}).
item four TAF adjoint set-ups (\{it code\_ad/\}):
begin\{enumerate\}
\begin{quote}

item standard experiment (\{it input\_ad/\}).
item with bottom drag as a control (\{it input\_ad.bottomdrag/\}).
item with kappa GM as a control (\{it input\_ad.kapgm/\}).
item with kappa Redi as a control (\{it input\_ad.kapredi/\}).
\end{quote}

end\{enumerate\}
\end{quote}

end\{enumerate\}

\item[{item texttt\{global\_ocean.cs32x15\} - Global ocean experiment on the}] \leavevmode
cubed sphere grid.\textbackslash{}
Also contains additional forward set-ups:
begin\{enumerate\}
\begin{quote}

item non-hydrostatic with biharmonic viscosity (\{it input.viscA4/\})
item using thermodynamic sea ice and bulk force (\{it input.thsice/\})
item using thermodynamic (\{it pkg/thsice\}) dynamic (\{it pkg/seaice\}) sea-ice
\begin{quote}

and \{it exf\} package (\{it input.icedyn/\})
\end{quote}
\begin{description}
\item[{item using thermodynamic - dynamic (\{it pkg/seaice\}) sea-ice}] \leavevmode
with \{it exf\} package (\{it input.seaice/\})

\end{description}
\end{quote}

end\{enumerate\}
and few additional adjoint set-ups (\{it code\_ad/\}):
begin\{enumerate\}
\begin{quote}

item standard experiment without sea-ice (\{it input\_ad/\}).
item using thermodynamic - dynamic sea-ice (\{it input\_ad.seaice/\})
item same as above without adjoint sea-ice dynamics (\{it input\_ad.seaice\_dynmix/\})
item using thermodynamic sea-ice from \{it thsice\} package (\{it input\_ad.thsice/\})
\end{quote}

end\{enumerate\}

\item[{item texttt\{global\_ocean\_ebm\} - Global ocean experiment on a lat-lon}] \leavevmode
grid coupled to an atmospheric energy balance model. Similar to
global\_ocean.90x40x15 experiment.\textbackslash{}
Also contains an adjoint set-up (\{it code\_ad/, input\_ad/\}).

\item[{item texttt\{global\_with\_exf\} - Global ocean experiment on a lat-lon}] \leavevmode
grid using the \{it exf\} package. Similar to tutorial\_global\_oce\_latlon
experiment.\textbackslash{}
Also contains a secondary set-up with yearly \{it exf\} fields (\{it input\_ad.yearly/\}).

\item[{item texttt\{halfpipe\_streamice\} - Example using package \sphinxquotedblleft{}streamice\sphinxquotedblright{}.\textbackslash{}}] \leavevmode\begin{description}
\item[{Also contains adjoint set-ups using TAF (\{it code\_ad/, input\_ad/\})}] \leavevmode
and using Open-AD (\{it code\_oad/, input\_oad/\}).

\end{description}

\item[{item texttt\{hs94.128x64x5\} - 3D atmosphere dynamics on lat-lon grid,}] \leavevmode
using Held and Suarez `94 forcing.

\item[{item texttt\{hs94.1x64x5\} - Zonal averaged atmosphere dynamics}] \leavevmode
using Held and Suarez `94 forcing.\textbackslash{}
Also contains adjoint set-ups using TAF (\{it code\_ad/, input\_ad/\})
\begin{quote}

and using Open-AD (\{it code\_oad/, input\_oad/\}).
\end{quote}

\item[{item texttt\{hs94.cs-32x32x5\} - 3D atmosphere dynamics using Held and}] \leavevmode
Suarez (1994) forcing on the cubed sphere, similar to tutorial\_held\_suarez\_cs
experiment but using linear free-surface and only 5 levels.\textbackslash{}
Also contains an additional set-up with Implicit Internal gravity waves
treatment and Adams-Bashforth 3 (\{it input.impIGW/\}).

\item[{item texttt\{ideal\_2D\_oce\} - Idealized 2D global ocean simulation on}] \leavevmode
an aqua planet.

\item[{item texttt\{internal\_wave\} - Ocean internal wave forced by open}] \leavevmode
boundary conditions.\textbackslash{}
Also contains an additional set-up using \{it pkg/kl10\} (see section
\begin{quote}

ref\{sec:pkg:kl10\}, Klymak and Legg, 2010) (\{it input.kl10/\}).
\end{quote}

\item[{item texttt\{inverted\_barometer\} - Simple test of ocean response to}] \leavevmode
atmospheric pressure loading.

\item[{item texttt\{isomip\} - ISOMIP like set-up including ice-shelf cavities}] \leavevmode
(\{it pkg/shelfice\}).\textbackslash{}
Also contains additional set-ups:
begin\{enumerate\}
\begin{quote}
\begin{description}
\item[{item with \sphinxquotedblleft{}htd\sphinxquotedblright{} (\{it input.htd/\})}] \leavevmode
but only Martin knows what \sphinxquotedblleft{}htd\sphinxquotedblright{} stands for.

\end{description}

item using package \{it icefront\} (\{it input.icefront\})
\end{quote}

end\{enumerate\}
and also adjoint set-ups using TAF (\{it code\_ad/, input\_ad/, input\_ad.htd/\})
or using Open-AD (\{it code\_oad/, input\_oad/\}).

\item[{item texttt\{lab\_sea\} - Regional Labrador Sea simulation on a lat-lon}] \leavevmode
grid using the sea ice package.\textbackslash{}
Also contains additional set-ups:
begin\{enumerate\}
\begin{quote}

item using the simple \sphinxquotedblleft{}free-drift\sphinxquotedblright{} assumption for seaice (\{it input.fd/\})
item using EVP dynamics (instead of LSR solver) and Hibler \& Bryan (1987)
\begin{quote}

sea-ice ocean stress (\{it input.hb87/\})
\end{quote}

item using package \{it salt\_plume\} (\{it input.salt\_plume/\})
\end{quote}

end\{enumerate\}
and also 3 adjoint set-ups (\{it code\_ad/, input\_ad/, input\_ad.noseaicedyn/,
\begin{quote}

input\_ad.noseaice/\}).
\end{quote}

\item[{item texttt\{matrix\_example\} - Test of experimental method to}] \leavevmode
accelerated convergence towards equilibrium.

\item[{item texttt\{MLAdjust\} - Simple tests for different viscosity formulations.\textbackslash{}}] \leavevmode
Also contains additional set-ups (see: \{it verification/MLAdjust/README\}):
begin\{enumerate\}
\begin{quote}

item (\{it input.A4FlxF/\})
item (\{it input.AhFlxF/\})
item (\{it input.AhVrDv/\})
item (\{it input.AhStTn/\})
\end{quote}

end\{enumerate\}

\item[{item texttt\{natl\_box\} - Eastern subtropical North Atlantic with KPP}] \leavevmode
scheme; 1 month integration

\item[{item texttt\{obcs\_ctrl\} - Adjoint test using Open-Boundary conditions}] \leavevmode
as control parameters.

\item[{item texttt\{offline\_exf\_seaice\} - Seaice on top of oceanic surface layer}] \leavevmode
in an idealized channel. Forcing is computed by bulk-formulae (\{it pkg/exf\})
with temperature relaxation to prescribed SST (offline ocean).\textbackslash{}
Also contains additional set-ups:
begin\{enumerate\}
\begin{quote}
\begin{description}
\item[{item sea-ice dynamics-only using JFNK solver}] \leavevmode
and \{it pkg/thsice\} advection (\{it input.dyn\_jfnk/\})

\item[{item sea-ice dynamics-only using LSR solver}] \leavevmode
and \{it pkg/seaice\} advection (\{it input.dyn\_lsr/\})

\end{description}

item sea-ice thermodynamics-only using \{it pkg/seaice\} (\{it input.thermo/\})
item sea-ice thermodynamics-only using \{it pkg/thsice\} (\{it input.thsice/\})
\end{quote}

end\{enumerate\}
and also 2 adjoint set-ups (\{it code\_ad/, input\_ad/, input\_ad.thsice/\}).

\item[{item texttt\{OpenAD\} - Simple Adjoint experiment (used also to test}] \leavevmode
Open-AD compiler)

\item[{item texttt\{rotating\_tank\} - Rotating tank simulation in cylindrical}] \leavevmode
coordinates.  This experiment is described in detail in section
ref\{sec:eg-tank\}.

\item[{item texttt\{seaice\_itd\} - Seaice example using Ice Thickness Distribution (ITD).\textbackslash{}}] \leavevmode
Also contains additional set-ups:
begin\{enumerate\}
\begin{quote}

item (\{it input.thermo/\})
item (\{it input.lipscomb07/\})
\end{quote}

end\{enumerate\}

\item[{item texttt\{seaice\_obcs\} - Similar to \sphinxquotedblleft{}lab\_sea\sphinxquotedblright{} (\{it input.salt\_plume/\})}] \leavevmode
experiment with only a fraction of the domain and open-boundary conditions
derived from \sphinxquotedblleft{}lab\_sea\sphinxquotedblright{} experiment.\textbackslash{}
Also contains additional set-ups:
begin\{enumerate\}
\begin{quote}

item (\{it input.seaiceSponge/\})
item (\{it input.tides/\})
\end{quote}

end\{enumerate\}

\item[{item texttt\{short\_surf\_wave\} - Short surface wave adjusment}] \leavevmode
(non-hydrostatic) in homogeneous 2-D vertical section (x-z).

\item[{item texttt\{so\_box\_biogeo\} - Open-boundary Southern ocean box around}] \leavevmode\begin{description}
\item[{Drake passage, using same model parameters and forcing as experiment}] \leavevmode
\sphinxquotedblleft{}tutorial\_global\_oce\_biogeo\sphinxquotedblright{} from which initial conditions

\end{description}

and OB conditions have been extracted.

\item[{item texttt\{solid-body.cs-32x32x1\} - Solid body rotation test for cube}] \leavevmode
sphere grid.

\item[{item texttt\{tidal\_basin\_2d\} - 2-D vertical section (x-z) with tidal forcing}] \leavevmode
(untested)

\item[{item texttt\{vermix\} - Simple test in a small domain (3 columns) for}] \leavevmode
ocean vertical mixing schemes. The standard set-up (\{it input/\}) uses
KPP scheme cite{[}{]}\{lar-eta:94\}.\textbackslash{}
Also contains additional set-ups:
begin\{enumerate\}
\begin{quote}

item with Double Diffusion scheme from KPP (\{it input.dd/\})
item with cite\{gas-eta:90\} (\{it pkg/ggl90\}) scheme (\{it input.ggl90/\})
item with cite\{Mellor:Yamada1982\} level 2. (\{it pkg/my82\}) scheme (\{it input.my82/\})
item with cite\{pal-rom:97\} (\{it pkg/opps\}) scheme (\{it input.opps/\})
item with cite\{Pacanowski:Philander1981\} (\{it pkg/pp81\}) scheme (\{it input.pp81/\})
\end{quote}

end\{enumerate\}

\end{description}

end\{enumerate\}

subsection\{Directory structure of model examples\}

Each example directory has the following subdirectories:

begin\{itemize\}
item texttt\{code\}: contains the code particular to the example. At a
\begin{quote}

minimum, this directory includes the following files:

begin\{itemize\}
item texttt\{code/packages.conf\}: declares the list of packages or
\begin{quote}

package groups to be used.  If not included, the default version
is located in texttt\{pkg/pkg\_default\}.  Package groups are
simply convenient collections of commonly used packages which are
defined in texttt\{pkg/pkg\_default\}.  Some packages may require
other packages or may require their absence (that is, they are
incompatible) and these package dependencies are listed in
texttt\{pkg/pkg\_depend\}.
\end{quote}
\begin{description}
\item[{item texttt\{code/CPP\_EEOPTIONS.h\}: declares CPP keys relative to}] \leavevmode
the {\color{red}\bfseries{}{}`{}`}execution environment'' part of the code. The default
version is located in texttt\{eesupp/inc\}.

\item[{item texttt\{code/CPP\_OPTIONS.h\}: declares CPP keys relative to}] \leavevmode
the {\color{red}\bfseries{}{}`{}`}numerical model'' part of the code. The default version is
located in texttt\{model/inc\}.

\item[{item texttt\{code/SIZE.h\}: declares size of underlying}] \leavevmode
computational grid.  The default version is located in
texttt\{model/inc\}.

\end{description}

end\{itemize\}

In addition, other include files and subroutines might be present in
texttt\{code\} depending on the particular experiment. See Section 2
for more details.
\end{quote}
\begin{description}
\item[{item texttt\{input\}: contains the input data files required to run}] \leavevmode
the example. At a minimum, the texttt\{input\} directory contains the
following files:

begin\{itemize\}
item texttt\{input/data\}: this file, written as a namelist,
\begin{quote}

specifies the main parameters for the experiment.
\end{quote}
\begin{description}
\item[{item texttt\{input/data.pkg\}: contains parameters relative to the}] \leavevmode
packages used in the experiment.

\item[{item texttt\{input/eedata\}: this file contains {\color{red}\bfseries{}{}`{}`}execution}] \leavevmode
environment'' data. At present, this consists of a specification
of the number of threads to use in \$X\$ and \$Y\$ under multi-threaded
execution.

\end{description}

end\{itemize\}

In addition, you will also find in this directory the forcing and
topography files as well as the files describing the initial state
of the experiment.  This varies from experiment to experiment. See
the verification directories refered to in this chapter for more details.

\item[{item texttt\{results\}: this directory contains the output file}] \leavevmode
texttt\{output.txt\} produced by the simulation example. This file is
useful for comparison with your own output when you run the
experiment.

\item[{item texttt\{build\}: this directory is initially empty and is used}] \leavevmode
to compile and load the model, and to generate the executable.

\item[{item texttt\{run\}: this directory is initially empty and is used}] \leavevmode
to run the executable.

\end{description}

end\{itemize\}

Once you have chosen the example you want to run, you are ready to
compile the code.

newpage


\section{Barotropic Gyre MITgcm Example}
\label{\detokenize{examples/examples:barotropic-gyre-mitgcm-example}}\label{\detokenize{examples/examples:sec-eg-baro}}
begin\{center\}
(in directory: \{it verification/tutorial\_barotropic\_gyre/\})
end\{center\}

This example experiment demonstrates using the MITgcm to simulate
a Barotropic, wind-forced, ocean gyre circulation. The files for this
experiment can be found in the verification directory tutorial\_barotropic\_gyre.
The experiment is a numerical rendition of the gyre circulation problem similar
to the problems described analytically by Stommel in 1966
cite\{Stommel66\} and numerically in Holland et. al cite\{Holland75\}.

In this experiment the model
is configured to represent a rectangular enclosed box of fluid,
\$1200 times 1200 \$\textasciitilde{}km in lateral extent. The fluid is \$5\$\textasciitilde{}km deep and is forced
by a constant in time zonal wind stress, \$tau\_x\$, that varies sinusoidally
in the {\color{red}\bfseries{}{}`{}`}north-south'' direction. Topologically the grid is Cartesian and
the coriolis parameter \$f\$ is defined according to a mid-latitude beta-plane
equation
\begin{quote}
\phantomsection\label{\detokenize{examples/examples:equation-eq_eg_baro_fcori}}\begin{equation}\label{equation:examples/examples:eq_eg_baro_fcori}
\begin{split}f(y) = f_{0}+\beta y\end{split}
\end{equation}\end{quote}

where \$y\$ is the distance along the {\color{red}\bfseries{}{}`{}`}north-south'' axis of the
simulated domain. For this experiment \$f\_\{0\}\$ is set to \$10\textasciicircum{}\{-4\}s\textasciicircum{}\{-1\}\$ in \eqref{equation:examples/examples:eq_eg_baro_fcori} and \$beta = 10\textasciicircum{}\{-11\}s\textasciicircum{}\{-1\}m\textasciicircum{}\{-1\}\$.

The sinusoidal wind-stress variations are defined according to
\begin{quote}
\begin{equation}\label{equation:examples/examples:eq_eg_baro_taux}
\begin{split}\tau_x(y) = \tau_{0}\sin(\pi \frac{y}{L_y})\end{split}
\end{equation}\end{quote}

where \(L_{y}\) is the lateral domain extent (1200\textasciitilde{}km) and
\(\tau_0\) is set to \(0.1N m^{-2}\).

\hyperref[\detokenize{examples/examples:fig-eg-baro-simulation-config}]{Figure \ref{\detokenize{examples/examples:fig-eg-baro-simulation-config}}} summarizes the configuration simulated.
\begin{quote}
\begin{figure}[htbp]
\centering
\capstart

\noindent\sphinxincludegraphics[width=0.700\linewidth]{{simulation_config}.pdf}
\caption{Schematic of simulation domain and wind-stress forcing function for barotropic gyre numerical experiment. The domain is enclosed bu solid walls at \$x=\$\textasciitilde{}0,1200km and at \$y=\$\textasciitilde{}0,1200km.}\label{\detokenize{examples/examples:fig-eg-baro-simulation-config}}\end{figure}
\end{quote}


\subsection{Equations Solved}
\label{\detokenize{examples/examples:equations-solved}}
The model is configured in hydrostatic form. The implicit free surface form of the
pressure equation described in Marshall et. al cite\{marshall:97a\} is
employed.
A horizontal Laplacian operator \$nabla\_\{h\}\textasciicircum{}2\$ provides viscous
dissipation. The wind-stress momentum input is added to the momentum equation
for the {\color{red}\bfseries{}{}`{}`}zonal flow'`, \$u\$. Other terms in the model
are explicitly switched off for this experiment configuration (see section
ref\{sec:eg-baro-code\_config\} ), yielding an active set of equations solved
in this configuration as follows

begin\{eqnarray\}
label\{eq:eg-baro-model\_equations\}
frac\{Du\}\{Dt\} - fv +
\begin{quote}

gfrac\{partial eta\}\{partial x\} -
A\_\{h\}nabla\_\{h\}\textasciicircum{}2u
\end{quote}

\& = \&
frac\{tau\_\{x\}\}\{rho\_\{0\}Delta z\}
\textbackslash{}
frac\{Dv\}\{Dt\} + fu + gfrac\{partial eta\}\{partial y\} -
\begin{quote}

A\_\{h\}nabla\_\{h\}\textasciicircum{}2v
\end{quote}

\& = \&
0
\textbackslash{}
frac\{partial eta\}\{partial t\} + nabla\_\{h\}cdot vec\{u\}
\&=\&
0
end\{eqnarray\}

noindent where \$u\$ and \$v\$ and the \$x\$ and \$y\$ components of the
flow vector \$vec\{u\}\$.
\textbackslash{}

subsection\{Discrete Numerical Configuration\}
\%label\{www:tutorials\}
\begin{quote}

The domain is discretised with
\end{quote}
\begin{description}
\item[{a uniform grid spacing in the horizontal set to}] \leavevmode
\$Delta x=Delta y=20\$\textasciitilde{}km, so

\end{description}

that there are sixty grid cells in the \$x\$ and \$y\$ directions. Vertically the
model is configured with a single layer with depth, \$Delta z\$, of \$5000\$\textasciitilde{}m.

subsubsection\{Numerical Stability Criteria\}
\%label\{www:tutorials\}

The Laplacian dissipation coefficient, \$A\_\{h\}\$, is set to \$400 m s\textasciicircum{}\{-1\}\$.
This value is chosen to yield a Munk layer width cite\{adcroft:95\},

begin\{eqnarray\}
label\{eq:eg-baro-munk\_layer\}
M\_\{w\} = pi ( frac \{ A\_\{h\} \}\{ beta \} )\textasciicircum{}\{frac\{1\}\{3\}\}
end\{eqnarray\}

noindent  of \$approx 100\$km. This is greater than the model
resolution \$Delta x\$, ensuring that the frictional boundary
layer is well resolved.
\textbackslash{}

noindent The model is stepped forward with a
time step \$delta t=1200\$secs. With this time step the stability
parameter to the horizontal Laplacian friction cite\{adcroft:95\}

begin\{eqnarray\}
label\{eq:eg-baro-laplacian\_stability\}
S\_\{l\} = 4 frac\{A\_\{h\} delta t\}\{\{Delta x\}\textasciicircum{}2\}
end\{eqnarray\}

noindent evaluates to 0.012, which is well below the 0.3 upper limit
for stability.
\textbackslash{}

noindent The numerical stability for inertial oscillations
cite\{adcroft:95\}

begin\{eqnarray\}
label\{eq:eg-baro-inertial\_stability\}
S\_\{i\} = f\textasciicircum{}\{2\} \{delta t\}\textasciicircum{}2
end\{eqnarray\}

noindent evaluates to \$0.0144\$, which is well below the \$0.5\$ upper
limit for stability.
\textbackslash{}

noindent The advective CFL cite\{adcroft:95\} for an extreme maximum
horizontal flow speed of \$ \textbar{} vec\{u\} \textbar{} = 2 ms\textasciicircum{}\{-1\}\$

begin\{eqnarray\}
label\{eq:eg-baro-cfl\_stability\}
S\_\{a\} = frac\{\textbar{} vec\{u\} \textbar{} delta t\}\{ Delta x\}
end\{eqnarray\}

noindent evaluates to 0.12. This is approaching the stability limit
of 0.5 and limits \$delta t\$ to \$1200s\$.


\subsection{Code Configuration}
\label{\detokenize{examples/examples:code-configuration}}\label{\detokenize{examples/examples:sec-eg-baro-code-config}}
The model configuration for this experiment resides under the
directory \sphinxcode{verification/tutorial\_barotropic\_gyre/}.
\begin{description}
\item[{The experiment files}] \leavevmode\begin{itemize}
\item {} 
\sphinxcode{input/data}

\item {} 
\sphinxcode{input/data.pkg}

\item {} 
\sphinxcode{input/eedata}

\item {} 
\sphinxcode{input/windx.sin\_y}

\item {} 
\sphinxcode{input/topog.box}

\item {} 
\sphinxcode{code/CPP\_EEOPTIONS.h}

\item {} 
\sphinxcode{code/CPP\_OPTIONS.h}

\item {} 
\sphinxcode{code/SIZE.h}

\end{itemize}

\end{description}

contain the code customizations and parameter settings for this
experiments. Below we describe the customizations
to these files associated with this experiment.

subsubsection\{File \{it input/data\}\}
\%label\{www:tutorials\}

This file, reproduced completely below, specifies the main parameters
for the experiment. The parameters that are significant for this configuration
are

begin\{itemize\}

item Line 7, begin\{verbatim\} viscAh=4.E2, end\{verbatim\} this line sets
the Laplacian friction coefficient to \$400 m\textasciicircum{}2s\textasciicircum{}\{-1\}\$
item Line 10, begin\{verbatim\} beta=1.E-11, end\{verbatim\} this line sets
\$beta\$ (the gradient of the coriolis parameter, \$f\$) to \$10\textasciicircum{}\{-11\} s\textasciicircum{}\{-1\}m\textasciicircum{}\{-1\}\$

item Lines 15 and 16
begin\{verbatim\}
rigidLid=.FALSE.,
implicitFreeSurface=.TRUE.,
end\{verbatim\}
these lines suppress the rigid lid formulation of the surface
pressure inverter and activate the implicit free surface form
of the pressure inverter.

item Line 27,
begin\{verbatim\}
startTime=0,
end\{verbatim\}
this line indicates that the experiment should start from \$t=0\$
and implicitly suppresses searching for checkpoint files associated
with restarting an numerical integration from a previously saved state.

item Line 29,
begin\{verbatim\}
endTime=12000,
end\{verbatim\}
this line indicates that the experiment should start finish at \$t=12000s\$.
A restart file will be written at this time that will enable the
simulation to be continued from this point.

item Line 30,
begin\{verbatim\}
deltaTmom=1200,
end\{verbatim\}
This line sets the momentum equation timestep to \$1200s\$.

item Line 39,
begin\{verbatim\}
usingCartesianGrid=.TRUE.,
end\{verbatim\}
This line requests that the simulation be performed in a
Cartesian coordinate system.

item Line 41,
begin\{verbatim\}
delX=60*20E3,
end\{verbatim\}
This line sets the horizontal grid spacing between each x-coordinate line
in the discrete grid. The syntax indicates that the discrete grid
should be comprise of \$60\$ grid lines each separated by \$20 times 10\textasciicircum{}\{3\}m\$
(\$20\$\textasciitilde{}km).

item Line 42,
begin\{verbatim\}
delY=60*20E3,
end\{verbatim\}
This line sets the horizontal grid spacing between each y-coordinate line
in the discrete grid to \$20 times 10\textasciicircum{}\{3\}m\$ (\$20\$\textasciitilde{}km).

item Line 43,
begin\{verbatim\}
delZ=5000,
end\{verbatim\}
This line sets the vertical grid spacing between each z-coordinate line
in the discrete grid to \$5000m\$ (\$5\$\textasciitilde{}km).

item Line 46,
begin\{verbatim\}
bathyFile='topog.box'
end\{verbatim\}
This line specifies the name of the file from which the domain
bathymetry is read. This file is a two-dimensional (\$x,y\$) map of
depths. This file is assumed to contain 64-bit binary numbers
giving the depth of the model at each grid cell, ordered with the x
coordinate varying fastest. The points are ordered from low coordinate
to high coordinate for both axes. The units and orientation of the
depths in this file are the same as used in the MITgcm code. In this
experiment, a depth of \$0m\$ indicates a solid wall and a depth
of \$-5000m\$ indicates open ocean. The matlab program
\{it input/gendata.m\} shows an example of how to generate a
bathymetry file.

item Line 49,
begin\{verbatim\}
zonalWindFile='windx.sin\_y'
end\{verbatim\}
This line specifies the name of the file from which the x-direction
surface wind stress is read. This file is also a two-dimensional
(\$x,y\$) map and is enumerated and formatted in the same manner as the
bathymetry file. The matlab program \{it input/gendata.m\} includes example
code to generate a valid \{bf zonalWindFile\} file.

end\{itemize\}

noindent other lines in the file \{it input/data\} are standard values
that are described in the MITgcm Getting Started and MITgcm Parameters
notes.

\begin{sphinxVerbatim}[commandchars=\\\{\}]
\PYG{c+c1}{\PYGZsh{} Model parameters}
\PYG{c+c1}{\PYGZsh{} Continuous equation parameters}
 \PYG{o}{\PYGZam{}}\PYG{n}{PARM01}
 \PYG{n}{tRef}\PYG{o}{=}\PYG{l+m+mf}{20.}\PYG{p}{,}
 \PYG{n}{sRef}\PYG{o}{=}\PYG{l+m+mf}{10.}\PYG{p}{,}
 \PYG{n}{viscAz}\PYG{o}{=}\PYG{l+m+mf}{1.E\PYGZhy{}2}\PYG{p}{,}
 \PYG{n}{viscAh}\PYG{o}{=}\PYG{l+m+mf}{4.E2}\PYG{p}{,}
 \PYG{n}{diffKhT}\PYG{o}{=}\PYG{l+m+mf}{4.E2}\PYG{p}{,}
 \PYG{n}{diffKzT}\PYG{o}{=}\PYG{l+m+mf}{1.E\PYGZhy{}2}\PYG{p}{,}
 \PYG{n}{beta}\PYG{o}{=}\PYG{l+m+mf}{1.E\PYGZhy{}11}\PYG{p}{,}
 \PYG{n}{tAlpha}\PYG{o}{=}\PYG{l+m+mf}{2.E\PYGZhy{}4}\PYG{p}{,}
 \PYG{n}{sBeta} \PYG{o}{=}\PYG{l+m+mf}{0.}\PYG{p}{,}
 \PYG{n}{gravity}\PYG{o}{=}\PYG{l+m+mf}{9.81}\PYG{p}{,}
 \PYG{n}{gBaro}\PYG{o}{=}\PYG{l+m+mf}{9.81}\PYG{p}{,}
 \PYG{n}{rigidLid}\PYG{o}{=}\PYG{o}{.}\PYG{n}{FALSE}\PYG{o}{.}\PYG{p}{,}
 \PYG{n}{implicitFreeSurface}\PYG{o}{=}\PYG{o}{.}\PYG{n}{TRUE}\PYG{o}{.}\PYG{p}{,}
 \PYG{n}{eosType}\PYG{o}{=}\PYG{l+s+s1}{\PYGZsq{}}\PYG{l+s+s1}{LINEAR}\PYG{l+s+s1}{\PYGZsq{}}\PYG{p}{,}
 \PYG{n}{readBinaryPrec}\PYG{o}{=}\PYG{l+m+mi}{64}\PYG{p}{,}
 \PYG{o}{\PYGZam{}}

\PYG{c+c1}{\PYGZsh{} Elliptic solver parameters}
 \PYG{o}{\PYGZam{}}\PYG{n}{PARM02}
 \PYG{n}{cg2dMaxIters}\PYG{o}{=}\PYG{l+m+mi}{1000}\PYG{p}{,}
 \PYG{n}{cg2dTargetResidual}\PYG{o}{=}\PYG{l+m+mf}{1.E\PYGZhy{}7}\PYG{p}{,}
 \PYG{o}{\PYGZam{}}

\PYG{c+c1}{\PYGZsh{} Time stepping parameters}
 \PYG{o}{\PYGZam{}}\PYG{n}{PARM03}
 \PYG{n}{startTime}\PYG{o}{=}\PYG{l+m+mi}{0}\PYG{p}{,}
\PYG{c+c1}{\PYGZsh{}endTime=311040000,}
 \PYG{n}{endTime}\PYG{o}{=}\PYG{l+m+mf}{12000.0}\PYG{p}{,}
 \PYG{n}{deltaTmom}\PYG{o}{=}\PYG{l+m+mf}{1200.0}\PYG{p}{,}
 \PYG{n}{deltaTtracer}\PYG{o}{=}\PYG{l+m+mf}{1200.0}\PYG{p}{,}
 \PYG{n}{abEps}\PYG{o}{=}\PYG{l+m+mf}{0.1}\PYG{p}{,}
 \PYG{n}{pChkptFreq}\PYG{o}{=}\PYG{l+m+mf}{2592000.0}\PYG{p}{,}
 \PYG{n}{chkptFreq}\PYG{o}{=}\PYG{l+m+mf}{120000.0}\PYG{p}{,}
 \PYG{n}{dumpFreq}\PYG{o}{=}\PYG{l+m+mf}{2592000.0}\PYG{p}{,}
 \PYG{n}{monitorSelect}\PYG{o}{=}\PYG{l+m+mi}{2}\PYG{p}{,}
 \PYG{n}{monitorFreq}\PYG{o}{=}\PYG{l+m+mf}{1.}\PYG{p}{,}
 \PYG{o}{\PYGZam{}}

\PYG{c+c1}{\PYGZsh{} Gridding parameters}
 \PYG{o}{\PYGZam{}}\PYG{n}{PARM04}
 \PYG{n}{usingCartesianGrid}\PYG{o}{=}\PYG{o}{.}\PYG{n}{TRUE}\PYG{o}{.}\PYG{p}{,}
 \PYG{n}{usingSphericalPolarGrid}\PYG{o}{=}\PYG{o}{.}\PYG{n}{FALSE}\PYG{o}{.}\PYG{p}{,}
 \PYG{n}{delX}\PYG{o}{=}\PYG{l+m+mi}{60}\PYG{o}{*}\PYG{l+m+mi}{20}\PYG{n}{E3}\PYG{p}{,}
 \PYG{n}{delY}\PYG{o}{=}\PYG{l+m+mi}{60}\PYG{o}{*}\PYG{l+m+mi}{20}\PYG{n}{E3}\PYG{p}{,}
 \PYG{n}{delZ}\PYG{o}{=}\PYG{l+m+mf}{5000.}\PYG{p}{,}
 \PYG{o}{\PYGZam{}}

\PYG{c+c1}{\PYGZsh{} Input datasets}
 \PYG{o}{\PYGZam{}}\PYG{n}{PARM05}
 \PYG{n}{bathyFile}\PYG{o}{=}\PYG{l+s+s1}{\PYGZsq{}}\PYG{l+s+s1}{topog.box}\PYG{l+s+s1}{\PYGZsq{}}\PYG{p}{,}
 \PYG{n}{hydrogThetaFile}\PYG{o}{=}\PYG{p}{,}
 \PYG{n}{hydrogSaltFile}\PYG{o}{=}\PYG{p}{,}
 \PYG{n}{zonalWindFile}\PYG{o}{=}\PYG{l+s+s1}{\PYGZsq{}}\PYG{l+s+s1}{windx.sin\PYGZus{}y}\PYG{l+s+s1}{\PYGZsq{}}\PYG{p}{,}
 \PYG{n}{meridWindFile}\PYG{o}{=}\PYG{p}{,}
 \PYG{o}{\PYGZam{}}
\end{sphinxVerbatim}

begin\{small\}
input\{s\_examples/barotropic\_gyre/input/data\}
end\{small\}

subsubsection\{File \{it input/data.pkg\}\}
\%label\{www:tutorials\}

This file uses standard default values and does not contain
customizations for this experiment.

subsubsection\{File \{it input/eedata\}\}
\%label\{www:tutorials\}

This file uses standard default values and does not contain
customizations for this experiment.

subsubsection\{File \{it input/windx.sin\_y\}\}
\%label\{www:tutorials\}

The \{it input/windx.sin\_y\} file specifies a two-dimensional (\$x,y\$)
map of wind stress ,\$tau\_\{x\}\$, values. The units used are \$Nm\textasciicircum{}\{-2\}\$.
Although \$tau\_\{x\}\$ is only a function of \$y\$n in this experiment
this file must still define a complete two-dimensional map in order
to be compatible with the standard code for loading forcing fields
in MITgcm. The included matlab program \{it input/gendata.m\} gives a complete
code for creating the \{it input/windx.sin\_y\} file.

subsubsection\{File \{it input/topog.box\}\}
\%label\{www:tutorials\}

The \{it input/topog.box\} file specifies a two-dimensional (\$x,y\$)
map of depth values. For this experiment values are either
\$0m\$ or \{bf -delZ\}m, corresponding respectively to a wall or to deep
ocean. The file contains a raw binary stream of data that is enumerated
in the same way as standard MITgcm two-dimensional, horizontal arrays.
The included matlab program \{it input/gendata.m\} gives a complete
code for creating the \{it input/topog.box\} file.

subsubsection\{File \{it code/SIZE.h\}\}
\%label\{www:tutorials\}

Two lines are customized in this file for the current experiment

begin\{itemize\}

item Line 39,
begin\{verbatim\} sNx=60, end\{verbatim\} this line sets
the lateral domain extent in grid points for the
axis aligned with the x-coordinate.

item Line 40,
begin\{verbatim\} sNy=60, end\{verbatim\} this line sets
the lateral domain extent in grid points for the
axis aligned with the y-coordinate.

end\{itemize\}

begin\{small\}
input\{s\_examples/barotropic\_gyre/code/SIZE.h\}
end\{small\}

subsubsection\{File \{it code/CPP\_OPTIONS.h\}\}
\%label\{www:tutorials\}

This file uses standard default values and does not contain
customizations for this experiment.

subsubsection\{File \{it code/CPP\_EEOPTIONS.h\}\}
\%label\{www:tutorials\}

This file uses standard default values and does not contain
customizations for this experiment.

newpage
input\{s\_examples/baroclinic\_gyre/fourlayer.tex\}

newpage
input\{s\_examples/advection\_in\_gyre/adv\_gyre.tex\}

newpage
input\{s\_examples/global\_oce\_latlon/climatalogical\_ogcm.tex\}

newpage
input\{s\_examples/global\_oce\_in\_p/ogcm\_in\_pressure.tex\}

newpage
input\{s\_examples/held\_suarez\_cs/held\_suarez\_cs.tex\}

newpage
input\{s\_examples/deep\_convection/convection.tex\}

newpage
input\{s\_examples/plume\_on\_slope/plume\_on\_slope.tex\}

newpage
input\{s\_examples/global\_oce\_biogeo/biogeochem.tex\}

newpage
input\{s\_examples/global\_oce\_optim/global\_oce\_estimation.tex\}

newpage
input\{s\_examples/tracer\_adjsens/doc\_ad\_examples.tex\}

newpage
input\{s\_examples/cfc\_offline/offline\_tutorial.tex\}

newpage
input\{s\_examples/rotating\_tank/tank.tex\}
\phantomsection\label{\detokenize{zreferences:references}}


\begin{sphinxthebibliography}{ACHM04}
\bibitem[AC04]{\detokenize{AC04}}{\phantomsection\label{\detokenize{zreferences:adcroft-04a}} 
A. Adcroft and J.-M. Campin. Re-scaled height coordinates for accurate representation of free-surface flows in ocean circulation models. \sphinxstyleemphasis{Ocean Modelling}, 7:269\textendash{}284, 2004. \sphinxhref{https://doi.org/10.1016/j.ocemod.2003.09.003}{doi:10.1016/j.ocemod.2003.09.003}.
}
\bibitem[ACHM04]{\detokenize{ACHM04}}{\phantomsection\label{\detokenize{zreferences:adcroft-04b}} 
A. Adcroft, J.-M. Campin, C. Hill, and J. Marshall. Implementation of an atmosphere-ocean general circulation model on the expanded spherical cube. \sphinxstyleemphasis{Mon.\textasciitilde{}Wea.\textasciitilde{}Rev.}, 132:2845\textendash{}2863, 2004. URL: \sphinxurl{http://mitgcm.org/pdfs/mwr\_2004.pdf}, \sphinxhref{https://doi.org/10.1175/MWR2823.1}{doi:10.1175/MWR2823.1}.
}
\bibitem[AHC+04]{\detokenize{AHC+04}}{\phantomsection\label{\detokenize{zreferences:adcroft-04c}} 
A. Adcroft, C. Hill, J.-M. Campin, J. Marshall, and P. Heimbach. Overview of the formulation and numerics of the MITgcm. In \sphinxstyleemphasis{Proceedings of the ECMWF seminar series on Numerical Methods, Recent developments in numerical methods for atmosphere and ocean modelling}, 139\textendash{}149. ECMWF, 2004. URL: \sphinxurl{http://mitgcm.org/pdfs/ECMWF2004-Adcroft.pdf}.
}
\bibitem[AHM97]{\detokenize{AHM97}}{\phantomsection\label{\detokenize{zreferences:adcroft-97}} 
A.J. Adcroft, C.N. Hill, and J. Marshall. Representation of topography by shaved cells in a height coordinate ocean model. \sphinxstyleemphasis{Mon.\textasciitilde{}Wea.\textasciitilde{}Rev.}, 125:2293\textendash{}2315, 1997. URL: \sphinxurl{http://mitgcm.org/pdfs/mwr\_1997.pdf}, \sphinxhref{https://doi.org/10.1175/1520-0493\%281997\%29125\textless{}2293:ROTBSC\textgreater{}2.0.CO;2}{doi:10.1175/1520-0493\%281997\%29125\textless{}2293:ROTBSC\textgreater{}2.0.CO;2}.
}
\bibitem[AM99]{\detokenize{AM99}}{\phantomsection\label{\detokenize{zreferences:adcroft-99}} 
Hill C. Adcroft, A. and J. Marshall. A new treatment of the coriolis terms in c-grid models at both high and low resolutions. \sphinxstyleemphasis{Mon.\textasciitilde{}Wea.\textasciitilde{}Rev.}, 127:1928\textendash{}1936, 1999. URL: \sphinxurl{http://mitgcm.org/pdfs/mwr\_1999.pdf}, \sphinxhref{https://doi.org/10.1175/1520-0493\%281999\%29127\textless{}1928:ANTOTC\textgreater{}2.0.CO;2}{doi:10.1175/1520-0493\%281999\%29127\textless{}1928:ANTOTC\textgreater{}2.0.CO;2}.
}
\bibitem[CHM99]{\detokenize{CHM99}}{\phantomsection\label{\detokenize{zreferences:hill-99}} 
Daniel Jamous Chris Hill, Alistair Adcroft and John Marshall. A strategy for terascale climate modeling. In \sphinxstyleemphasis{In Proceedings of the Eighth ECMWF Workshop on the Use of Parallel Processors in Meteorology}, 406\textendash{}425. World Scientific, 1999.
}
\bibitem[HM95]{\detokenize{HM95}}{\phantomsection\label{\detokenize{zreferences:hill-95}} 
C. Hill and J. Marshall. Application of a parallel navier-stokes model to ocean circulation in parallel computational fluid dynamics. In N. Satofuka A. Ecer, J. Periaux and S. Taylor, editors, \sphinxstyleemphasis{Implementations and Results Using Parallel Computers}, pages 545\textendash{}552. Elsevier Science B.V.: New York, 1995.
}
\bibitem[MGZ+99]{\detokenize{MGZ+99}}{\phantomsection\label{\detokenize{zreferences:maro-eta-99}} 
J. Marotzke, R. Giering, K.Q. Zhang, D. Stammer, C. Hill, and T. Lee. Construction of the adjoint mit ocean general circulation model and application to atlantic heat transport variability. \sphinxstyleemphasis{J.\textasciitilde{}Geophys.\textasciitilde{}Res.}, 104, C12:29,529\textendash{}29,547, 1999.
}
\bibitem[MAC+04]{\detokenize{MAC+04}}{\phantomsection\label{\detokenize{zreferences:marshall-04}} 
J. Marshall, A. Adcroft, J.-M. Campin, C. Hill, and A. White. Atmosphere-ocean modeling exploiting fluid isomorphisms. \sphinxstyleemphasis{Mon.\textasciitilde{}Wea.\textasciitilde{}Rev.}, 132:2882\textendash{}2894, 2004. URL: \sphinxurl{http://mitgcm.org/pdfs/a\_o\_iso.pdf}, \sphinxhref{https://doi.org/10.1175/MWR2835.1}{doi:10.1175/MWR2835.1}.
}
\bibitem[MAH+97]{\detokenize{MAH+97}}{\phantomsection\label{\detokenize{zreferences:marshall-97b}} 
J. Marshall, A. Adcroft, C. Hill, L. Perelman, and C. Heisey. A finite-volume, incompressible navier stokes model for studies of the ocean on parallel computers. \sphinxstyleemphasis{J.\textasciitilde{}Geophys.\textasciitilde{}Res.}, 102(C3):5753\textendash{}5766, 1997. URL: \sphinxurl{http://mitgcm.org/pdfs/96JC02776.pdf}.
}
\bibitem[MHPA97]{\detokenize{MHPA97}}{\phantomsection\label{\detokenize{zreferences:marshall-97a}} 
J. Marshall, C. Hill, L. Perelman, and A. Adcroft. Hydrostatic, quasi-hydrostatic, and nonhydrostatic ocean modeling. \sphinxstyleemphasis{J.\textasciitilde{}Geophys.\textasciitilde{}Res.}, 102(C3):5733\textendash{}5752, 1997. URL: \sphinxurl{http://mitgcm.org/pdfs/96JC02775.pdf}.
}
\bibitem[MJH98]{\detokenize{MJH98}}{\phantomsection\label{\detokenize{zreferences:mars-eta-98}} 
J. Marshall, H. Jones, and C. Hill. Efficient ocean modeling using non-hydrostatic algorithms. \sphinxstyleemphasis{J.\textasciitilde{}Mar.\textasciitilde{}Sys.}, 18:115\textendash{}134, 1998. URL: \sphinxurl{http://mitgcm.org/pdfs/journal\_of\_marine\_systems\_1998.pdf}, \sphinxhref{https://doi.org/10.1016/S0924-7963\%2898\%2900008-6}{doi:10.1016/S0924-7963\%2898\%2900008-6}.
}
\end{sphinxthebibliography}



\renewcommand{\indexname}{Index}
\printindex
\end{document}